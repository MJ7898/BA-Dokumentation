\chapter{Fazit und Ausblick}
\section{Fazit}
Das Ziel dieser Arbeit war die Grundlage und Architekturkonzeption einer Software zur Erstellung von Schaltplänen zu erarbeiten. Dafür wurden in den ersten Schritten die genauen Anforderungen an das System festgelegt. Darauf basierend wurden mit C\#, .NET und WPF die Technologien ausgewählt. Um eine Modularität zu gewährleisten, damit auch künfigt an der Software weiterentwickelt werden kann, musste ein geeignetes Entwurfsmuster gefunden werden, nach dem die Architektur aufgebaut wird. Mit MVVM wurde ein passendes Muster, dass vorallem für WPF und C\# besonders geeignet ist gefunden. Bevor es in die konkrete Umsetzung ging, wurde die Architektur anhand von Use Cases und Klassendiagrammen modeliert. Anhand dieser Überlegungen konnte dann die Architektur implementiert werden. Eine große Rolle für die Implementierung auf Basis des MVVM Musters hat dabei das MVVM Light Framework gespielt. 
\\
\linebreak
Aufgrund der hohen Entkopplung zwischen den einzelnen Modulen und speziell der Benutzeroberfläche und Logik, wird eine einfachere Wartbarkeit sowie Erweiterbarkeit des Projekts gewährleistet. Aber auch die Möglichkeit die einzelnen Bestandteile zu testen, ist durch die hohe Entkopplung möglich. Neben den Vorteilen bringt die Moldularität auch Nachteile bei der Entwicklung. So traten immer wieder Schwierigkeiten und komplexe Probleme auf, insbesondere in der Verbindung zwischen der Benutzeroberfläche und der Logik. In einzelnen Fällen mussten Synchronisierungen zwischen Benutzeroberfläche und Logik durch Hilfsklassen unterstützt werden. Auch für die Anzeige bestimmter Elemente mussten Converter Klassen geschrieben werden. Änderungen in der Projektstruktur haben teilweise zu komplexen Verschiebungen, vorallem in den Namensräumen innherlab der Visual Studio IDE geführt. Daher wird nicht empfohlen die Struktur in der künftigen Entwicklung groß zu ändern. In dieser Arbeit wurde eine fundierte Grundlage geschaffen, auf der eine Weiterentwicklung stattfinden kann. 
\\
\linebreak
Im gesamten betrachtet bietet das Ergebnis dieser Arbeit einen klaren und leicht verständlichen Systemaufbau und die Grundlage für Weiterentwicklungen. Hält man sich an das vorgegebene Muster sowie die bisherig implementierte Struktur des Projekts, gelingt das Hinzufügen weiterer Funktionen einfach und erfolgreich.
\\
\section{Ausblick}
Bisher bietet die Software die Möglichkeit neue Projekte anzulegen. Dabei erzeugt das System automatisch eine Projektdatei. Nachdem ein Projekt angelegt wurde, öffnet sich der Editor. Er stellt die Eigentliche Funktionalität des Programms dar. Der Editor hat die grafische Struktur für alle nötigen Funktionen implementiert. Eine Struktur der Komponenten für die Toolbox der Elektro-Komponenten liegt ebenfalls vor und kann beliebig erweitert werden. Auch die Zeichenfläche ist verfügbar. Sie erkennt die Mausposition und registriert einen Klick. Dabei wird die Position gespeichert und anhand der zwei Positionen, eine Linie gezeichnet. Die Linie existiert im Programmcode, wird allerdings bislang noch nicht an der Benutzeroberfläche angezeigt.
\\ 
In weiteren Entwicklungen können die Funktionalitäten erweitert werden. Auf Basis des Systems kann ein Speicherverfahren implementiert werden, dass die Zeichnungen innerhalb der vorhandenen Text-Datei abspeichert. Die Möglichkeit Schaltpläne zu zeichnen kann ebenfalls Funktional erweitert werden. So ist das langfristige Ziel vollständige 2D Schaltpläne zu entwerfen, die in einem fertigen Format (z.B. PDF) exportiert werden können. Eine weitere Konzeption sieht vor, die Schaltpläne in einer 3D Sicht abzubilden um festzustellen, wo genau innerhalb eines Gebäudes bestimmte Komponenten angebracht werden können. Abschließend lässt sich sagen, dass die Grundlage und vorallem eine grundlegene Architektur innerhalb dieser Arbeit erstellt und entwickelt wurden. Diese können als ausreichende Basis für zukünftige Entwicklungen genutzt werden um ein vollständiges, für den Anwender bestimmtes Programm zu erstellen.
