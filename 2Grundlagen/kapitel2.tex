%%%%%%%%%%%%%%%%%%%%%%%%%%%%%%%%%%%%%%%%%%%%%%%%%%%%%%%%%%%%%%%%%%%%%%%%%%%%%
%% Descr:       Vorlage für Berichte der DHBW-Karlsruhe, Ein Kapitel
%% Author:      Prof. Dr. Jürgen Vollmer, vollmer@dhbw-karlsruhe.de
%% $Id: kapitel2.tex,v 1.5 2017/10/06 14:02:51 vollmer Exp $
%%  -*- coding: utf-8 -*-
%%%%%%%%%%%%%%%%%%%%%%%%%%%%%%%%%%%%%%%%%%%%%%%%%%%%%%%%%%%%%%%%%%%%%%%%%%%%%%%

\chapter{Grundlagen}
\label{chap:Grundlagen}
In diesem Kapitel werden die für diese Bachelorarbeit notwendigen Grundlagen geschaffen, um ein fundiertes Wissen und Verständnis 
über verwendete Technologien zu schaffen. Auf alle diese Informationen und Voraussetzungen wird im Folgenden eingegangen, um nachfolgende 
Konzeption und Umsetzung zu verstehen.

\section{Augmented Reality}
\label{chap:Augmented Reality}
Eine der wichtigsten Grundlagen dieser Arbeit ist das Verständnis des Begriffs der Augmented Reality.
\\ 
\acl{AR}, deutsch erweiterte Realität, ist eine durch den Computer gestützte Erweiterung der Realität, bzw. des menschlichen 
Wahrnehmungsvermögens. Die Definition, welche sich in der Wissenschaft weitestgehend durchgesetzt und etabliert hat ist die Definition nach 
Azuma aus dem Jahre 1997.
\begin{quote}
    „Augmented Reality (AR) is a variation of Virtual Environments (VE), or Virtual Reality as it is more commonly called. VE 
    technologies completely immerse a user inside a synthetic environment. While immersed, the user cannot see the real world around him. 
    In contrast, AR allows the user to see the real world, with virtual objects superimposed upon or composited with the real world. Therefore, 
    AR supplements reality, rather than completely replacing it.“ \cite{azuma.1997a}
\end{quote}



\subsection{Virtual Reality, Augmented Reality und Mixed Reality}
\subsection{Varianten der Augmented Reality}
\subsection{Augmented Reality in der Industrie}

\section{SLAM - Simultanious Localization And Mapping}
\label{chap:SLAM}
\subsection{Localization}
\subsection{Mapping}

\section{Quaternionen}
\label{chap:Quaternionen}
\subsection{Transformation}
\subsection{Rotation}

\section{Technologien}
\label{chap:Technologien}
\subsection{Google ARCore}
\subsection{Android Jetpack}
\subsection{Sceneform SDK}
\subsection{SQLite}
%\section{Sceneform SDK}
%\subsection{Physically Based Renderer PBR}
%\subsection{Filament}

