%%%%%%%%%%%%%%%%%%%%%%%%%%%%%%%%%%%%%%%%%%%%%%%%%%%%%%%%%%%%%%%%%%%%%%%%%%%%%%
%% Descr:       Vorlage für Berichte der DHBW-Karlsruhe, Ein Kapitel
%% Author:      Prof. Dr. Jürgen Vollmer, vollmer@dhbw-karlsruhe.de
%% $Id: kapitel1.tex,v 1.17 2018/10/23 08:58:41 vollmer Exp $
%% -*- coding: utf-8 -*-
%%%%%%%%%%%%%%%%%%%%%%%%%%%%%%%%%%%%%%%%%%%%%%%%%%%%%%%%%%%%%%%%%%%%%%%%%%%%%%%

\chapter{Einleitung}
In diesem Teil der Arbeit wird auf die Motivation des Themas eingegangen. Darüber hinaus wird sowohl die Aufgabenstellung als auch der 
Aufbau der Arbeit genauesten dargelegt. Eine nähere Betrachtung des Standes der Technik untermauert die Beweggründe der Ausarbeitung 
dieser Arbeit. 
\section{Motivation}
Jede neu entwickelte Technologie durchlebt im Laufe der Entstehung ein enormes Aufsehen. Es wird viel darüber debattiert, fantasiert 
und geplant ohne jedoch genau die Resultate abwägen zu können. Durch fehlende Erfahrung und nicht ausgereifte Konzepte werden Highlights 
erwartet die zu diesem Zeitpunkt technisch nicht umsetzbar sind. Jede neue technologische Idee macht diese Phasen der Entwicklung durch. 
\\ 
\linebreak
Ein sogenannter Hype Cycle, dt. Hype-Zyklus, ist ein visualisiertes Modell, das die Entwicklung einer neuen Technologie von der 
Innovation über die Umsetzung bis hin zur ausgereiften Marktfähigkeit repräsentiert und so die Phasen der Entwicklung verdeutlicht. 
Nachdem eine Innovation den Gipfel der überzogenen Erwartungen passiert hat, folgt das Tal der Enttäuschung, wobei die Technologie 
an Interesse verliert. Nach der erneuten Sammlung, der \textit{"Kurs-Korrektur"} \cite{hypecycle.2019o}, wird die präsente Innovation 
realistischer beurteilt. Durch die objektive Betrachtungsweise entsteht ein neues und realistisches Bild der Möglichkeiten, aber auch 
der Grenzen der Technologie. Zum Ende hin geht die ehemals neue Innovation in eine routinierte Technologie über, wobei diese an 
Anerkennung gewinnt und sich weiterentwickelt. Diese Position des Modells signalisiert und bestätigt die Marktreife einer Technologie 
und wir ab diesem Zeitpunkt nichtmehr als Zukunftsvision, Hype oder Highlight angesehen.  



Weiterentwicklung, der immer größer werdenden Nutzung und dem ausgänglichen Nutzen von Augmented Reality 


\section{cjt Systemsoftware AG}
Die Arbeit wurde bei der Firma cjt Systemsoftware AG durchgeführt. Diese wurde
1999 von Christian J. Tauber und Ulrich Beck gegründet. Damals mit einem Team
von 20 Personen, beschäftigt die cjt Systemsoftware AG heute mehr als 60 Mitarbeiter. 
Mit ihrem Sitz in Karlsruhe ist sie in einer der größten Technologiestädten Deutschlands angesiedelt.
\\
\linebreak
Durch das stetige Wachstum der cjt Systemsoftware AG vergrößert sich auch deren
Portfolio kontinuierlich. Dabei setzt das Consulting-Unternehmen hauptsächlich auf maßgeschneiderte
Software- und Netzwerklösungen. Großkunden wie Siemens AG, Lufthansa Cargo,
Forschungszentrum Karlsruhe (KIT) und Fraunhofer IOSB zeugen von der hohen Qualität der geleisteten Arbeit. 
Dabei agiert das Unternehmen nicht nur in Deutschland sondern auch international, darunter in Ländern wie China und den USA.

\section{Aufgabenstellung}
\section{Aufbau der Arbeit}
\section{Stand der Technik}