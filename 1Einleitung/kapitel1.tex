%%%%%%%%%%%%%%%%%%%%%%%%%%%%%%%%%%%%%%%%%%%%%%%%%%%%%%%%%%%%%%%%%%%%%%%%%%%%%%
%% Descr:       Vorlage für Berichte der DHBW-Karlsruhe, Ein Kapitel
%% Author:      Prof. Dr. Jürgen Vollmer, vollmer@dhbw-karlsruhe.de
%% $Id: kapitel1.tex,v 1.17 2018/10/23 08:58:41 vollmer Exp $
%% -*- coding: utf-8 -*-
%%%%%%%%%%%%%%%%%%%%%%%%%%%%%%%%%%%%%%%%%%%%%%%%%%%%%%%%%%%%%%%%%%%%%%%%%%%%%%%

\chapter{Einleitung}
\label{chap:Einleitung}
In diesem Teil der Arbeit wird auf die Motivation des Themas eingegangen. Darüber hinaus wird sowohl die Aufgabenstellung als auch der 
Aufbau der Arbeit genauestens dargelegt. Eine nähere Betrachtung des Standes der Technik untermauert die Beweggründe dieser Ausarbeitung.  

\section{Motivation}
\label{chap:Motivation}
Jede neu entwickelte Technologie durchlebt im Laufe der Entstehung ein enormes Aufsehen. Es wird viel darüber debattiert, fantasiert 
und geplant ohne jedoch genau die Resultate abwägen zu können. Durch fehlende Erfahrung und nicht ausgereifte Konzepte werden Highlights 
erwartet, die zu diesem Zeitpunkt technisch nicht umsetzbar sind. Jede neue technologische Idee durchläuft bestimmte Phasen der Entwicklung.
\cite{studiob12.2020j} 
\\ 
\linebreak
Ein sogenannter Hype Cycle, dt. Hype-Zyklus, ist ein visualisiertes Modell, das die Entwicklung einer neuen Technologie von der 
Innovation über die Umsetzung bis hin zur ausgereiften Marktfähigkeit repräsentiert und so diese Phasen der Entwicklung verdeutlicht.
\\
Der Hype Cycle, entwickelt von der Forschungsgruppe Gartner Inc. und die durch deren Mitarbeiterin Jackie Fenn geprägten Definitionen, ist 
in fünf Entwicklungsphasen untergliedert:
\begin{enumerate}
    \item \textit{Technologische Auslösung:} Bekanntwerden der Technologie und erste Projekte, die auf beachtliches Interesse in der 
    Öffentlichkeit stoßen. 
    \item \textit{Gipfel der überzogenen Erwartungen:} Die Technologie ist auf dem Gipfel der Aufmerksamkeit. Es werden unrealistische 
    und enthusiastische Erwartungen veröffentlicht und diskutiert.
    \item \textit{Tal der Enttäuschung:} Da die Technologie nicht die zuvor aufgebauten Erwartungen erfüllen kann, sinkt die Aufmerksamkeit 
    der enttäuschten Enthusiasten und der Medienvertreter auf den Tiefpunkt.
    \item \textit{Pfad der Erleuchtung:} Die neue Technologie wird in dieser Phase realistisch mit ihren Stärken und Schwächen betrachtet.
    \item \textit{Plateau der Produktivität:} Die neuen Möglichkeiten durch die Technologie werden als Vorteile akzeptiert, und 
    Geschäftsprozesse werden entwickelt. \cite{gartnerhc.2016s}
\end{enumerate} 
%Bsp. welche Erwartungen und welche darauffolgenden Enttäuschungen \listoftodos{bsps. für Erwartungen}
Nachdem eine Innovation den Gipfel der überzogenen Erwartungen passiert hat, z.B. die vollständige Revolutionierung der Industrie oder Szenarien, wie z.B. 
Hologramme, die man nur aus Science-Fiction Filmen kennt, 
folgt das Tal der Enttäuschung. Dabei wird festgestellt, dass die Erwartungen nicht übertragbar sind, bzw. nur zum Teil in die Realität 
umgesetzt werden können und die Technologie an Interesse verliert. Nach der erneuten Sammlung, der \textit{"Kurs-Korrektur"} 
\cite{hypecycle.2019o}, wird die präsente Innovation realistischer beurteilt. 
Durch die objektive und realitätsnähere Betrachtungsweise entsteht ein neues und realistisches Bild der Möglichkeiten, als auch 
der Grenzen der Technologie. Zum Ende hin geht die ehemals neue Innovation in eine routinierte Technologie über, wobei diese an 
Anerkennung gewinnt und sich bewährt. Bei den Nutzern findet sie immer mehr Zuspruch und zu Beginn utopisch gedachte Konzepte sind
längst einer realen Praxis gewichen. Durch die steigende Zuwendung zu der Technologie erfährt diese eine stetige Weiterentwicklung
und die ursprünglich kleine Nutzergruppe kann zu einer großen Community wachsen. Bezogen auf die Phasen des Hype Cycles befindet sich diese 
Technologie dann an der Stelle des Plateaus der Produktivität und bestätigt die Marktreife. Ab diesem Zeitpunkt handelt es sich nicht mehr 
um eine Zukunftsvision, Hype oder Highlight, sondern um eine am Markt etablierte Technologie.
\\ 
\linebreak                                      % Name der Phase, nicht religiös zu verstehen!!!!!! 
Momentan befindet sich Augmented Reality auf dem Pfad der Erleuchtung und ist auf sehr gutem Wege zu einer ausgereiften Technologie, 
da das Wachstum der Verwendung von Augmented Reality stetig steigt und mittlerweile ein weites Portfolio an möglichen Einsatzgebieten 
vorweist. Die jetzige Erfahrung und der technologische Fortschritt bringt Augmented Reality den ursprünglich angedachten Visionen und Ideen 
einen Schritt näher, sodass in naher Zukunft die letzte Phase, das Erreichen des Plateaus der Produktivität erzielt werden kann. Den finalen 
Schritt der endgültigen Marktreife zu erlangen ist ein faszinierender und wichtiger Grund für meine Motivation mich dieser Technologie und 
der dahinterstehenden Theorie zu widmen. 
Die Technologie der Augmented Reality weist, wie bereits erwähnt, ein immer größer werdendes Portfolio an Einsatzgebieten vor, unter anderem: 
\begin{itemize}
    \item Industrie, Wartung und Reparatur
    \item Produktion und Lagerlogistik
    %\item Wartung und Reparatur
    \item Spiele, bzw. Gaming und Unterhaltung
    \item Medizin
    \item Marketing, Schulung und Training %und Werbung
    %\item Navigation
    %\item Unterhaltung und Fernsehen 
    %\item Schulung und Training
    \item Militär
\end{itemize} 
Neben der Affinität der Technologie bringt \acs{AR} Vorteile mit sich, z.B. Arbeitsprozesse und Herangehensweisen an Projekte zu 
modernisieren. Ebenso wird die Fehleranfälligkeit durch menschliche Unachtsamkeit mit einer Kombination aus vielen Informationen und 
deren visueller Darstellung in Echtzeit reduziert. Hinzu kommt die Steigerung der Produktivität und Verbesserung des Wissenstransfers 
durch die dargebotenen Angaben die visualisiert werden.
\\ 
\linebreak
Schon im Jahr 2016 sprach Apple CEO Tim Cook die Innovation \textit{Augmented Reality} bei zahlreichen Events, in Keynotes und Interviews 
an und war zu diesem Zeitpunkt schon enorm begeistert davon. Er beteuerte: \textit{„...using the tech would become as normal as eating three 
meals a day.“} \cite{timcook2016.2016o} Bei jeder sich ihm bietenden Möglichkeit ging der Apple CEO auf seine Überzeugung gegenüber \acl{AR} ein. 
Erst vor Kurzem bestätigte er, dass \acs{AR} zahlreiche und innovative Einsatzgebiete erlangen und immer mehr an Wichtigkeit zunehmen 
würde, als er sagte: \textit{„Augmented reality pervades your life, it will play a big role..."} \cite{timcook.2020j} 
\\ 
\linebreak
Ein sehr großer Anwendungsbereich mit viel Potential, das bei weitem noch nicht ausgeschöpft sei, ist der Industriezweig. Die 
Marktstudie zu Augmented Reality der IDG Research Services und PTC untermauern das immer weiter steigende Wachstum der Anwendungen in 
Unternehmen. „Fast \textit{75\%} der deutschen Unternehmen setzen bereits \textit{Virtual oder Augmented Reality} ein oder planen dies." 
\cite{studieptc.2020j} Eine Prognose berechnet einen Umsatz in zweistelliger Milliarden-Höhe unter Einsatz und Anwendung von \acs{AR} der 
bis 2021 erreicht werden soll. So wird umso deutlicher, dass es sich bei Augmented Reality um ein sehr zukunftsorientiertes Marktsegment 
mit viel Potential handelt. Die Technologie rückt immer weiter in den Vordergrund und immer mehr Unternehmen setzen sich damit auseinander.
\\ 
\linebreak
Durch den Trend zur Vollautomatisierung von Produktions- und Industrieprozessen werden Unmengen an Daten erzeugt. Die moderne Industrie 4.0 
steigert das Wachstum der Daten, indem Maschinen immer mehr digitalisiert und Protokolle parallel zur Laufzeit der Anlage erzeugt werden. 
Die entstehenden Daten werden über eine Cloud zur Verfügung gestellt. Dieser Vorgang des Datentransfers ist Teil einer globalen 
Infrastruktur, die es ermöglicht physische und virtuelle Gegenstände miteinander zu vernetzen, auch bekannt als \ac{IoT}.
%aktuelle Thema der Industrie 4.0 und dem damit in Verbindung gebrachte \ac{IoT} werden weitere Daten erzeugt und über eine Cloud zur 
%Verfügung gestellt. 
\\ 
Entstehende Daten sind z.B. Maschinendaten aus Maschinen-Protokollen, Prozessinformationen, Produktionsdaten oder 
Informationen der Endverbraucher. \cite{industrie40.2019f} Augmented Reality ermöglicht es, einen Großteil der Daten sinnvoll zu nutzen, d.h. 
dem Nutzer Auskünfte über Maschinen besonders hervorzuheben. Darüber hinaus können unter anderem durch \acs{AR} veraltete Papierprozesse 
abgelöst oder digitale Lösungen in bestehende IT-Infrastrukturen eingebunden werden. \cite{industrie40ar.2019n} 
Dabei sind Rückschlüsse auf die allgemeine Arbeitsoptimierung zu ziehen. Notwendige Informationen sind ohne 
großen Aufwand, schnell und zentral an einem Ort durch \acs{AR} zur Verfügung gestellt und abrufbar. Damit können 
Wartungen schneller durchgeführt, Defekte besser behoben, Leerläufe oder Fehler am Endprodukt vermieden werden. Mit solch einer Unterstützung 
kann z.B. auch die Orientierung in riesigen Produktionshallen aufrecht erhalten werden, um einen besseren Überblick über alle Maschinen 
zu erhalten. Auch wird die Effizienz von Prozessen erhöht. Ein Mechaniker kann alle notwendigen Informationen, die er benötigt, der 
\acl{AR} Applikation entnehmen, um z.B. eine Wartung einer Anlage schneller durchzuführen oder über Vorfälle an der Maschinerie in Echtzeit 
informiert zu sein. 
\section{cjt Systemsoftware AG}
\label{chap:cjt}
Die Arbeit wurde bei der Firma cjt Systemsoftware AG durchgeführt. Diese wurde
1999 von Christian J. Tauber und Ulrich Beck gegründet. Das Unternehmen beschäftigt mittlerweile mehr als 60 Mitarbeiterinnen und Mitarbeiter.
Mit ihrem Sitz in Karlsruhe ist sie in einer der größten Technologiestädten Deutschlands angesiedelt.
\\
\linebreak
Durch das stetige Wachstum der cjt Systemsoftware AG vergrößert sich auch deren
Portfolio kontinuierlich. Dabei setzt das Consulting-Unternehmen hauptsächlich auf maßgeschneiderte
Software- und Netzwerklösungen. Großkunden wie Siemens AG, Lufthansa Cargo,
\ac{KIT} und Fraunhofer \acs{Fraunhofer IOSB} zeugen von der hohen Qualität der geleisteten Arbeit. 
Dabei agiert das Unternehmen nicht nur in Deutschland, sondern auch international, darunter in Ländern wie China und den USA.
%\\ 
%\linebreak
%Einer der größten Auftraggeber der Firma ist ebenso das in Karlsruhe angesiedelte Unternehmen Siemens AG, dass eine beispielhafte 
%Anwendung für das Assistenzsystem bietet und in Zukunft auch der Firma zur Nutzung unterbreitet werden könnte. 
%\pagebreak
\section{Aufgabenstellung}
\label{chap:Aufgabenstellung}          %konzeptioniert
Im Rahmen dieser Ausarbeitung soll ein System konzipiert, entwickelt und umgesetzt werden, welches basierend auf \acl{AR} ein 
Informations- und Unterstützungssystem im industriellen Bereich grundlegend realisiert.
\\
Die entstehende prototypische Applikation soll es ermöglichen, einen Überblick über eine Produktions- oder Industriehalle zu 
verschaffen, indem die Umgebung und alle in der Halle stehenden Maschinen über die Kamera eines Smart-Devices erkannt und als 
Overlay über das Live-Bild gelegt und angezeigt werden. Die erkannten Objekte können vom Nutzer eingetragen werden, um die Position des 
Gegenstandes festhalten zu können. Nach dem Positionieren des Objektes kann der Nutzer die benötigten Informationen eintragen, um so %Setzen
grundlegende Informationen über diesen Gegenstand griffbereit zu haben. Damit soll der Grundbaustein für ein 
Unterstützungssystem gelegt werden, das beliebig erweiterbar ist. % um ein nützliches \textit{Gadget} mit vielen \textit{Features} 
%in der Industrie zur Verfügung zu stellen.
\\ 
\linebreak
Grundlegend ist das Projekt in drei Aufgabenbereiche unterteilt:
\begin{itemize}
    \item Planung der Architektur
    \item Implementierung der Grundfunktionen der Applikation
    \item Grundlage der Objektinformation als Datenmodell 
\end{itemize}
Um eine Applikation übersichtlich zu gestalten und für die Zukunft Erweiterungen 
möglichst einfach integrieren zu können, ist es zu Anfang die Aufgabe, eine solide und übersichtliche Grundstruktur, bzw. Software-
Architektur zu erstellen. Dadurch werden Funktionen und Klassen einer klaren Struktur zugeordnet und eine einheitliche Linie vorgegeben. 
\\
Ein weiterer Aspekt, der bei der Erstellung der Architektur berücksichtigt werden soll, ist der Ansatz der modularen 
Softwarearchitektur. Damit können einzelne Funktionen unabhängig voneinander getestet und Abhängigkeiten oder Frameworks leichter 
ersetzt, hinzugefügt oder entfernt werden. Darüber hinaus begünstigt eine modulare Konzeption bessere Kontrollierbarkeit und Übersichtlichkeit 
in großen Softwareprojekten.
\\
\linebreak
Die Hauptaufgabe ist die Realisierung der \acl{AR}-Funktion, sozusagen der Kern der Anwendung. Dabei wird die Applikation in zwei 
Phasen unterteilt, welche es separat zu planen und implementieren gilt.
\\ 
\linebreak
Die erste Phase, genannt Scan-Phase, beschäftigt sich mit dem Scannen der Umgebung, bzw. des Raumes. Die Aufgabe dabei besteht darin, 
die Umgebung mittels \acs{SLAM} Verfahren zu realisieren und anhand dieser Informationen virtuelle Objekte im virtuellen Raum setzen zu können. 
Diese dienen als Referenz zu dem existierenden Objekt in der Realität. Im weiteren Verlauf wird darauf genauer eingegangen.
%mittels dem \ac{SLAM} - Verfahren eine Karte der Umgebung zu erstellen und die räumliche Lage innerhalb dieser Karte zu schätzen, um 
%auf Basis dieser erstellten Karte virtuell Objekte auf der Karte platzieren zu können. Mit den gewonnen Informationen der räumlichen 
%Darstellung durch das \acs{SLAM} - Verfahren kann der Nutzer virtuelle Objekte im virtuellen Raum an Ort und Stelle platzieren, als 
%Referenz zu dem existierenden Objekt in der Realität. %, welches erkannt wurde. 
%Bei der Erstellung eines Objekts soll der Nutzer die 
%Möglichkeit haben, Informationen über das Objekt in das System einzupflegen, damit er diese immer abrufen kann. 
\\ 
\linebreak
In der zweiten Phase, genannt Visualisierungs-Phase, sollte der Nutzer die Möglichkeit haben, sich im Raum frei bewegen zu können. Dabei sollen die 
generierten Objekte, die zuvor durch die Scan-Phase erzeugt wurden, angezeigt werden und deren Informationen zuf Verfügung stellen. Im Laufe 
dieser Arbeit wird darauf genauer eingegangen. 
%Mit der Lokalisierung des Nutzer-Geräts und den bekannten Informationen der in Phase eins gesetzten Objekte, sollten dem Nutzer 
%die virtuellen Objekte in seiner unmittelbaren Umgebung angezeigt werden. Mit dem Wissen, dass sich im Blickfeld der \acs{AR}-Applikation 
%ein Objekt befindet, können für dieses in der Datenbank weitere Informationen abgefragt und dem Nutzer zur Verfügung gestellt werden.
\\ 
Ein weiterer wichtiger Punkt dieses Konzeptes wird die Modellierung eines geeigneten, %grundlegenden und
prototypischen Datenmodells sein. 
Dieses Datenmodell gibt vor, welche Informationen in Phase eins beim Erstellen eines Objektes vom Nutzer eingegeben und damit erfasst 
werden sollten. %, um die wichtigsten Daten zur Verfügung zu haben. 
\section{Aufbau der Arbeit}
\label{chap:Aufbau der Arbeit}
Nach den soeben genannten einleitenden Informationen widmet sich das Kapitel (\ref{chap:Grundlagen}) den essentiellen und wichtigsten 
Grundlagen dieser Arbeit. Zu Anfang wird dem Leser der Terminus der \acl{AR} (\ref{chap:Augmented Reality}) offenbart, um allgemein 
Kontexte im Bezug zu dieser Arbeit zu begreifen, gefolgt von einer Einführung in die Thematik des Verfahrens \ac{SLAM}-\acl{SLAM} 
(\ref{chap:SLAM}) der überbegrifflichen Materie der Robotik. Eine weitere notwendige Grundlage ist das Verständnis von Quaternionen 
(\ref{chap:Quaternionen}) und die damit zusammenhängende Rotation und Translation von Objekten in einem dreidimensionalen Raum. Nach 
den erworbenen Grundkenntnissen der Basis-Thematiken, wird das Wissen über die Voraussetzungen und verwendeten Technologien 
(\ref{chap:Technologien}) geschaffen. Darauffolgend wird im Allgemeinen auf Softwarearchitektur %und OpenGL (\ref{chap:OpenGL})
(\ref{chap:Softwarearchitektur}) und Modulare Software Architektur (\ref{chap:Modulare Software Architektur}) eingegangen. Abschließend zu 
Kapitel (\ref{chap:Grundlagen}) wird zu guter Letzt der Bereich der Datenmodellierung (\ref{chap:Datenmodellierung}) thematisiert.
\\ 
\linebreak
An Kapitel (\ref{chap:Grundlagen}) anschließend wird in Kapitel (\ref{chap:Konzeption}) die Konzeption dargelegt. Anfänglich werden in 
diesem Abschnitt der Arbeit Gedanken, Überlegungen und vorläufige Konzeptionen der Arbeit aufgefasst und erläutert. Unter anderem 
welche Bedingungen die Arbeitsumgebung (\ref{chap:Arbeitsumgebung}), in der die Applikation ihren Nutzen erweist, mit sich bringt. 
Erweiternd dazu wird darauf eingegangen, wie die beiden Phasen Scan-Phase (\ref{chap:Scan-Phase}) und Visualisierungs-Phase 
(\ref{chap:Visualisierungs-Phase}) konzipiert wurden. Ebenso wird das Architekturkonzept (\ref{chap:Architekturkonzept}), welches für das System 
vorgesehen war, genauesten dargelegt. 
Im Anschluss wird auf das ebenso tragende Softwarekonzept (\ref{chap:Softwarekonzept}) eingegangen. Darauffolgend eine kurze Evaluierung wieso sich 
für das angewendete \acs{AR}-Framework (\ref{chap:Auswahl des AR Frameworks}) entschieden wurde und abschließend zu Kapitel 
(\ref{chap:Konzeption}) die Intension des konzipierten Datenmodell's (\ref{chap:Datenmodell}). 
Kapitel (\ref{chap:Umsetzung}) befasst sich mit der Umsetzung des Konzepts, dem Ablauf der Entwicklungsphase, den besonders erwähnenswerten 
Lösungen und den dabei aufgetretenen Problemen.
\\ 
Nach der Umsetzung (\ref{chap:Umsetzung}) gibt es eine Evaluierung (\ref{chap:Evaluation}), um das Projekt sach- und fachgerecht zu beurteilen 
und einen eigenes Fazit zu ziehen.
\\ 
Die letzten zwei Kapitel, Fazit (\ref{chap:Fazit}) und Ausblick (\ref{chap:Ausblick}), %runden die Dokumentation ab und
schließen die Arbeit. Die vorzuweisenden Ergebnisse werden analysiert und Verbesserungsvorschläge angemerkt.
\\ 
Der Ausblick gibt Aufschluss darüber, welche Erweiterungsmöglichkeiten es für diese Arbeit gibt und wie innovativ sich dieser 
Grundbaustein in Zukunft erweisen könnte. 
\section{Stand der Technik}
\label{chap:Stand der Technik}
Die moderne Technologie \acl{AR} findet in aller Munde Zuspruch und offenbart ein weites Spektrum an Einsatzmöglichkeiten. 
Speziell im Hinblick auf \acl{AR} in kombinierter Nutzung mit Smartphones ist dies eine \textit{State of the Art}-Methode, 
um Interaktionen mit virtuellen Objekten zu fördern. Vor allem im Bereich der mobilen \acs{AR} (siehe Abschnitt \ref{sec:mobilesAR}) 
gibt es nützliche Apps, die sich am Markt etabliert haben. Die nach aktuellem Stand am %meisten 
aufsehenerregendsten Marktsektoren sind unter anderem 
Gaming, beispielsweise Pokémon Go und E-Commerce, unter anderem IKEA Place und die DHL Packset App. Dazu kommt der Bereich Bildung und Wissen, mit z. B. GeoGebra 3D 
Grafikrechner und SketchAR und der Sektor Reisen, darunter der \acs{AR} gestützte Google Übersetzer zählt. Auch die Umsetzung von \acl{AR} 
Navigation ist in kleineren Ausführungen durch beispielgebend Google entwickelte Anwendungen vertreten. %Satz der Marktsektoren ggf. umschreiben
\\ 
Im Bereich der Industrie gibt es hinsichtlich mobiler \acs{AR} 
reichlich wenige Anwendungen die allseits bekannt sind, bzw. ein hohes Aufsehen erregt haben. Im Einsatz befinden sich hier verstärkt Smart-Glasses und 
weniger der Einsatz von mobilen Endgeräten. Aufschluss über den allgemeinen Einsatz von \acs{AR} in der Industrie wird in Kapitel 
\ref{chap:Grundlagen} Abschnitt \ref{sec:ARIndustrie} gegeben. 
\\ 
\linebreak
Eine Anwendung die eine Räumlichkeit scannt, ausgewählte Objekte virtuell reproduziert und diese entstehenden Informationen dauerhaft zur 
Verfügung stellt, gibt es derzeit keine vergleichbaren. Die meisten Anwendungen, am Beispiel der IKEA Place Anwendung, beschränken sich 
auf bestimmte Positionen, bzw. darzustellende Objekte an den vorgesehenen Positionen und einer kurzen Gültigkeit und Lebensdauer. 
%Damit ist die Langlebigkeit der Informationen adressiert.
Die Objekte werden bei erneutem Aufruf neu generiert und platziert. Dies bedeutet, 
dass es keinerlei Anhaltspunkte für schon bereits verwendete Objekte gibt. Sie dienen damit lediglich der Ansicht, um die 
Vorstellungskraft des Menschen zu unterstützen. 
\\ 
\linebreak
Die hinter \acs{AR} verborgene Technologie ist in den letzten Jahren durch intensivste Entwicklung und auch Forschung weit gekommen. Durch 
die anfängliche Integration von einfachen Bildern und Texten ist es heutzutage üblich, adaptive Anpassungen der gegebenen Informationen 
vorzunehmen. \cite{standDerAR.2020} Damit können sämtliche Objekte und Bewegungen erstellt werden. Auch ist es möglich, komplette Bewegungsmuster der Realität 
nachzuahmen und diese täuschend echt darzustellen.
\\ 
\linebreak
Im Fokus der \acl{AR} gibt es mittlerweile unzählige SDKs und Frameworks, die alle unterschiedlichste Präferenzen haben. Darunter gibt es 
auch Frameworks der großen Firmen, die vielen ein Begriff sind, hierunter Google ARCore, ARKit von Apple, vuforia, Wikitude und ARToolKit. 
Diese sind ebenso die meist benutzten Frameworks zur Umsetzung von \acs{AR}-Anwendungen. Google ARCore bietet eine tiefgreifende und ausführliche 
\ac{API}, welche in Kombination mit vielen weiteren SDKs verwendet werden kann, um eine beeindruckende User- als auch \acs{AR}-Experience zu verschaffen. 
Demnach befindet sich ARCore, neben Apple ARKit, zur Entwicklung einer Anwendung in einer führenden Position. 
