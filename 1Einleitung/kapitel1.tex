%%%%%%%%%%%%%%%%%%%%%%%%%%%%%%%%%%%%%%%%%%%%%%%%%%%%%%%%%%%%%%%%%%%%%%%%%%%%%%
%% Descr:       Vorlage für Berichte der DHBW-Karlsruhe, Ein Kapitel
%% Author:      Prof. Dr. Jürgen Vollmer, vollmer@dhbw-karlsruhe.de
%% $Id: kapitel1.tex,v 1.17 2018/10/23 08:58:41 vollmer Exp $
%% -*- coding: utf-8 -*-
%%%%%%%%%%%%%%%%%%%%%%%%%%%%%%%%%%%%%%%%%%%%%%%%%%%%%%%%%%%%%%%%%%%%%%%%%%%%%%%

\chapter{Einleitung}
\label{chap:Einleitung}
In diesem Teil der Arbeit wird auf die Motivation des Themas eingegangen. Darüber hinaus wird sowohl die Aufgabenstellung als auch der 
Aufbau der Arbeit genauesten dargelegt. Eine nähere Betrachtung des Standes der Technik untermauert die Beweggründe dieser Ausarbeitung.  

\section{Motivation}
\label{chap:Motivation}
Jede neu entwickelte Technologie durchlebt im Laufe der Entstehung ein enormes Aufsehen. Es wird viel darüber debattiert, fantasiert 
und geplant ohne jedoch genau die Resultate abwägen zu können. Durch fehlende Erfahrung und nicht ausgereifte Konzepte werden Highlights 
erwartet, die zu diesem Zeitpunkt technisch nicht umsetzbar sind. Jede neue technologische Idee durchläuft bestimmte Phasen der Entwicklung.
\cite{studiob12.2020j} 
\\ 
\linebreak
Ein sogenannter Hype Cycle, dt. Hype-Zyklus, ist ein visualisiertes Modell, das die Entwicklung einer neuen Technologie von der 
Innovation über die Umsetzung bis hin zur ausgereiften Marktfähigkeit repräsentiert und so diese Phasen der Entwicklung verdeutlicht.
Der Hype Cycle, entwickelt von der Forschungsgruppe Gartner Inc. und durch deren Mitarbeiterin Jackie Fenn geprägten Definitionen, ist er 
in fünf Entwicklungsphasen untergliedert:
\begin{enumerate}
    \item \textit{Technologische Auslösung:} Bekanntwerden der Technologie und erste Projekte, die auf beachtliches Interesse in der 
    Öffentlichkeit stoßen. 
    \item \textit{Gipfel der überzogenen Erwartungen:} Die Technologie ist auf dem Gipfel der Aufmerksamkeit. Es werden unrealistische 
    und enthusiastische Erwartungen veröffentlicht und diskutiert.
    \item \textit{Tal der Enttäuschung:} Da die Technologie nicht die zuvor aufgebauten Erwartungen erfüllen kann, sinkt die Aufmerksamkeit 
    der enttäuschen Enthusiasten und der Medienvertreter auf den Tiefpunkt.
    \item \textit{Pfad der Erleuchtung:} Die neue Technologie wird in dieser Phase realistisch mit ihren Stärken und Schwächen betrachtet.
    \item \textit{Plateau der Produktivität:} Die neuen Möglichkeiten durch die Technologie werden als Vorteile akzeptiert, und 
    Geschäftsprozesse werden entwickelt. \cite{gartnerhc.2016s}
\end{enumerate} 
%Bsp. welche Erwartungen und welche darauffolgenden Enttäuschungen \listoftodos{bsps. für Erwartungen}
Nachdem eine Innovation den Gipfel der überzogenen Erwartungen passiert hat, folgt das Tal der Enttäuschung, wobei die Technologie 
an Interesse verliert. Nach der erneuten Sammlung, der \textit{"Kurs-Korrektur"} \cite{hypecycle.2019o}, wird die präsente Innovation 
realistischer beurteilt. Durch die objektive Betrachtungsweise entsteht ein neues und realistisches Bild der Möglichkeiten, aber auch 
der Grenzen der Technologie. Zum Ende hin geht die ehemals neue Innovation in eine routinierte Technologie über, wobei diese an 
Anerkennung gewinnt, da die Anwendung sich bewährt, sie bei den Nutzern immer mehr Zuspruch findet und zu Beginn utopisch gedachte Konzepte 
längst einer realen Praxis gewichen sind. Durch die steigende Zuwendung zu der Technologie  erfährt diese eine stetig Weiterentwicklung
und die ursprünglich kleine Nutzergruppe kann zu einer großen Community wachsen. Bezogen auf die Phasen des Hype Cycles befindet sich diese 
Technologie dann an der Stelle des Plateaus der Produktivität und bestätigt die Marktreife. Ab diesem Zeitpunkt handelt es sich nicht mehr 
um eine Zukunftsvision, Hype oder Highlight, sondern um eine am Markt etablierte Technologie.
\\ 
\linebreak
Momentan befindet sich Augmented Reality auf dem Pfad der Erleuchtung und ist auf sehr gutem Wege zu einer ausgereiften Technologie, 
da das Wachstum der Verwendung von Augmented Reality stetig steigt und mittlerweile ein weites Portfolio an möglichen Einsatzgebieten 
vorweist. Die jetzige Erfahrung und der technologische Fortschritt bringt Augmented Reality den  ursprünglich angedachten Visionen und Ideen 
einen Schritt näher, sodass in naher Zukunft die letzte Phase, das Erreichen des Plateaus der Produktivität erzielt werden kann. Den finalen 
Schritt der endgültigen Marktreife zu erlangen ist ein faszinierender und wichtiger Grund für meine Motivation mich dieser Technologie und 
der dahinter stehenden Theorie zu widmen. 
%Durch die gewonnenen Erfahrungen rückt Augmented Reality wieder in den Vordergrund und weckt ein enormes Interesse die 
%gewonnene Technologie vollends zu nutzen. 
Die Technologie der Augmented Reality weist, wie bereits erwähnt, ein immer größer werdendes Portfolio an Einsatzgebieten vor, unter anderem: 
\begin{itemize}
    \item Industrie 
    \item Produktion und Lagerlogistik
    \item Wartung und Reparatur
    \item Spiele, bzw. Gaming 
    \item Medizin
    \item Marketing und Werbung
    \item Navigation
    \item Unterhaltung und Fernsehen 
    \item Schulung und Training
    \item Militär
\end{itemize} 
Neben der Affinität der Technologie bringt es auch viele Vorteile mit sich. Arbeitsprozesse und Herangehensweisen an Projekte werden 
modernisiert. Die Kombination aus vielen Informationen und der visuellen Darstellung der Informationen in Echtzeit reduzieren die 
Fehleranfälligkeit von menschlicher Unachtsamkeit, steigert die Produktivität und verbessert den Wissenstransfer durch die gegebene 
Visualisierung. 
\\ 
\linebreak
Schon im Jahr 2016 sprach Apple CEO Tim Cook die Innovation \textit{Augmented Reality} an zahlreichen Events, Keynotes und Interviews 
an und war zu diesem Zeitpunkt schon enorm begeistert. Er beteuerte: \textit{„...using the tech would become as normal as eating three 
meals a day.“} \cite{timcook2016.2016o} Mit jeder Möglichkeit ging der Apple CEO auf seine Überzeugung gegenüber Augmented Reality ein. 
Erst vor kurzer Zeit bestätigte er, das AR zahlreiche und innovative Einsatzgebiete erlangen und immer mehr an Wichtigkeit zunehmen 
würde, als er sagte: \textit{„pervade your life, it will play a big role..."} \cite{timcook.2020j} 
\\ 
\linebreak
Ein sehr großer Anwendungsbereich mit viel Potential, das bei weitem noch nicht ausgeschöpft sei, ist der Industriezweig. Die 
Marktstudie zu Augmented Reality der IDG Research Services und PTC untermauern das immer weiter steigende Wachstum der Anwendungen in 
Unternehmen. „Fast \textit{75\%} der deutschen Unternehmen setzen bereits \textit{Virtual oder Augmented Reality} ein oder planen dies." 
\cite{studieptc.2020j} Eine Prognose berechnet einen Umsatz in zweistelliger Milliarden-Höhe der bis 2021 erreicht werden soll. So wird 
um so deutlicher das es sich bei Augmented Reality um ein sehr zukunftsorientiertes Marktsegment handelt mit viel Potential, immer 
weiter in den Vordergrund rückt und viele Unternehmen sich damit auseinandersetzen.
\\ 
\linebreak
Durch den Trend zur Vollautomatisierung von Produktions- und Industrieprozessen werden Unmengen an Daten erzeugt. Durch das ebenso 
aktuelle Thema der Industrie 4.0 und dem damit in Verbindung gebrachte \ac{IoT} werden weitere Daten erzeugt und über eine Cloud zu 
Verfügung gestellt. Entstehende Daten sind z.B. Maschinendaten aus Maschinen-Protokollen, Prozessinformationen, Produktionsdaten oder 
Informationen der Endverbraucher. \cite{industrie40.2019f} Augmented Reality ermöglicht es einen Großteil sinnvoll zu nutzen, sei es 
die Ablösung von Papierprozessen oder die einfache Einbindung von digitalen Lösungen in die bestehenden IT-Infrastrukturen. \cite{industrie40ar.2019n} 
Dabei sind Rückschlüsse auf die allgemeine Arbeitsoptimierung zu ziehen. Notwendige Informationen können ohne 
großen Aufwand, schnell und zentral an einem Ort durch AR zur Verfügung gestellt und eingeblendet werden. Damit können 
Wartungen schneller durchgeführt, Defekte besser behoben, Leerläufe oder Fehler am Endprodukt vermieden werden. Mit solch einer Hilfe 
kann z.B. auch die Orientierung in riesigen Produktionshallen aufrecht erhalten werden, um einen besseren Überblick über alle Maschinen 
zu erhalten. Dadurch können auch Prozesse effizienter gestaltet werden. Ein Mechaniker kann alle notwendigen Informationen der Augmented 
Reality Applikation entnehmen, um z.B. eine Wartung schneller durchzuführen oder über Vorfälle in Echtzeit informiert zu sein. 

\section{cjt Systemsoftware AG}
\label{chap:cjt}
Die Arbeit wurde bei der Firma cjt Systemsoftware AG durchgeführt. Diese wurde
1999 von Christian J. Tauber und Ulrich Beck gegründet. Damals mit einem Team
von 20 Personen, beschäftigt die cjt Systemsoftware AG heute mehr als 60 Mitarbeiter. 
Mit ihrem Sitz in Karlsruhe ist sie in einer der größten Technologiestädten Deutschlands angesiedelt.
\\
\linebreak
Durch das stetige Wachstum der cjt Systemsoftware AG vergrößert sich auch deren
Portfolio kontinuierlich. Dabei setzt das Consulting-Unternehmen hauptsächlich auf maßgeschneiderte
Software- und Netzwerklösungen. Großkunden wie Siemens AG, Lufthansa Cargo,
\ac{KIT} und Fraunhofer \acs{Fraunhofer IOSB} zeugen von der hohen Qualität der geleisteten Arbeit. 
Dabei agiert das Unternehmen nicht nur in Deutschland sondern auch international, darunter in Ländern wie China und den USA.
\\ 
\linebreak
Einer der größten Auftraggeber der Firma ist ebenso das in Karlsruhe angesiedelte Unternehmen Siemens AG das eine beispielhafte 
Anwendung für das Assistenzsystem bietet und in Zukunft auch der Firma zur Nutzung unterbreitet werden könnte. 
\pagebreak

\section{Aufgabenstellung}
\label{chap:Aufgabenstellung}          %konzeptioniert
Im Rahmen dieser Arbeit soll ein System konzipiert, entwickelt und umgesetzt werden, welches basierend auf Augmented Reality ein 
Informations- und Unterstützungssystem im industriellen Bereich grundlegend realisiert. 
\\Die entstehende prototypische Applikation soll es ermöglichen einen Überblick über eine Produktions- oder Industriehalle zu 
verschaffen, indem die Umgebung und alle in der Halle stehenden Maschinen erkannt und angezeigt werden. Die erkannten Objekte können 
vom Nutzer eingetragen werden, um die Position des Gegenstands festhalten zu können. Nach dem setzen des Objektes kann der Nutzer 
die benötigten Informationen eintragen, um so grundlegende Informationen über diesen Gegenstand griffbereit zu haben. Mit dieser Arbeit 
soll der Grundbaustein für ein Unterstützungssystem gelegt und das beliebig erweitert werden kann, um eine nützliches \textit{Gadget} 
mit vielen \textit{Features} in der Industrie zur Verfügung zu stellen.
\\ 
\linebreak
Grundlegend ist die Arbeit in drei Aufgabenbereiche unterteilt.
\\
\linebreak
Um eine Applikation übersichtlich zu gestalten und für die Zukunft nicht zu vermeidende Optimierungen, Änderungen oder Erweiterungen 
möglichst einfach integrieren zu können, ist es zu Anfang die Aufgabe eine solide und übersichtliche Grundstruktur, bzw. Software-
Architektur zu erstellen. Dadurch werden Funktionen und Klassen einer klaren Struktur zugeordnet, um eine einheitliche Linie vorzugeben. 
\\
Ein weiterer Aspekt der bei der Erstellung der Architektur berücksichtigt werden soll, ist der Ansatz der modularen 
Softwarearchitektur. Damit können einzelne Funktionen unabhängig voneinander getestet und Abhängigkeiten oder Frameworks leichter 
ersetzt, hinzu oder entfernt werden. Darüber hinaus begünstigt eine modulare Konzeption bessere Kontrollierbarkeit und Übersichtlichkeit 
in großen Softwareprojekten.
\\
\linebreak
Die Hauptaufgabe ist die Realisierung der Augmented Reality Funktion, der Kern der Anwendung. Dabei wird die Applikation in zwei 
Phasen unterteilt, welche es gilt separat zu planen und implementieren.
\\ 
\linebreak
Die erste Phase, genannt Scan-Phase, beschäftigt sich mit dem scannen der Umgebung, bzw. des Raumes. Die Aufgabe dabei besteht darin, 
mittels dem \ac{SLAM} - Verfahren eine Karte der Umgebung zu erstellen und die räumliche Lage innerhalb dieser Karte zu schätzen, um 
auf Basis dieser erstellten Karte virtuell Objekte auf der Karte platzieren zu können. Mit den gewonnen Informationen der räumlichen 
Darstellung durch das \acs{SLAM} - Verfahren kann der Nutzer virtuelle Objekte im virtuellen Raum an Ort und Stelle platzieren, als 
Referenz zu dem existierenden Objekt in der Realität, welches erkannt wurde. Bei der Erstellung eines Objekts soll der Nutzer die 
Möglichkeit haben Informationen über das Objekt in das System einzupflegen, um diese immer abrufen zu können. 
\\ 
\linebreak
In der zweiten Phase, genannt Visualisierungs-Phase, sollte der Nutzer die Möglichkeit haben sich im Raum frei bewegen zu können. 
Mit der Lokalisierung des Nutzer-Geräts und den bekannten Informationen der in Phase eins gesetzten Objekte, sollte dem Nutzer 
die virtuellen Objekte in seiner unmittelbaren Umgebung angezeigt werden. Mit dem Wissen, dass sich im Blickfeld der AR-Applikation 
ein Objekt befindet, können für dieses in der Datenbank weitere Informationen abgefragt und dem Nutzer zur Verfügung gestellt werden.
\\ 
\linebreak
Ein weiterer wichtiger Punkt dieser Arbeit wird die Modellierung eines geeigneten, grundlegenden und prototypischen Datenmodells sein. 
Dieses Datenmodell gibt vor welche Informationen in Phase eins, beim Erstellen eines Objekts, von der Eingabe des Nutzers 
erfasst werden sollten. 


\section{Aufbau der Arbeit}
\label{chap:Aufbau der Arbeit}
Nach den soeben genannten einleitenden Informationen widmet sich das Kapitel (\ref{chap:Grundlagen}) den essentiellen und wichtigsten 
Grundlagen dieser Arbeit. Zu Anfang wird dem Leser der Terminus der \acl{AR} (\ref{chap:Augmented Reality}) offenbart, um allgemein 
Kontexte im Bezug zu dieser Arbeit zu begreifen, gefolgt von einer Einführung in die Thematik des Verfahrens \ac{SLAM}-\acl{SLAM} 
(\ref{chap:SLAM}) der überbegrifflichen Materie der Robotik. Eine weitere notwendige Grundlage ist das Verständnis von Quaternionen 
(\ref{chap:Quaternionen}) und die damit zusammenhängende Rotation und Translation von Objekten in einem dreidimensionalen Raum. Nach 
den erworbenen Grundkenntnisse der Basis-Thematiken, wird das Wissen über die Voraussetzungen und verwendeten Technologien 
(\ref{chap:Technologien}) und OpenGL (\ref{chap:OpenGL}) geschaffen. Darauf folgend wird im Allgemeinen auf Softwarearchitektur 
\ref{chap:Softwarearchitektur} und Modulare Software Architektur (\ref{chap:Modulare Software Architektur}) eingegangen. Abschließend zu 
Kapitel (\ref{chap:Grundlagen}) wird zu guter Letzt der Bereich der Datenmodellierung (\ref{chap:Datenmodellierung}) thematisiert.
\\ 
\linebreak
Anschließend auf Kapitel (\ref{chap:Grundlagen}) wird in Kapitel (\ref{chap:Konzeption}) die Konzeption dargelegt. Anfänglich werden in 
diesem Abschnitt der Arbeit Gedanken, Überlegungen und vorläufige Konzeptionen der Arbeit aufgefasst und erläutert. Unter anderem 
welche Bedingungen die Arbeitsumgebung (\ref{chap:Arbeitsumgebung}), in der die Applikation ihren Nutzen erweist, mit sich bringt. 
Erweiternd dazu, wird darauf eingegangen wie die beiden Phasen Scan-Phase (\ref{chap:Scan-Phase}) und Visualisierungs-Phase 
(\ref{chap:Visualisierungs-Phase}) konzipiert wurden. 
\\ 
Ebenso wird das Architekturkonzept (\ref{chap:Architekturkonzept}), welches für das System vorgesehen war, genauesten dargelegt. 
Im Anschluss wird auf das ebenso tragende Softwarekonzept (\ref{chap:Softwarekonzept}) eingegangen. Eine kurze Evaluierung, wieso sich 
für das angewendete AR-Framework (\ref{chap:Auswahl des AR Frameworks}) entschieden wurde und abschließend zu Kapitel 
(\ref{chap:Konzeption}) die Intension des konzipierten Datenmodell's (\ref{chap:Datenmodell}). 
\\ 
\linebreak
Kapitel (\ref{chap:Umsetzung}) befasst sich mit der Umsetzung des Konzepts, dem Ablauf, den besonders erwähnenswerten Lösungen und den 
dabei aufgetretenen Problemen.
\\ 
\linebreak 
Die letzten zwei Kapitel, Fazit (\ref{chap:Fazit}) und Ausblick (\ref{chap:Ausblick}), runden die Dokumentation ab und schließen die 
Arbeit. Die vorzuweisenden Ergebnisse werden analysiert und Verbesserungsvorschläge angemerkt.
\\ 
Der Ausblick gibt Aufschluss darüber welche Erweiterungsmöglichkeiten es für diese Arbeit gibt und wie innovativ sich dieser 
Grundbaustein in Zukunft erweist. 
\pagebreak

\section{Stand der Technik}
\label{chap:Stand der Technik}