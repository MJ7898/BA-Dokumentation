%%%%%%%%%%%%%%%%%%%%%%%%%%%%%%%%%%%%%%%%%%%%%%%%%%%%%%%%%%%%%%%%%%%%%%%%%%%%%%
%% Descr:       Vorlage für Berichte der DHBW-Karlsruhe, Ein Kapitel
%% Author:      Prof. Dr. Jürgen Vollmer, vollmer@dhbw-karlsruhe.de
%% $Id: kapitel1.tex,v 1.17 2018/10/23 08:58:41 vollmer Exp $
%% -*- coding: utf-8 -*-
%%%%%%%%%%%%%%%%%%%%%%%%%%%%%%%%%%%%%%%%%%%%%%%%%%%%%%%%%%%%%%%%%%%%%%%%%%%%%%%

\chapter{Einleitung}
In diesem Teil der Arbeit wird auf die Motivation des Themas eingegangen. 
Die Aufgabenstellung genau erläutert und sowohl die Ziele als auch der Aufbau der Arbeit dargelegt.
\section{Motivation}
Die herkömmlichen Schaltpläne von Gebäuden, Schaltschränken oder Maschinen werden aus der Historie heraus per Hand auf normales Papier
gezeichnet. Der Elektrotechniker oder gar ein Architekt nimmt Zeit in Anspruch einen solchen Plan detailgetreu und 
nach Maßstab zu zeichnen. Diese Arbeit ist sehr zeitintensiv und preislich sehr teuer. Zudem können handschriftliche Änderungen
das Dokument unübersichtlich machen, bzw. müsste bei jeder Änderung ein neuer Plan ausgefertigt werden. 
\\Aus diesen Gründen wird eine Zeichnung meistens vernachlässigt, wobei es für Zukünftige Arbeiten,
z.B. Sanierungen, Ausbauten o.ä. essentiell ist. Um diesem fatalen Fehler, den Plan zu vermeiden, entgegenzuwirken wurden Ideen und
Vorschläge gesammelt, wie dieser Prozess deutlich einfacher und ressourcenschonender vonstattengehen könnte.
\\
\linebreak
Diese Problemstellung der realen Welt führte dazu diese Aufgabe in Angriff zu nehmen und diesen Prozess zu modernisieren.
Mit dem Gedanken und der Intension der Digitalisierung in der Elektrotechnik wurden Überlegungen getätigt diese Modernisierung
umzusetzen.
\linebreak
\linebreak
Ein motivierender Aspekt dieses Vorgehens ist, dass die auf Papier gezeichneten Dokumente verloren gehen oder physische Schäden erleiden können
und so unbrauchbar werden. Durch die Digitalisierung des Verfahrens sind diese Aspekte ausgeschlossen und sind deutlich vielseitiger.
\linebreak
\linebreak
Im Zeitalter der Industrie 4.0, bei der der Schwerpunkt auf Digitalisierung liegt, war es leicht eine Idee zu generieren die 
standardmäßige Schaltpläne in dem Zeichnungs- und Ausarbeitungsprozess verbessert und ein bekanntest Problem löst.
Sowohl in zeitlicher und kostspieliger Hinsicht als auch die Übersichtlichkeit
solcher Dokumente kann enorm gesteigert werden. Veränderungen keine neue Zeichnung anzufertigen, sondern können in der bestehenden
Datei schnell und einfach ausgebessert werden. 
\newpage
\section{Aufgabenstellung}
Es soll ein Konzept für eine Software erstellt werden, welches erlaubt Schaltpläne digital zu zeichnen, anzuzeigen, 
flexibel zu ändern und alle wichtigen Informationen zu Verfügung zu stellen. 
\\ Dieses ausgearbeitete Konzept soll einen modularen Ansatz verfolgen, um in Zukunft beliebig erweiterbar zu sein
und die Integration von neuen Features zu gewährleisten. Nach Erstellung des Konzepts und der definierten Software-Architektur
soll der Grundbaustein der Applikation gelegt und gefestigt werden, indem die Grundfunktion implementiert und getestet werden.
\\Die Grundfunktionen werden in folgendem kurz aufgelistet, um einen groben Überblick zu verschaffen.
\begin{itemize}
    \item Startmenü - Übersicht
    \begin{itemize}
        \item Erstellen eines Schaltplans
        \item Öffnen einer vorhandenen Schaltplan-Datei
        \item Weiterarbeiten an einer Schaltplan-Datei
    \end{itemize}
    \item Editor
    \begin{itemize}
        \item Grundgerüst Zeichen (Grundriss, Türen, etc.)
        \item Leitungen Zeichen (Größe, Stärke der Leitung)
        \item Mögliche Komponenten einfügen (Steckdose, Lichtschalter, etc.) 
        \item Anzeige von Informationen zu Komponenten und Leitungen
        \item Legende zum Verständnis und zur Übersicht der einzelnen Zeichenkomponenten
    \end{itemize}
    
\end{itemize}
Die obenstehenden Punkte sehen die Grundfunktionen vor und werden in drei große Teilbereiche, wie zu erkennen, abgegrenzt. 
Das Design und der allgemeine Aufbau wird in folgenden Kapiteln genauer erläutert und anhand von Bildern,
Mockups und Diagrammen dargestellt.
\linebreak
\linebreak
Der Hauptaugenmerk dieser Aufgabe, bzw. der Problemstellung liegt darin die Möglichkeit zu schaffen
einen Schaltplan rentabel und einfach zu digitalisieren, um den Einsatz von Schaltplanzeichnungen erneut zu verbreiten beziehungsweise
zu modernisieren. 
\newpage
\section{Aufbau der Arbeit}
Nach den oben genannten einleitenden Informationen widmet sich das Kapitel 2 den essentiellen und wichtigsten Grundlagen dieser Arbeit.
Zu Anfang werden allgemein gültige Grundlagen zur Digitalisierung in der Gebäudetechnik (2.1) offenbart, um Kontexte zur Arbeit im Allgemeinen
zu verstehen, gefolgt von einer Einführung in die modulare Software Architektur (2.2) zum Ende des Kapitels.
\linebreak
\linebreak
Anschließend auf Kapitel 2 wird in Kapitel 3 auf die verwendeten Technologien und Tools eingegangen. Mit Windows Presentation Foundation
(3.1) wird erläutert was es damit auf sich hat und welche Programmiersprache diese Technologie sich zu eigen macht. In (3.2) wird das MVVM-Pattern
genauestens erklärt, was es bedeutet, wie und warum es angewendet wird. Das Windows eigene System.Drawing, welches zum 
Zeichnen der Schaltpläne verwendet wird in (3.3) aufgeführt.
\linebreak
\linebreak
Nach den verwendeten Technologien geht es in Kapitel 4 um den eigentlichen Systementwurf, zum einen um das Architekturkonzept (4.1)
und zum anderen um das Softwarekonzept (4.2). Dabei wird auch detailliert auf das Aussehen eingegangen und die graphischen Benutzeroberflächen, GUIs,
anhand mehreren Mockups dargelegt.
\\
\linebreak
In Kapitel 5 wird genauestens auf die praktische Umsetzung und Implementierung der vier definierten Use Cases eingegangen, 
die im einzelnen kurz chronologisch aufgeführt werden.
\begin{itemize}
    \item Startmenü (5.1)
    \item Toolbox (5.2)
    \item Zeichenfläche (5.3)
\end{itemize}
Die letzten zwei Kapitel, Ereignis (6) und Ausblick (7), runden die Dokumentation ab und schließen die Arbeit. Das Ergebnis wird hierbei
analysiert und Verbesserungsvorschläge angemerkt. 
\\ Der Ausblick gibt Aufschluss darüber welche möglichen Erweiterungsmöglichkeiten es gibt und wie die Zukunft dieser Arbeit möglicherweise aussehen könnte.
