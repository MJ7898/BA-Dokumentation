%%%%%%%%%%%%%%%%%%%%%%%%%%%%%%%%%%%%%%%%%%%%%%%%%%%%%%%%%%%%%%%%%%%%%%%%%%%%%%
%% Descr:       Vorlage für Berichte der DHBW-Karlsruhe, Datei mit Abkürzungen
%% Author:      Prof. Dr. Jürgen Vollmer, vollmer@dhbw-karlsruhe.de
%% $Id: abk.tex,v 1.4 2017/10/06 14:02:03 vollmer Exp $
%% -*- coding: utf-8 -*-
%%%%%%%%%%%%%%%%%%%%%%%%%%%%%%%%%%%%%%%%%%%%%%%%%%%%%%%%%%%%%%%%%%%%%%%%%%%%%%%

\chapter*{Abkürzungsverzeichnis}                   % chapter*{..} -->   keine Nummer, kein "Kapitel"
						         % Nicht ins Inhaltsverzeichnis
% \addcontentsline{toc}{chapter}{Akürzungsverzeichnis}   % Damit das doch ins Inhaltsverzeichnis kommt

% Hier werden die Abkürzungen definiert
\begin{acronym}[DHBW]
  % \acro{Name}{Darstellung der Abkürzung}{Langform der Abkürzung}
 \acro{Abk}[Abk.]{Abkürzung}
 \acro{AR}[AR]{Augmented Reality}
 \acro{UX}[UX]{User-Excperience, dt. Nutzererfahrung und Nutzererlebnis}
 \acro{UC}[UC]{Use Case}
 \acro{UCs}[UCs]{Use Cases}
 \acro{UI}[GUI]{Graphical User Interface, dt. Benutzeroberfläche}
 % Folgendes benutzen, wenn der Plural einer Abk. benöigt wird
 % \newacroplural{Name}{Darstellung der Abkürzung}{Langform der Abkürzung}
 \newacroplural{Abk}[Abk-en]{Abkürzungen}

 \acro{H2O}[\ensuremath{H_2O}]{Di-Hydrogen-Monoxid}

 % Wenn neicht benutzt, erscheint diese Abk. nicht in der Liste
 \acro{NUA}{Not Used Acronym}
\end{acronym}
