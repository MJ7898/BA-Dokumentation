%%%%%%%%%%%%%%%%%%%%%%%%%%%%%%%%%%%%%%%%%%%%%%%%%%%%%%%%%%%%%%%%%%%%%%%%%%%%%%
%% Descr:       Vorlage für Berichte der DHBW-Karlsruhe, Datei mit Abkürzungen
%% Author:      Prof. Dr. Jürgen Vollmer, vollmer@dhbw-karlsruhe.de
%% $Id: abk.tex,v 1.4 2017/10/06 14:02:03 vollmer Exp $
%% -*- coding: utf-8 -*-
%%%%%%%%%%%%%%%%%%%%%%%%%%%%%%%%%%%%%%%%%%%%%%%%%%%%%%%%%%%%%%%%%%%%%%%%%%%%%%%

\chapter*{Abkürzungsverzeichnis}                   % chapter*{..} -->   keine Nummer, kein "Kapitel"
						         % Nicht ins Inhaltsverzeichnis
% \addcontentsline{toc}{chapter}{Akürzungsverzeichnis}   % Damit das doch ins Inhaltsverzeichnis kommt

% Hier werden die Abkürzungen definiert
\begin{acronym}[DHBW]
  % \acro{Name}{Darstellung der Abkürzung}{Langform der Abkürzung}
 \acro{Abk}[Abk.]{Abkürzung}
 \acro{API}[API]{Application Programming Interface}
 \acro{AR}[AR]{Augmented Reality}
 \acro{DBMS}[DBMS]{Datenbank-Management-System}
 %\acro{ER}[ER]{deutsch erweiterte Realität}
 \acro{EKF}[EKF]{Erweiterter Kalman Filter}
 \acro{ERM}[ERM]{Entity-Relationship-Modell}
 \acro{Fraunhofer IOSB}[IOSB]{Fraunhofer Institut für Optronik, Systemtechnik und Bildauswertung IOSB}
 \acro{GPS}[GPS]{Global Positioning System}
 \acro{GUI}[GUI]{Graphical User Interface, dt. Benutzeroberfläche}
 \acro{HMD}[HMD]{Head-Mounted Display}
 \acro{ICP}[ICP]{Iterativ Closest Point}
 \acro{IDE}[IDE]{Integrated Development Environment}
 \acro{ID}[ID]{Identifikation}
 \acro{IMU}[IMU]{Inertial Measurement Unit}
 \acro{INS}[INS]{Inertial Navigation System}
 \acro{IoT}[IoT]{Internet of Things, dt. Internet der Dinge}
 \acro{KIT}[KIT]{Forschungszentrum Karlsruhe}
 \acro{MEMS}[MEMS]{Micro-Electro-Mechanical Systems}
 \acro{MR}[MR]{Mixed Reality}
 \acro{MVC}[MVC]{Model View Controller}
 \acro{MVVM}[MVVM]{Model View ViewModel}
 \acro{SDK}[SDK]{Software Development Kit}
 \acro{SLAM}[SLAM]{Simultanious Localization And Mapping}
 \acro{SQL}[SQL]{Structured Query Language}
 \acro{TOF}[TOF]{Time of flight}
 \acro{UX}[UX]{User-Excperience, dt. Nutzererfahrung und Nutzererlebnis}
 \acro{UC}[UC]{Use Case}
 \acro{UCs}[UCs]{Use Cases}
 \acro{UI}[UI]{User Interface}
 \acro{VR}[VR]{Virtual Reality}
 \acro{WMR}[WMR]{Windows Mixed Reality}

 % Folgendes benutzen, wenn der Plural einer Abk. benöigt wird
 % \newacroplural{Name}{Darstellung der Abkürzung}{Langform der Abkürzung}
 \newacroplural{Abk}[Abk-en]{Abkürzungen}


 \acro{H2O}[\ensuremath{H_2O}]{Di-Hydrogen-Monoxid}

 % Wenn nicht benutzt, erscheint diese Abk. nicht in der Liste
 \acro{NUA}{Not Used Acronym}
\end{acronym}
