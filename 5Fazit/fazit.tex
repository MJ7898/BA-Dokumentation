\chapter{Evaluation}
\label{chap:Evaluation}
Nach Beendigung der Entwicklungsphase wurde eine Evaluation des Softwareprodukts durchgeführt. Diese dient zur Kategorisierung und Einstufung der Applikation in 
die Gebrauchstauglichkeit. Dabei wurde auch ein Vergleich zu schon bestehenden, bzw. ähnlichen Produkten aufgestellt. Darunter werden 
ebenso Aspekte der Benutzerfreundlichkeit, der Weiterentwicklungsmöglichkeiten beachtet und eine Überprüfung auf gewünschte, bzw. vorab definierte Anforderungen 
und Ziele durchgeführt. 
\\ 
\linebreak
Im Bereich der Industrie gibt es derzeit einige Anwendungen, die auf \acl{AR} aufbauen, allerdings wenige Apps, die sich mit vorliegenden Prozessen 
auseinandersetzen. Es gibt vereinzelt Anwendungen, die auf Interaktionen in Prozessabläufen eingreifen, diese sind allerdings nicht zwangsläufig basierend auf 
einer Smartphone-Applikation. Auch gibt es viele Anwendungsfälle, bei denen auf \acs{VR}- oder \acs{AR}-Brillen vertraut wird. Diese sind meist dafür zuständig, 
Prozesse zu begleiten und zusätzliche Informationen oder Visualisierungen zu projizieren, um dem Nutzer erweiterte Einblicke zu verschaffen. Es gibt auch 
Anwendungen, die visualisierte Arbeitsprozesse vorgeben, um das Potenzial, den Effekt und die Effizienz des „learning by doing“-Ansatzes auszuschöpfen oder auch 
um Vorstellungen und Planungen zu visualisieren, um diese sich besser vorstellen zu können. Beispiel dazu sind zum einen der „augmented presenter“ \cite{tepcon.2020} 
und der „augmented instructor“ \cite{tepcon.2020} der tepcon GmbH. Der „augmented presenter“ bietet die Möglichkeit eine Inneneinrichtung einer Produktionshalle 
zu visualisieren, damit Maschinen an verschiedenen Stellen platziert werden können und anhand dieser Projektion wird überprüft, ob sich die vorgesehene Position 
für die Platzierung der zu planende Maschine eignet. Diese Anwendung dient zur optimierten Fabrikplanung, bevor die Produkte bestellt oder in Betrieb genommen 
werden. Auch kann diese Anwendung für Vertrieb und Marketing, Schulungen und Messen verwendet werden, um die Möglichkeiten der \acs{AR} zu demonstrieren. Der 
„augmented instructor“ basiert auf der Verwendung der \acs{AR}-Brille und visualisiert und projiziert Arbeitsschritte zur Veranschaulichung. 
\\ 
\linebreak
Das entwickelte, prototypische Assistenzsystem dient zur übersichtlichen Veranschaulichung der vorhandenen Maschinen in einer Produktionshalle. Dadurch kann die 
Umgebung eingescannt und Maschinen als Objekte virtuell dargestellt werden. Mit den nötigen Daten ermöglicht dieses System die schnelle und einfache 
Einsicht von Informationen zu einzelnen Maschinen. Auch die Option der Echtzeit-Statusüberprüfung kann in weiterer Entwicklung realisiert werden. 
\\ 
So kann das Assistenzsystem schnell Informationen zur Verfügung stellen, diese dann auch immer verfügbar sind. Durch die einfache Erweiterbarkeit, die 
durch die Architektur gewährleistet ist, können weitere nützliche Funktionen hinzugefügt werden. Demnach wird eine gute Gebrauchstauglichkeit aufgezeigt, die 
stetig erweitert und verbessert werden kann.
\\ 
Auch im Hinblick der Benutzerfreundlichkeit ist eine einfache \acs{UI} gegeben, die überschaubar und schnell zu verstehen ist. Auch nach den Prinzipien der 
Dialoggestaltung der ISO 9241-10 Norm wird eine gute Steuerbarkeit, Selbstbeschreibungsfähigkeit, Individualisierbarkeit und Aufgabenangemessenheit vorgewiesen, 
die durch Usability-Tests in kleinem Rahmen praktiziert wurden. Auf diese in Folgendem nicht näher eingegangen wird. Im Rahmen der Testung wurden 
Verbesserungsvorschläge geäußert, die unter anderem allerdings in diesem Bezug nur an Erweiterungen angeknüpft waren. Ebenso wurde die Genauigkeit bemängelt, da 
diese manchmal fehlerhaft Objekte erstellt und wiedergibt. Diese Ursache ist den Sensoren, je nach Lage und Ausrichtung zuzuschreiben, deshalb ist diese 
Auffälligkeit nur schwer zu umgehend, bzw. zu verhindern. 
\\ 
In der Grundgesamtheit wurde das System als überaus hilfreich eingestuft und mit viel Potenzial bewertet, da in dieser Arbeit lediglich die Konzeption, 
Grundlagenschaffung und prototypische Entwicklung im Fokus stand und so die Grundfunktion implementiert und zur Verfügung gestellt wurde.
\\ 
\linebreak
Nun folgt Abschließend zur Ausarbeitung, um die Dokumentation abzurunden, das allgemeine Fazit des Projekts und der Ausblick, der darlegt wie die Zukunft des 
Projekts aussehen könnte. 

\chapter{Fazit}
\label{chap:Fazit}
Ziel dieser Arbeit war die Konzeption und prototypische Umsetzung eines Assistenzsystems zur Unterstützung industrieller Prozesse. Dabei wurde der Fokus auf die 
übersichtliche und virtuelle Darstellung von Maschinen und Geräten gelegt. Zu der Übersicht kam auch dazu, dass zu einzelnen Objekten schnell Informationen eingesehen 
werden können, um diese bei Bedarf zur Hand zu haben. Dafür wurden in den ersten Schritten Anforderungen zur Umsetzung des Systems festgelegt. Unter Berücksichtigung der 
zu gewährleistenden Modularisierung, damit künftige Weiterentwicklungen an dem Projekt möglich sind und keine großen Herausforderungen darstellt, musste ein geeignetes 
Entwurfsmuster gefunden werden, auf dem die Architektur aufbaut. Durch die Android Architecture Components wurde sich an dem MVVM-Muster angelehnt, welches unter anderem 
die Anforderungen der Modularisierung erfüllt. Bevor die konkrete Umsetzung stattfinden konnte, wurden anhand der Anforderungen und geplanten Funktionen Use Cases für die 
Implementierung definiert. Nachdem diese modelliert waren, konnte das eigentliche System aufgestellt und entwickelt werden.
\\ 
\linebreak
Die Positionsberechnung des Smart-Devices basiert auf Kalkulationen der hardwareinternen Sensoren und unterliegt somit einer sehr sensiblen Basis. Je nach Ausrichtung und 
Lage des Geräts kann sich sowohl dessen Orientierung als auch dessen Blickrichtung unterscheiden und so das eigentliche Ergebnis, obwohl sich das Gerät an der physikalisch 
identischen Position befindet, verfälschen.  

\chapter{Ausblick}
\label{chap:Ausblick}
Der in dieser Arbeit entstandene Prototyp ist ein eigenständiges System zur visualisierten Übersicht von Maschinen und Geräten, beispielsweise in Produktionshallen. 
Bevor die Anwendung allerdings produktiv eingesetzt werden kann, sind noch einige Optimierungen vorzunehmen. Darunter die Verwendung von anderen Objekten 
zur visuellen Darstellung der Maschinen, eventuell auch detailgetreue Abbildungen in einem bestimmten Anwendungsumfeld oder die Verwendung universeller Objekte, die 
ortsunabhängig eingesetzt werden können. Des weiteren die Überarbeitung der Informationsanzeige der Objekte, die aktuell über ein vereinfachtes Informationsfenster 
eingeblendet werden. Eine weitere \acl{AR} basierte Anzeige der Informationen wäre dabei denkbar. In dieser Arbeit wurden lediglich die Grundsteine für ein Projekt 
gelegt, welches in Zukunft stetig weiterentwickelt werden kann und viel Potential für weitere Funktionen, auf Basis der \acs{AR}-Anwendung, bietet. Die Ergebnisse 
des Prototypen sind für erste Tests gut, allerdings für den eigentlichen Einsatz noch nicht geeignet, da noch kein erwähnenswerter Mehrwert daraus resultiert und die 
Maschinenbezogenen Daten im Hinblick auf die statusabhängigen Informationen keine Echtzeit-Auskünfte bietet. Diese können daher aktuell nur provisorisch beschrieben 
werden. Daher muss der Prototyp weiterentwickelt und verbessert werden. 
\\ 
\linebreak
Um Informationen über einzelne Maschinen in Echtzeit aufrufen und visualisieren zu können, wäre ein weiterer Entwicklungsschritt die Benutzung von Echtzeitdaten der 
Maschinen. Dafür wäre eine Datenaufbereitung und eine globale Verfügbarkeit der Informationen notwendig. Diese Idee könnte über eine \acl{IoT}-Lösung umgesetzt werden. 
Dabei könnten die Maschinendaten, die über eine Cloud ausgelagert sind, in Echtzeit abgegriffen und stets den aktuellen Zustand und Status der einzelnen Geräte 
abgerufen werden. Auch könnten die Daten direkt von den jeweiligen Geräten verwendet, aufbereitet und in der Anwendung genutzt werden. Dadurch wäre die genauer 
Informationsgebung zu den Objekten möglich.
\\ 
\linebreak
Eine zusätzlicher Erweiterung der Anwendung wäre das automatische lokalisieren von Anomalien sämtlicher Maschinen. Dabei würden auftretende Fehler registriert werden 
und über eine Benachrichtigung den Nutzer darüber in Kenntnis setzen. Somit könnte bei Ausfällen der Maschinen schnell reagiert und agiert werden, um diese auftretenden 
Fehler schnellstmöglich zu beheben. 
\\ 
Basierend auf dieser Erweiterung wäre es ebenso möglich bei großen Umgebungen, bzw. Produktionshallen eine räumliche Navigation einzubauen, die den Mechaniker auf 
schnellstem Wege zu der wartungsbedürftigen Maschine zu leiten, um eine Orientierung in der Räumlichkeit zu gewährleisten. 
\\ 
\linebreak
Derzeit steht nur eine Laufzeitumgebung für ein natives Android System zur Verfügung. Ein weiterer Schritt zur Systemoptimierung wäre die Einbindung von \acs{AR}-Brillen, 
im Vergleich zum Smartphone wäre der Nutzer in seinen Bewegungen uneingeschränkt und könnte sich mit den Händen frei bewegen. 
\\ 
Aufbauend darauf könnten auch Arbeitsschritte der Reparaturarbeiten optimiert werden, indem über die \acs{AR}-Brille Reparaturschritte und Anleitungen zur Wartung der 
Maschine bereitgestellt werden. 
\\ 
\linebreak
Dies sind einige Ansätze, mit denen die Einsatzmöglichkeiten des Projekts in Zukunft erweitert werden könnten, um ein umfängliches Assistenzsystem zu erschaffen und einen 
echten Mehrwert in unterstützenden, industriellen Maßnahmen zu gewährleisten.  
