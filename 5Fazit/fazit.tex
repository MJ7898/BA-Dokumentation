\chapter{Evaluation}
\label{chap:Evaluation}
Nach Beendigung der Entwicklungsphase wurde eine Evaluation des Softwareprodukts durchgeführt. Diese dient zur Kategorisierung und Einstufung der Applikation in 
die Gebrauchstauglichkeit. Dabei wurde auch ein Vergleich zu schon bestehenden, bzw. ähnlichen Produkten aufgestellt. Darunter befinden sich 
ebenso Aspekte der Benutzerfreundlichkeit, der Weiterentwicklungsmöglichkeiten und eine Überprüfung auf gewünschte, bzw. vorab definierte Anforderungen 
und Ziele. 
\\ 
\linebreak
Im Bereich der Industrie gibt es derzeit einige Anwendungen, die auf \acl{AR} aufbauen, darunter sind allerdings wenige Apps, die sich mit vorliegenden Prozessen 
auseinandersetzen. Es gibt vereinzelt Anwendungen, die mit Interaktionen in Prozessabläufen arbeiten. Diese basieren allerdings weniger auf 
einer Smartphone-Applikation. Es wird vermehrt auf \acs{VR}- und \acs{AR}-Brillen gesetzt. Diese sind meist für die Begleitung der 
Prozesse zuständig und projizieren zusätzliche Informationen oder Visualisierungen, um dem Nutzer erweiterte Einblicke zu verschaffen. Es gibt auch 
Anwendungen die visualisierte Arbeitsprozesse vorgeben, und das Potenzial, den Effekt und die Effizienz des „learning by doing“-Ansatzes ausschöpfen oder auch 
Vorstellungen und Planungen zum besseren Vorstellen visualisieren. Beispiele dazu sind zum einen der „augmented presenter“ \cite{tepcon.2020} 
und der „augmented instructor“ \cite{tepcon.2020} der tepcon GmbH. Der „augmented presenter“ bietet die Möglichkeit, eine Inneneinrichtung einer Produktionshalle 
zu visualisieren und damit Maschinen an verschiedenen Stellen zu platziert. Anhand dieser Projektion wird überprüft, ob sich die vorgesehene Position 
für die Platzierung der zu planende Maschine eignet. Diese Anwendung dient zur optimierten Fabrikplanung, bevor die Produkte bestellt oder in Betrieb genommen 
werden. Auch kann diese Anwendung für Vertrieb und Marketing, Schulungen und Messen zur Demonstration der Möglichkeiten der \acs{AR} verwendet werden. Der 
„augmented instructor“ basiert auf dem Gebrauch der \acs{AR}-Brille und visualisiert und projiziert Arbeitsschritte zur Veranschaulichung. 
\\ 
\linebreak
Das entwickelte, prototypische Assistenzsystem dient zur übersichtlichen Verdeutlichung der vorhandenen Maschinen in einer Produktionshalle. Dadurch kann die 
Umgebung eingescannt und Maschinen als Objekte virtuell dargestellt werden. Mit den nötigen Daten ermöglicht dieses System die schnelle Abrufung von Informationen 
zu einzelnen Maschinen. Auch die Option der Echtzeit-Statusüberprüfung kann in weiterer Entwicklung realisiert werden. 
\\ 
So kann das Assistenzsystem schnell Informationen zur Verfügung stellen, die immer verfügbar sind. Durch die einfache Erweiterbarkeit, die 
durch die Architektur gewährleistet ist, können weitere nützliche Funktionen hinzugefügt werden und bestätigt dessen stetig erweiterbare Alltagstauglichkeit.
\\ 
Auch im Hinblick auf die Benutzerfreundlichkeit ist eine einfache \acs{UI} gegeben, die überschaubar und schnell zu verstehen ist. Die Norm ISO 9241-10 schreibt eine gute 
Steuerbarkeit, Selbstbeschreibungsfähigkeit, Individualisierbarkeit und Aufgabenangemessenheit vor. In der Entwicklung der Anwendung wurde versucht, sich daran zu orientieren. 
In Usability-Tests wurden die Anforderungen der Norm in kleinem Rahmen, aber noch nicht abschließend, überprüft. Auf diese Tests wird im Folgenden aus Relevanzgründen nicht 
näher eingegangen. Im Rahmen der Testungen wurden 
Verbesserungsvorschläge geäußert, die sich in diesem Bezug nur auf Erweiterungen bezogen haben. Ebenso wurde die Genauigkeit bemängelt, da 
diese manchmal fehlerhaft Objekte erstellt und wiedergibt. Die Ursache ist der Handhabung durch den Nutzer zuzuschreiben, der je nach Lage und Ausrichtung des Smart-Device 
die Berechnung beeinflusst. 
\\ 
In der Gesamtheit wurde das System als überaus hilfreich und entwicklungsfähig eingestuft. In dieser Arbeit stehen lediglich die Konzeption, 
Grundlagenschaffung und prototypische Entwicklung im Fokus. Die Grundfunktion wurde implementiert und zur Verfügung gestellt.
\\ 
\linebreak
Nun folgt abschließend, um die Dokumentation abzurunden, das allgemeine Fazit des Projekts und der Ausblick, der die Zukunftsaussichten des Projekts darlegt. 
\chapter{Fazit}
\label{chap:Fazit}
Ziel dieser Arbeit war die Konzeption und prototypische Umsetzung eines Assistenzsystems zur Unterstützung industrieller Prozesse. Dabei wurde der Fokus auf die 
übersichtliche und virtuelle Darstellung von Maschinen gelegt. Zu der Übersicht kam auch dazu, dass zu einzelnen Objekten schnell Informationen eingesehen 
werden können, um diese bei Bedarf zur Hand zu haben. Dafür wurden in den ersten Schritten Anforderungen zur Umsetzung des Systems festgelegt. Unter Berücksichtigung der 
zu gewährleistenden Modularisierung musste ein geeignetes Entwurfsmuster gefunden werden, auf dem die Architektur aufbaut, um künftige Weiterentwicklungen an dem Projekt 
zu ermöglichen und diese keine großen Herausforderungen darstellen. Durch die Android Architecture Components wurde sich an dem MVVM-Muster angelehnt, welches unter anderem 
die Anforderungen der Modularisierung erfüllt. Bevor die konkrete Umsetzung stattfinden konnte, wurden anhand der Anforderungen und geplanten Funktionen Use Cases für die 
Implementierung definiert. Nachdem diese modelliert waren, konnte das eigentliche System aufgestellt und entwickelt werden.
\\ 
\linebreak
Die Positionsberechnung des Smart-Devices basiert auf Kalkulationen der hardwareinternen Sensoren und unterliegt somit einer sehr sensiblen Basis. Je nach Ausrichtung und 
Lage des Geräts kann sich sowohl dessen Orientierung, als auch dessen Blickrichtung unterscheiden und so das eigentliche Ergebnis verfälschen, obwohl sich das Gerät an der physikalisch 
identischen Position befindet. Dadurch ist die Benutzung der Applikation stark von den Aktionen und der garantierten Handhabung des Nutzers abhängig. 
\\ 
\linebreak
Das verwendete Framework Google ARCore erweist einen effektiven Nutzen und ist eine gute Grundlage für die Implementierung einer \acl{AR} Applikation. Das 
Framework stellt eine fundamentierte \acs{API} zur Verfügung, die eine gute Basis für die darauf aufbauenden Entwicklungen schafft. Bei der praktischen 
Anwendung des Frameworks gibt es jedoch vereinzelt Probleme mit der Anzeige der virtuellen Objekte, da aus den Positionsberechnungen phasenweise Übereinstimmungen mit diesen 
nicht erkannt und dadurch die Objekte nicht angezeigt werden. Daher ist es in diesem Fall notwendig, an den Ausgangspunkt zurückzukehren, damit die 
Anzeige der Objekte aktualisiert werden kann und so die nicht angezeigten Visualisierungen wiederauftauchen.
\\ 
\linebreak
Werden Objekte über die Scan-Phase an eine bestimmte Position gesetzt, wird die daraus resultierende Differenz zum Ursprungsmarker gespeichert. Bei der 
Ausführung der Visualisierungs-Phase werden die gespeicherten Objekte und deren Positionsdifferenz abhängig von der Ursprungsmarkierung wieder virtuell im Raum platziert. 
Dabei ist die Ausgangssituation der Markierungsverfolgung entscheidend. Dies bedeutet, dass das Smart-Device vom Nutzer richtig angewendet werden muss und es beispielsweise 
nicht über Kopf gehalten werden darf. Dadurch würde die Berechnung und Darstellung der Objekte nicht exakt mit der vorgesehenen Position übereinstimmen. Dies hat eine 
ungenaue Repräsentation der Objekte im Raum zufolge. 
\\ 
Es ist durch diese Sensibilität der Anwendung vorauszusetzen, dass der Nutzer das System verantwortungsvoll und gewissenhaft verwendet. Da dies von dem Assistenzsystem nicht überprüft 
und auch nicht korrigiert werden kann, muss die fachgerechte Handhabung vorausgesetzt werden. 
\\ 
\linebreak
Die Benutzeroberflächen sind überschaubar und bieten eine gute Grundlage für die Anwendung der \acl{AR} Applikation. 
\\ 
\linebreak
In der Gesamtheit bietet das Ergebnis dieser Arbeit einen klaren und leicht verständlichen Aufbau des Systems, sowie eine Grundlage für Weiterentwicklungen. 
Das Konzept wurde eigenständig erarbeitet und wie geplant umgesetzt. Somit ist der Grundbaustein für weitere, darauf aufbauende Funktionen gelegt.

\chapter{Ausblick}
\label{chap:Ausblick}
Der in dieser Arbeit entstandene Prototyp ist ein eigenständiges System zur visualisierten Übersicht von Maschinen und Geräten, beispielsweise in Produktionshallen. 
Bevor die Anwendung allerdings produktiv eingesetzt werden kann, sind noch einige Optimierungen vorzunehmen. Darunter die Verwendung von weiteren Objekten 
zur visuellen Darstellung der Maschinen. Eventuell können auch detailgetreue Abbildungen in einem bestimmten Anwendungsumfeld oder die Verwendung universeller 
ortsunabhängiger Objekte eingesetzt werden. Des Weiteren muss die Überarbeitung der Informationsanzeige der Objekte, die aktuell über ein vereinfachtes Informationsfenster 
eingeblendet werden, erfolgen. Eine \acl{AR} basierte Anzeige der Informationen wäre dabei denkbar. In dieser Arbeit wurden lediglich die Grundsteine für ein Projekt 
gelegt, welches in Zukunft stetig weiterentwickelt werden kann und viel Potential für weitere Funktionen, auf Basis der \acs{AR}-Anwendung, bietet. Die Ergebnisse 
des Prototyps sind für erste Tests geeignet, allerdings für den %derzeitigen
Einsatz in derartiger Form noch nicht vorgesehen. Da zum jetzigen Zeitpunk noch kein Zusatznutzen für Anwender 
daraus resultiert und die maschinenbezogenen Daten im Hinblick auf die statusabhängigen Informationen keine Echtzeit-Auskünfte bietet, ist das Assistenzsystem für Firmen 
noch uninteressant und unrentabel. Der Prototyp muss der Prototyp weiterentwickelt und verbessert werden. 
\\ 
\linebreak
Um Informationen über einzelne Maschinen in Echtzeit aufrufen und visualisieren zu können, wäre ein weiterer Entwicklungsschritt die Benutzung von Echtzeitdaten der 
Maschinen. Dafür wäre eine Datenaufbereitung und eine globale Verfügbarkeit der Informationen notwendig. Diese Idee könnte über eine \acl{IoT}-Lösung, umgesetzt werden. 
Dabei könnten die Maschinendaten, die über eine Cloud ausgelagert sind, in Echtzeit abgegriffen und stets den aktuellen Zustand und Status der einzelnen Geräte 
abgerufen werden. Auch könnten die Daten direkt von den jeweiligen Geräten verwendet, aufbereitet und in der Anwendung genutzt werden. Dadurch wäre die Möglichkeit 
gegeben, genauere Informationen zu den Objekten zu erhalten.
\\ 
\linebreak
In der Visualisierungs-Phase sind derzeit Änderungen an den Objekten nach deren Erstellung durch den Nutzer nicht möglich. Dafür müssten alle Informationen aus 
der Datenbank gelöscht und die Scan-Phase erneut begonnen werden. Eine mögliche Funktion zur Erweiterung wäre die Editierung einzelner Objekte. Dadurch würde das 
Löschen aller Informationen entfallen. 
\\ 
\linebreak
Eine zusätzliche Erweiterung der Anwendung wäre das automatische Lokalisieren von Anomalien sämtlicher Maschinen. Dabei würden auftretende Fehler registriert werden 
und über eine Benachrichtigung den Nutzer darüber in Kenntnis setzen. Somit könnte bei Ausfällen der Maschinen schnell reagiert und agiert werden, um diese auftretenden 
Fehler schnellstmöglich zu beheben. 
\\ 
Basierend auf dieser Erweiterung wäre es ebenso möglich, bei großen Umgebungen, bzw. Produktionshallen eine räumliche Navigation einzubauen. Der Mechaniker wird auf 
direktem Wege zu der wartungsbedürftigen Maschine geleitet und muss sich nicht zusätzlich in der Räumlichkeit orientieren. 
\\ 
\linebreak
Im Moment steht nur eine Laufzeitumgebung für ein natives Android System zur Verfügung. Ein weiterer Schritt zur Systemoptimierung wäre die Einbindung von \acs{AR}-Brillen, 
da der Nutzer im Vergleich zum Smartphone in seinen Bewegungen nicht eingeschränkt wäre und sich mit den Händen frei bewegen könnte. Dabei müssten unter anderem 
die Berechnungen der Navigation und Positionsbestimmung angepasst werden.
\\ 
\linebreak
Aufbauend darauf könnten auch Schritte der Reparaturarbeiten optimiert werden, indem über die \acs{AR}-Brille Anweisungen und Anleitungen zur Wartung der 
Maschine über die \acs{AR}-Brille zusätzlich bereitgestellt werden. 
\\ 
\linebreak
Dies sind einige Ansätze für Einsatzmöglichkeiten des Projekts und wie sie in Zukunft erweitert werden könnten, um ein umfängliches Assistenzsystem zu erschaffen und zu einer 
echten Unterstützung industrieller Prozesse zu werden.  
