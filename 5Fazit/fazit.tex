\chapter{Evaluation}
\label{chap:Evaluation}
Nach Beendigung der Entwicklungsphase wurde eine Evaluation des Softwareprodukts durchgeführt. Diese dient zur Kategorisierung und Einstufung der Applikation in 
die Gebrauchstauglichkeit. Dabei wurde auch ein Vergleich zu schon bestehenden, bzw. ähnlichen Produkten aufgestellt. Darunter werden 
ebenso Aspekte der Benutzerfreundlichkeit, der Weiterentwicklungsmöglichkeiten beachtet und eine Überprüfung auf gewünschte, bzw. vorab definierte Anforderungen 
und Ziele durchgeführt. 
\\ 
\linebreak
Im Bereich der Industrie gibt es derzeit einige Anwendungen, die auf \acl{AR} aufbauen, allerdings wenige Apps, die sich mit vorliegenden Prozessen 
auseinandersetzen. Es gibt vereinzelt Anwendungen die auf Interaktionen in Prozessabläufen eingreifen, diese sind allerdings nicht zwangsläufig basierend auf 
einer Smartphone-Applikation. Auch gibt es viele Anwendungsfälle, bei denen auf \acs{VR}- oder \acs{AR}-Brillen vertraut wird. Diese sind meist dafür zuständig, 
Prozesse zu begleiten und zusätzliche Informationen oder Visualisierungen zu projizieren, um dem Nutzer erweiterte Einblicke zu verschaffen. Es gibt auch 
Anwendungen die visualisierte Arbeitsprozesse vorgeben, um das Potenzial, den Effekt und die Effizienz des „learning by doing“-Ansatzes auszuschöpfen oder auch 
um Vorstellungen und Planungen zu visualisieren, um diese sich besser vorstellen zu können. Beispiel dazu sind zum einen der „augmented presenter“ \cite{tepcon.2020} 
und der „augmented instructor“ \cite{tepcon.2020} der tepcon GmbH. Der „augmented presenter“ bietet die Möglichkeit eine Inneneinrichtung einer Produktionshalle 
zu visualisieren, damit Maschinen an verschiedenen Stellen platziert werden können und anhand dieser Projektion wird überprüft, ob sich die vorgesehene Position 
für die Platzierung der zu planende Maschine eignet. Diese Anwendung dient zur optimierten Fabrikplanung, bevor die Produkte bestellt oder in Betrieb genommen 
werden. Auch kann diese Anwendung für Vertrieb und Marketing, Schulungen und Messen verwendet werden, um die Möglichkeiten der \acs{AR} zu demonstrieren. Der 
„augmented instructor“ basiert auf der Verwendung der \acs{AR}-Brille und visualisiert und projiziert Arbeitsschritte zur Veranschaulichung. 
\\ 
\linebreak
Das entwickelte, prototypische Assistenzsystem dient zur übersichtlichen Veranschaulichung der vorhandenen Maschinen in einer Produktionshalle. Dadurch kann die 
Umgebung eingescannt und Maschinen als Objekte virtuell dargestellt werden. Mit den nötigen Daten ermöglicht dieses System die schnelle und einfache 
Einsicht von Informationen zu einzelnen Maschinen. Auch die Option der Echtzeit-Statusüberprüfung kann in weiterer Entwicklung realisiert werden. 
\\ 
So kann das Assistenzsystem schnell Informationen zur Verfügung stellen, diese dann auch immer verfügbar sind. Durch die einfache Erweiterbarkeit, die 
durch die Architektur gewährleistet ist, können weitere nützliche Funktionen hinzugefügt werden. Demnach wird eine gute Gebrauchstauglichkeit aufgezeigt, die 
stetig erweitert und verbessert werden kann.
\\ 
Auch im Hinblick der Benutzerfreundlichkeit ist eine einfache \acs{UI} gegeben, die überschaubar und schnell zu verstehen ist. Auch nach den Prinzipien der 
Dialoggestaltung der ISO 9241-10 Norm wird eine gute Steuerbarkeit, Selbstbeschreibungsfähigkeit, Individualisierbarkeit und Aufgabenangemessenheit vorgewiesen, 
die durch Usability-Tests in kleinem Rahmen praktiziert wurden. Auf diese in Folgendem nicht näher eingegangen wird. Im Rahmen der Testung wurden 
Verbesserungsvorschläge geäußert, die unter anderem allerdings in diesem Bezug nur an Erweiterungen angeknüpft waren. Ebenso wurde die Genauigkeit bemängelt, da 
diese manchmal fehlerhaft Objekte erstellt und wiedergibt. Diese Ursache ist den Sensoren, je nach Lage und Ausrichtung zuzuschreiben, deshalb ist diese 
Auffälligkeit nur schwer zu umgehend, bzw. zu verhindern. 
\\ 
In der Grundgesamtheit wurde das System als überaus hilfreich eingestuft und mit viel Potenzial bewertet, da in dieser Arbeit lediglich die Konzeption, 
Grundlagenschaffung und prototypische Entwicklung im Fokus stand und so die Grundfunktion implementiert und zu Verfügung gestellt wurde.
\\ 
\linebreak
Nun folgt Abschließend zur Ausarbeitung, um die Dokumentation abzurunden, das allgemeine Fazit des Projekts und der Ausblick, der darlegt wie die Zukunft des 
Projekts aussehen könnte. 

\chapter{Fazit}
\label{chap:Fazit}

\chapter{Ausblick}
\label{chap:Ausblick}