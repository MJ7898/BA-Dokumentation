%%%%%%%%%%%%%%%%%%%%%%%%%%%%%%%%%%%%%%%%%%%%%%%%%%%%%%%%%%%%%%%%%%%%%%%%%%%%%
%% Descr:       Vorlage für Berichte der DHBW-Karlsruhe, Ein Kapitel
%% Author:      Prof. Dr. Jürgen Vollmer, vollmer@dhbw-karlsruhe.de
%% $Id: kapitel2.tex,v 1.5 2017/10/06 14:02:51 vollmer Exp $
%%  -*- coding: utf-8 -*-
%%%%%%%%%%%%%%%%%%%%%%%%%%%%%%%%%%%%%%%%%%%%%%%%%%%%%%%%%%%%%%%%%%%%%%%%%%%%%%%

\chapter{Konzeption}
\label{chap:Konzeption}
In diesem Kapitel wird das erarbeitete Konzept dieser Ausarbeitung dargelegt. Unter dessen die Überlegungen zu einzelnen 
Arbeitsschritten und dem grundsätzlich angedachten Aufbau des Projekts, sowie Entscheidungs- und Beweggründe wieso bestimmte 
Technologien gewählt wurden. Zu Beginn wird auf die Einsatzmöglichkeiten (\ref{chap:Arbeitsumgebung}), in der die Applikation Anwendung 
finden könnte, eingegangen. Anschließend werden die Grundgedanken zu den einzelnen Phasen, Scan-Phase (\ref{chap:Scan-Phase}) und 
Visualisierungs-Phase (\ref{chap:Visualisierungs-Phase}) des Systems erläutert, was sie beinhalten und wie sie funktionstechnisch 
angedacht sind. Mit den zugrundeliegenden Informationen wird auf das Architekturkonzept (\ref{chap:Architekturkonzept}), sowie auf das 
Softwarekonzept (\ref{chap:Architekturkonzept}) eingegangen. Des Weiteren werden die Hintergründe der Wahl des AR-Frameworks 
(\ref{chap:Auswahl des AR Frameworks}) aufgezeigt und abschließend wird noch das konzipierte und prototypische Datenmodell 
(\ref{chap:Datenmodell}) dargelegt.

\section{Arbeitsumgebung / Umfeld}
\label{chap:Arbeitsumgebung}

\section{Objekterkennung / Scan-Phase}
\label{chap:Scan-Phase}

\section{Visualisierungs-Phase}
\label{chap:Visualisierungs-Phase}

\section{Architekturkonzept}
\label{chap:Architekturkonzept}

\section{Softwarekonzept}
\label{chap:Softwarekonzept}

\section{Auswahl des AR Frameworks}
\label{chap:Auswahl des AR Frameworks}
Mittlerweile gibt es eine enorme Auswahl an \acs{AR}-Frameworks, die alle unterschiedlich unterschiedliche Präferenzen und 
Einsatzmöglichkeiten haben. Somit sind, auf den Bereich bezogen, Vor- und Nachteile im Vergleich von mehrere Frameworks nicht ausgeschlossen. 
Einige Alternativen wurden getestet und auf deren Brauchbarkeit evaluiert und analysiert. Dazu wurden Kriterien ausgearbeitet, die die 
Auswahl an Frameworks einschränken und nach Möglichkeit das passendste ergeben sollte: 
\begin{enumerate}
    \item Eine performante Darstellung von Objekten.
    \item Möglichkeiten zur Positionsbestimmung.
    \item Eine aktive Community und stetige Weiterentwicklung des Systems.
    \item Möglichkeit zur Integration weiterer Technologien.
    \item Open Source-Projekt, um Flexibilität und weitestgehende Unabhängigkeit zu gewährleisten.
\end{enumerate}
Aufgrund der großen Anzahl an \acs{AR}-Frameworks wurde sich dazu entschlossen diese nicht detailliert in dieser Ausarbeitung aufzuführen, 
lediglich die engere Auswahl der Tools wird aufgegriffen. 
\subsection{Google ARCore}
\subsection{ARToolKit}

\section{Datenmodell}
\label{chap:Datenmodell}