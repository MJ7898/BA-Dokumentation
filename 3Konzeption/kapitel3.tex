%%%%%%%%%%%%%%%%%%%%%%%%%%%%%%%%%%%%%%%%%%%%%%%%%%%%%%%%%%%%%%%%%%%%%%%%%%%%%
%% Descr:       Vorlage für Berichte der DHBW-Karlsruhe, Ein Kapitel
%% Author:      Prof. Dr. Jürgen Vollmer, vollmer@dhbw-karlsruhe.de
%% $Id: kapitel2.tex,v 1.5 2017/10/06 14:02:51 vollmer Exp $
%%  -*- coding: utf-8 -*-
%%%%%%%%%%%%%%%%%%%%%%%%%%%%%%%%%%%%%%%%%%%%%%%%%%%%%%%%%%%%%%%%%%%%%%%%%%%%%%%

\chapter{Konzeption}
\label{chap:Konzeption}
In diesem Kapitel wird das erarbeitete Konzept dieser Ausarbeitung dargelegt. Unter dessen die Überlegungen zu einzelnen 
Arbeitsschritten und dem grundsätzlich angedachten Aufbau des Projekts, sowie Entscheidungs- und Beweggründe wieso bestimmte 
Technologien gewählt wurden. Zu Beginn wird auf die Einsatzmöglichkeiten (\ref{chap:Arbeitsumgebung}), in der die Applikation Anwendung 
finden könnte, eingegangen. Anschließend werden die Grundgedanken zu den einzelnen Phasen, Scan-Phase (\ref{chap:Scan-Phase}) und 
Visualisierungs-Phase (\ref{chap:Visualisierungs-Phase}) des Systems erläutert, was sie beinhalten und wie sie funktionstechnisch 
angedacht sind. Mit den zugrundeliegenden Informationen wird auf das Architekturkonzept (\ref{chap:Architekturkonzept}), sowie auf das 
Softwarekonzept (\ref{chap:Architekturkonzept}) eingegangen. Des Weiteren werden die Hintergründe der Wahl des AR-Frameworks 
(\ref{chap:Auswahl des AR Frameworks}) aufgezeigt und abschließend wird noch das konzipierte und prototypische Datenmodell 
(\ref{chap:Datenmodell}) dargelegt.

\section{Arbeitsumgebung / Umfeld}
\label{chap:Arbeitsumgebung}

\section{Objekterkennung / Scan-Phase}
\label{chap:Scan-Phase}
\section{Visualisierungs-Phase}
\label{chap:Visualisierungs-Phase}

\section{Architekturkonzept}
\label{chap:Architekturkonzept}

\section{Softwarekonzept}
\label{chap:Softwarekonzept}

\section{Auswahl des AR Frameworks}
\label{chap:Auswahl des AR Frameworks}
\subsection{Google ARCore}
\subsection{ARToolKit}

\section{Datenmodell}
\label{chap:Datenmodell}