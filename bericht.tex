%%%%%%%%%%%%%%%%%%%%%%%%%%%%%%%%%%%%%%%%%%%%%%%%%%%%%%%%%%%%%%%%%%%%%%%%%%%%%%%
%% Descr:       Vorlage für Berichte der DHBW-Karlsruhe
%% Author:      Prof. Dr. Jürgen Vollmer, juergen.vollmer@dhbw-karlsruhe.de
%% $Id: bericht.tex,v 1.24 2018/10/23 09:10:06 vollmer Exp $
%%  -*- coding: utf-8 -*-
%%%%%%%%%%%%%%%%%%%%%%%%%%%%%%%%%%%%%%%%%%%%%%%%%%%%%%%%%%%%%%%%%%%%%%%%%%%%%%%

\documentclass[
   ngerman          % neue deutsche Rechtschreibung
  ,a4paper          % Papiergrösse
% ,twoside          % Zweiseitiger Druck (rechts/links)
% ,10pt             % Schriftgrösse
  ,11pt
% ,12pt
  ,pdftex
%  ,disable         % Todo-Markierungen auschalten
]{report}

% Bitte die Codierung Ihrer Dateien auswählen:
% \usepackage[latin1]{inputenc}    % Für UNIX mit ISO-LATIN-codierten Dateien
% \usepackage[applemac]{inputenc}  % Für Apple Mac
% \usepackage[ansinew]{inputenc}   % Für Microsoft Windows
\usepackage[utf8]{inputenc}        % UTF-8 codierte Dateien
                                   % Dieses Dokument ist unter Unix erstellt, daher
                                   % wird diese Input-Codierung benutzt.
\usepackage[english,ngerman]{babel}
\usepackage{bericht}
\usepackage{subfigure}
\usepackage{amsmath}
\usepackage{blindtext}

%%%%%%%%%%%%%%%%%%%%%%%%%%%%%%%%%%%%%%%%%%%%%%%%%%%%%%%%%%%%%%%%%%%%%%%%%%%%%%%
%% Angaben zur Arbeit
%%%%%%%%%%%%%%%%%%%%%%%%%%%%%%%%%%%%%%%%%%%%%%%%%%%%%%%%%%%%%%%%%%%%%%%%%%%%%%%

\newcommand{\Autor}{Mikka Jenne}
\newcommand{\MatrikelNummer}{2062885}
\newcommand{\Kursbezeichnung}{TINF17B4}

\newcommand{\FirmenName}{cjt Systemsoftware AG}
\newcommand{\FirmenStadt}{Karlsruhe}
\newcommand{\cjtFirmenLogo}{\includegraphics[width=2.5cm]{cjtFirmenLogo}}

% Falls es kein Firmenlogo gibt:
\newcommand{\FirmenLogoDeckblatt}{}

\newcommand{\BetreuerFirma}{M. Sc. Florian Dunz}
\newcommand{\BetreuerDHBW}{Prof. Dr. Marcus Strand}

%%%%%%%%%%%%%%%%%%%%%%%%%%%%%%%%%%%%%%%%%%%%%%%%%%%%%%%%%%%%%%%%%%%%%%%%%%%%%%%%%%%%%

% Wird auf dem Deckblatt und in der Erklärung benutzt:
%\newcommand{\Was}{Projekt-/Studien-/Bachleorarbeit}
%\newcommand{\Was}{Projektrarbeit}
%\newcommand{\Was}{Studienarbeit}
\newcommand{\Was}{Bachelorarbeit}

%%%%%%%%%%%%%%%%%%%%%%%%%%%%%%%%%%%%%%%%%%%%%%%%%%%%%%%%%%%%%%%%%%%%%%%%%%%%%%%%%%%%%

\newcommand{\Titel}{Konzeption und Umsetzung eines Augmented Reality basierten Assistenzsystems zur Unterstützung industrieller Prozesse}
\newcommand{\AbgabeDatum}{31. August 2020}

\newcommand{\Dauer}{12 Wochen}

% \newcommand{\Abschluss}{Bachelor of Engineering}
\newcommand{\Abschluss}{Bachelor of Science}

\newcommand{\Studiengang}{Informatik}
% \newcommand{\Studiengang}{Informatik / Angewandte Informatik}

\hypersetup{%%
  pdfauthor={\Autor},
  pdftitle={\Titel},
  pdfsubject={\Was}
}

%%%%%%%%%%%%%%%%%%%%%%%%%%%%%%%%%%%%%%%%%%%%%%%%%%%%%%%%%%%%%%%%%%%%%%%%%%%%%%%

% Wenn \includeonly{..} benutzt wird, werden nur diese Kaptitel ausgegeben.
%\includeonly{
 % abk
 %,kapitel1
 %,kapitel2
 %,kapitel3
 %,systementwurf
 %,changelog
%}

%%%%%%%%%%%%%%%%%%%%%%%%%%%%%%%%%%%%%%%%%%%%%%%%%%%%%%%%%%%%%%%%%%%%%%%%%%%%%%%

% Benutzt man das "biblatex"-Paket, dann muß das hier stehen:
% siehe auch die mit BIBLATEX markierten Zeilen in bericht.sty
\bibliography{bericht}

\begin{document}

%%%%%%%%%%%%%%%%%%%%%%%%%%%%%%%%%%%%%%%%%%%%%%%%%%%%%%%%%%%%%%%%%%%%%%%%%%%%%%%%

\begin{titlepage}
\begin{center}
\vspace*{-2cm}
\cjtFirmenLogo
\FirmenLogoDeckblatt\hfill\includegraphics[width=4cm]{dhbw-logo}\\[2cm]
{\Huge \Titel}\\[1cm]
{\Huge\scshape \Was}\\[1cm]
{\large für die Prüfung zum}\\[0.5cm]
{\Large \Abschluss}\\[0.5cm]
{\large des Studienganges \Studiengang}\\[0.5cm]
{\large an der}\\[0.5cm]
{\large Dualen Hochschule Baden-Württemberg Karlsruhe}\\[0.5cm]
{\large von}\\[0.5cm]
{\large\bfseries \Autor}\\[1cm]
{\large Abgabedatum \AbgabeDatum}
\vfill
\end{center}
\begin{tabular}{l@{\hspace{4cm}}l}
Bearbeitungszeitraum	         & \Dauer 			\\
Matrikelnummer	                 & \MatrikelNummer		\\
Kurs			         & \Kursbezeichnung		\\
Ausbildungsfirma	         & \FirmenName			\\
			         & \FirmenStadt			\\
Betreuer der Ausbildungsfirma	 & \BetreuerFirma		\\
Gutachter der Studienakademie	 & \BetreuerDHBW		\\
\end{tabular}
\end{titlepage}
 
%%%%%%%%%%%%%%%%%%%%%%%%%%%%%%%%%%%%%%%%%%%%%%%%%%%%%%%%%%%%%%%%%%%%%%%%%%%%%%%

\input{erklaerung.tex}

%%%%%%%%%%%%%%%%%%%%%%%%%%%%%%%%%%%%%%%%%%%%%%%%%%%%%%%%%%%%%%%%%%%%%%%%%%%%%%%
\selectlanguage{ngerman}
\begin{abstract}
  Augmented Reality ist eine Technologie, die dem Nutzer ein visuelles 
  Erlebnis mit einer angereicherten Welt voller virtueller Objekte ermöglicht. Sozusagen wird eine Kombination aus Realität und Virtualität geschaffen. Das Resultat 
  bietet dem Benutzer eine neue Art der Wahrnehmung der Gegenwart. %Realitätswahrnehmung.  
  \\ 
  \linebreak
  Diese Bachelorarbeit befasst sich mit der Konzeption und Umsetzung eines industriellen Assistenzsystems unter Verwendung der Augmented Reality Technologie. Dabei soll die Umgebung 
  mit Hilfe des SLAM Verfahrens analysiert werden. Auf dieser Basis können dreidimensionale Objekte als Referenz zu realen Objekten im Raum virtuell platziert werden. 
  Durch die entstehende Visualisierung können Informationen zu den jeweiligen Objekten in eine Datenbank eingetragen und angezeigt werden. Diese Arbeit schafft die Grundlage 
  für die Vereinfachung der Überwachen von Industriemaschinen. 
  \\ 
  \linebreak
  Zu dem Konzept gehört sowohl die Ausarbeitung der grundlegenden Softwarearchitektur, als auch ein allgemein-gültiges Datenmodell zur 
  Persistierung der generierten Daten. Für die bestmögliche Umsetzung der Augmented Reality Experience werden hierzu bereits schon bestehende Frameworks und Software 
  Development Kits verwendet, darunter Google ARCore. 
  \\ 
  Der entstandene Prototyp ist ein eigenständiges System. Die Architektur ist modular aufgebaut. Damit ist eine stetige Weiterentwicklung gewährleistet.  
\end{abstract}
\newpage
\selectlanguage{english}
\begin{abstract}
  Augmented Reality is a technology that enables the user to have a visual 
  experience with an enriched world full of virtual objects. The result offers the user a new 
  way of perceiving surrounding as an combination of reality and virtuality.
  \\
  \linebreak
  This bachelor thesis deals with the conception and implementation of an industrial assistance system using augmented reality technology. The environment %should
  can be analyzed with the help of the SLAM method in order to be able to place three-dimensional objects virtually as a reference to real objects in space.
  The resulting visualization enables information on the respective objects to be entered in a database and displayed. This enables the simplified 
  monitoring of Industrial machines.
  \\
  \linebreak
  The concept includes the development of the basic software architecture as well as a generally applicable data model for
  saving the generated data. In order to archieve the best augmented reality experience, existing frameworks such as Google ARCore were used.
  \\
  The created prototype is an standalone system. The architecture is build modular in order to ensure continuous further development.
\end{abstract}
\newpage
\selectlanguage{ngerman}
%\pagestyle{plain}
\chead{\headmark}
\cfoot{\pagemark}
%\pagestyle{headings}
\pagenumbering{Roman}
\tableofcontents           % Inhaltsverzeichnis hier ausgeben
\listoffigures             % Liste der Abbildungen
\listoftables              % Liste der Tabellen
\lstlistoflistings         % Liste der Listings
%\listofequations           % Liste der Formeln
% Jetzt kommt der "eigentliche" Text
%%%%%%%%%%%%%%%%%%%%%%%%%%%%%%%%%%%%%%%%%%%%%%%%%%%%%%%%%%%%%%%%%%%%%%%%%%%%%%
%% Descr:       Vorlage für Berichte der DHBW-Karlsruhe, Datei mit Abkürzungen
%% Author:      Prof. Dr. Jürgen Vollmer, vollmer@dhbw-karlsruhe.de
%% $Id: abk.tex,v 1.4 2017/10/06 14:02:03 vollmer Exp $
%% -*- coding: utf-8 -*-
%%%%%%%%%%%%%%%%%%%%%%%%%%%%%%%%%%%%%%%%%%%%%%%%%%%%%%%%%%%%%%%%%%%%%%%%%%%%%%%

\chapter*{Abkürzungsverzeichnis}                   % chapter*{..} -->   keine Nummer, kein "Kapitel"
						         % Nicht ins Inhaltsverzeichnis
% \addcontentsline{toc}{chapter}{Akürzungsverzeichnis}   % Damit das doch ins Inhaltsverzeichnis kommt

% Hier werden die Abkürzungen definiert
\begin{acronym}[DHBW]
  % \acro{Name}{Darstellung der Abkürzung}{Langform der Abkürzung}
 \acro{Abk}[Abk.]{Abkürzung}
 \acro{AR}[AR]{Augmented Reality}
 %\acro{ER}[ER]{deutsch erweiterte Realität}
 \acro{Fraunhofer IOSB}[IOSB]{Fraunhofer Institut für Optronik, Systemtechnik und Bildauswertung IOSB}
 \acro{GPS}[GPS]{Global Positioning System}
 \acro{HMD}[HMD]{Head-Mounted Display}
 \acro{IoT}[IoT]{Internet of Things, dt. Internet der Dinge}
 \acro{INS}[INS]{Inertial Navigation System}
 \acro{KIT}[KIT]{Forschungszentrum Karlsruhe}
 \acro{MR}[MR]{Mixed Reality}
 \acro{SLAM}[SLAM]{Simultanious Localization And Mapping}
 \acro{UX}[UX]{User-Excperience, dt. Nutzererfahrung und Nutzererlebnis}
 \acro{UC}[UC]{Use Case}
 \acro{UCs}[UCs]{Use Cases}
 \acro{UI}[GUI]{Graphical User Interface, dt. Benutzeroberfläche}
 \acro{VR}[VR]{Virtual Reality}
 \acro{WMR}[WMR]{Windows Mixed Reality}
 % Folgendes benutzen, wenn der Plural einer Abk. benöigt wird
 % \newacroplural{Name}{Darstellung der Abkürzung}{Langform der Abkürzung}
 \newacroplural{Abk}[Abk-en]{Abkürzungen}


 \acro{H2O}[\ensuremath{H_2O}]{Di-Hydrogen-Monoxid}

 % Wenn neicht benutzt, erscheint diese Abk. nicht in der Liste
 \acro{NUA}{Not Used Acronym}
\end{acronym}
              % Abkürzungsverzeichnis
\pagenumbering{arabic}
%%%%%%%%%%%%%%%%%%%%%%%%%%%%%%%%%%%%%%%%%%%%%%%%%%%%%%%%%%%%%%%%%%%%%%%%%%%%%%
%% Descr:       Vorlage für Berichte der DHBW-Karlsruhe, Ein Kapitel
%% Author:      Prof. Dr. Jürgen Vollmer, vollmer@dhbw-karlsruhe.de
%% $Id: kapitel1.tex,v 1.17 2018/10/23 08:58:41 vollmer Exp $
%% -*- coding: utf-8 -*-
%%%%%%%%%%%%%%%%%%%%%%%%%%%%%%%%%%%%%%%%%%%%%%%%%%%%%%%%%%%%%%%%%%%%%%%%%%%%%%%

\chapter{Einleitung}
In diesem Teil der Arbeit wird auf die Motivation des Themas eingegangen. Darüber hinaus wird sowohl die Aufgabenstellung als auch der 
Aufbau der Arbeit genauesten dargelegt. Eine nähere Betrachtung des Standes der Technik untermauert die Beweggründe der Ausarbeitung 
dieser Arbeit. 

\section{Motivation}
\label{chap:Motivation}
Jede neu entwickelte Technologie durchlebt im Laufe der Entstehung ein enormes Aufsehen. Es wird viel darüber debattiert, fantasiert 
und geplant ohne jedoch genau die Resultate abwägen zu können. Durch fehlende Erfahrung und nicht ausgereifte Konzepte werden Highlights 
erwartet die zu diesem Zeitpunkt technisch nicht umsetzbar sind. Jede neue technologische Idee macht diese Phasen der Entwicklung durch. 
\\ 
\linebreak
Ein sogenannter Hype Cycle, dt. Hype-Zyklus, ist ein visualisiertes Modell, das die Entwicklung einer neuen Technologie von der 
Innovation über die Umsetzung bis hin zur ausgereiften Marktfähigkeit repräsentiert und so die Phasen der Entwicklung verdeutlicht. 
Nachdem eine Innovation den Gipfel der überzogenen Erwartungen passiert hat, folgt das Tal der Enttäuschung, wobei die Technologie 
an Interesse verliert. Nach der erneuten Sammlung, der \textit{"Kurs-Korrektur"} \cite{hypecycle.2019o}, wird die präsente Innovation 
realistischer beurteilt. Durch die objektive Betrachtungsweise entsteht ein neues und realistisches Bild der Möglichkeiten, aber auch 
der Grenzen der Technologie. Zum Ende hin geht die ehemals neue Innovation in eine routinierte Technologie über, wobei diese an 
Anerkennung gewinnt und sich weiterentwickelt. Diese Position des Modells signalisiert und bestätigt die Marktreife einer Technologie 
und wir ab diesem Zeitpunkt nichtmehr als Zukunftsvision, Hype oder Highlight angesehen.
\\ 
\linebreak
Momentan befindet sich Augmented Reality auf dem Pfad der Erleuchtung und ist auf sehr gutem Wege zu einer ausgereiften Technologie, 
da das Wachstum der Verwendung von Augmented Reality stetig steigt und mittlerweile ein weites Portfolio an möglichen Einsatzgebieten 
vorweist. Mit der jetzigen Erfahrung und dem technologischen Fortschritt können Visionen und Ideen in Bezug auf Augmented Reality, 
die zu Beginn der Innovation geäußert wurden, umgesetzt werden. Durch die gewonnenen Erfahrungen rückte Augmented Reality wieder in 
den Vordergrund und weckt ein enormes Interesse die gewonnene Technologie vollends zu nutzen. Die Technologie der Augmented Reality 
weist, wie bereits erwähnt, ein riesiges Portfolio an Einsatzgebieten vor, unter anderem: 

\begin{itemize}
    \item Industrie 
    \item Produktion und Lagerlogistik
    \item Wartung und Reparatur
    \item Spiele, bzw. Gaming 
    \item Medizin
    \item Marketing und Werbung
    \item Navigation
    \item Unterhaltung und Fernsehen 
    \item Schulung und Training
    \item Militär
\end{itemize} 
Neben der Affinität der Technologie bringt es auch viele Vorteile mit sich. Arbeitsprozesse und Herangehensweisen an Projekte werden 
modernisiert. Die Kombination aus vielen Informationen und der visuellen Darstellung der Informationen in Echtzeit reduzieren die 
Fehleranfälligkeit von menschlicher Unachtsamkeit, steigert die Produktivität und verbessert den Wissenstransfer durch die gegebene 
Visualisierung. 
\\ 
\linebreak
Schon im Jahr 2016 sprach Apple CEO Tim Cook die Innovation \textit{Augmented Reality} an zahlreichen Events, Keynotes und Interviews 
an und war zu diesem Zeitpunkt schon enorm begeistert. Er beteuerte: \textit{„...using the tech would become as normal as eating three 
meals a day.“} \cite{timcook2016.2016o} Mit jeder Möglichkeit ging der Apple CEO auf seine Überzeugung gegenüber Augmented Reality ein. 
Erst vor kurzer Zeit bestätigte er, das AR zahlreiche und innovative Einsatzgebiete erlangen und immer mehr an Wichtigkeit zunehmen 
würde, als er sagte: \textit{„pervade your life is because it will play a big role..."} \cite{timcook.2020j}


\section{cjt Systemsoftware AG}
\label{chap:cjt}
Die Arbeit wurde bei der Firma cjt Systemsoftware AG durchgeführt. Diese wurde
1999 von Christian J. Tauber und Ulrich Beck gegründet. Damals mit einem Team
von 20 Personen, beschäftigt die cjt Systemsoftware AG heute mehr als 60 Mitarbeiter. 
Mit ihrem Sitz in Karlsruhe ist sie in einer der größten Technologiestädten Deutschlands angesiedelt.
\\
\linebreak
Durch das stetige Wachstum der cjt Systemsoftware AG vergrößert sich auch deren
Portfolio kontinuierlich. Dabei setzt das Consulting-Unternehmen hauptsächlich auf maßgeschneiderte
Software- und Netzwerklösungen. Großkunden wie Siemens AG, Lufthansa Cargo,
Forschungszentrum Karlsruhe (KIT) und Fraunhofer IOSB zeugen von der hohen Qualität der geleisteten Arbeit. 
Dabei agiert das Unternehmen nicht nur in Deutschland sondern auch international, darunter in Ländern wie China und den USA.

\section{Aufgabenstellung}
\label{chap:Aufgabenstellung}
\section{Aufbau der Arbeit}
\label{chap:Aufbau der Arbeit}
\section{Stand der Technik}
\label{chap:Stand der Technik}
%%%%%%%%%%%%%%%%%%%%%%%%%%%%%%%%%%%%%%%%%%%%%%%%%%%%%%%%%%%%%%%%%%%%%%%%%%%%%
%% Descr:       Vorlage für Berichte der DHBW-Karlsruhe, Ein Kapitel
%% Author:      Prof. Dr. Jürgen Vollmer, vollmer@dhbw-karlsruhe.de
%% $Id: kapitel2.tex,v 1.5 2017/10/06 14:02:51 vollmer Exp $
%%  -*- coding: utf-8 -*-
%%%%%%%%%%%%%%%%%%%%%%%%%%%%%%%%%%%%%%%%%%%%%%%%%%%%%%%%%%%%%%%%%%%%%%%%%%%%%%%

\chapter{Grundlagen}
\label{chap:Grundlagen}
In diesem Kapitel werden die für diese Bachelorarbeit notwendigen Grundlagen geschaffen, um ein fundiertes Wissen und Verständnis 
über verwendete Technologien zu schaffen. Auf alle diese Informationen und Voraussetzungen wird im Folgenden eingegangen, um nachfolgende 
Konzeption und Umsetzung besser zu verstehen.

\section{Augmented Reality}
\label{chap:Augmented Reality}
Eine der wichtigsten Grundlagen dieser Arbeit ist das Verständnis des Begriffs der Augmented Reality.
\\ 
\acl{AR}, deutsch erweiterte Realität, ist eine durch den Computer gestützte Erweiterung der Realität, bzw. der menschlichen 
Wahrnehmung. Es ermöglicht dem Nutzer die reale Welt mit Überlagerung oder Zusammensetzung virtueller Objekte und visueller Informationen
zu sehen. Mittels einer Art Overlay werden diese Objekte und Informationen über die reale Welt gelegt und dem Nutzer zur Verfügung gestellt. 
Allgemein soll damit dem Nutzer ein weit gefächerter Überblick verschafft werden und Hilfestellung leisten, aber den Nutzer in keinerlei 
Interaktion mit der Umgebung einschränken. Die Definition, welche sich in der Wissenschaft weitestgehend durchgesetzt und etabliert hat ist 
die Definition nach Azuma aus dem Jahre 1997.
\begin{quote}
    „Augmented Reality (AR) is a variation of Virtual Environments (VE), or Virtual Reality as it is more commonly called. VE 
    technologies completely immerse a user inside a synthetic environment. While immersed, the user cannot see the real world around him. 
    In contrast, AR allows the user to see the real world, with virtual objects superimposed upon or composited with the real world. 
    Therefore, AR supplements reality, rather than completely replacing it.“ \cite{azuma.1997a}
\end{quote}
Ein Augmented Reality System verfügt nach \cite{azuma.1997a} über folgende drei charakteristische Merkmale: 
\begin{enumerate}
    \item Es kombiniert Realität und Virtualität.
    \item Es ist interaktiv in Echtzeit.
    \item Die virtuellen Inhalte sind im 3D registriert.
\end{enumerate}
Das erste genannte Merkmal kombiniert die reale Welt mit dem oben genannten Overlay, der Überlagerung der Realität um künstliche virtuelle 
Objekte und visuelle Informationen. Dies bedeutet, der Nutzer nimmt die reale Umgebung gleichzeitig mit den darin liegenden virtuellen 
Objekten als ein Ganzes wahr. Daraus resultiert die Interaktion von virtuellen Objekten und Informationen mit der realen Welt in Echtzeit, 
damit sie als Teil der Realität registriert werden können. Das dritte Merkmal umfasst die Darstellung von Objekten als scheinbar reales 
Objekt. Mit dem letzt genannten Merkmal wird das Ziel verfolgt die projizierten, bzw. nicht realen Teile täuschend echt in die Umgebung zu 
integrieren.
\\ 
\linebreak  
%Darüber hinaus sind die virtuellen Inhalte in 3D (d. h. geometrisch) registriert. Dies bedeutet nichts anderes, als dass in einer 
%AR-Umgebung ein virtuelles Objekt scheinbar einen festen Platz in Realität hat und diesen, sofern es nicht durch eine Benutzerinteraktion 
%verändert wird oder sich z. B. in Form einer Animation selbst verändert, auch beibehält. Mit anderen Worten: Es verhält sich aus Nutzersicht 
%genauso, wie ein reales Objekt, was sich an diesem Ort befinden würde
Eine etwas allgemein formuliertere Definition ist die nach \cite{springer.2019s}, welche die drei charakteristischen Merkmale besonders 
aufgreift:
\begin{quote}
    „Augmentierte Realität (AR) ist eine (unmittelbare und interaktive) um virtuelle Inhalte (für beliebige Sinne) angereicherte Wahrnehmung der 
    realen Umgebung in Echtzeit, welche sich in ihrer Ausprägung und Anmutung soweit wie möglich an der Realität orientiert, sodass im 
    Extremfall (so dies gewünscht ist) eine Unterscheidung zwischen realen und virtuellen (Sinnes-) Eindrücken nicht mehr möglich ist.„ \cite{springer.2019s}
\end{quote}
Diese Definition nimmt sich als Grundlage die oben aufgeführte Definition nach \cite{azuma.1997a}.
\\ 
\linebreak
Der Author L. Frank Baum \cite{frankbaum.1856m} verkündete die ersten Ideen und Gedanken einer Augmented Reality Anwendung in 
\textit{„The Master Key“} \cite{masterkey.1996f}. Eine erste tatsächliche Realisierung eines Augmented Reality Systems erfolgte erst über 
60 Jahr später. Ivan Edward Sutherland \cite{sutherlandbio.1938m} stellte sein Projekt 1968 an der University of Utah vor. Dabei handelte es 
sich um ein sogenanntes \textit{\ac{HMD}}. Ziel dieser Entwicklung war weniger das Erweitern der Realität, sondern dreidimensionale 
Illusionen zu erzeugen die reale Objekte mit einer einfachen Grafik in Echtzeit überlagert. %\cite{display.1965f}
Trotz dessen gilt er als erste Person mit der Vision, einen Nutzer in realer Umgebung mit virtuellen Objekten interagieren zu lassen.
\\ 
Anfang der 90er Jahre prägten zwei Forscher, Thomas P. Caudell und David W. Mizell, den Begriff der Augmented Reality durch ein Pilotprojekt
bei Boeing. Das Projekt diente dazu Informationen in das Gesichtsfeld über eine Brille einzusetzen, um Arbeitern das Verlegen von Kabeln im und um das 
Flugzeug zu erleichtern. Nach dieser bahnbrechenden Erfindung begann eine stetige Weiterentwicklung der Technologie. Im Jahre 1999 wurde 
von Hirokazu Kato und Mark Billinghurst \textit{ARToolKit}, ein Computer-Vision-basiertes Tracking für AR, veröffentlicht und „löste eine 
große Welle an Forschungsarbeiten auf der ganzen Welt aus.“ \cite{springer.2019s} 
\begin{quote}
    We describe an augmented reality conferencing system which uses the overlay of virtual images on the real world. Remote collaborators 
    are represented on Virtual Monitors which can be freely positioned about a user in space. Users can collaboratively view and interact 
    with virtual objects using a shared virtual whiteboard. This is possible through precise virtual image registration using fast and 
    accurate computer vision techniques and HMD calibration. We propose a method for tracking fiducial markers and a calibration method 
    for optical see-through HMD based on the marker tracking. \cite{artoolkitsheet.1999o}
\end{quote}
Dieser Ausschnitt war der grundlegende Baustein des Durchbruchs dieser Technologie und den vorangestellten Forschungen und eine fundierte 
Grundlage für alle Forschungen und Entwicklungen die darauf folgten. 
\\ 
Heutzutage dreht sich die Entwicklung um mobile AR, welche durch die anfängliche Revolution von \textit{ARToolkit} und die darauf entstehenden 
Entwicklungen und Produktionen von großen Firmen, wie z.B. Google, Microsoft, Apple und Facebook entstand. Die zuletzt große Bewegung in dem 
Bereich der AR waren die Vorstellungen großer Software-Plattformen für mobile \acs{AR}-Applikationen. Durch \textit{Apple's ARKit} und \textit{Google's ARCore} 
kamen im Jahr 2017 zwei moderne und innovative Frameworks auf den Markt, die die Entwicklung von \acl{AR}-Applikationen stark beeinflussen.
Die Frameworks wurden bei den ersten Produktionen für Entertainment-Anwendungen genutzt, um z.B. mobile Spiele zur Unterhaltung oder 
Funktionen bei Sport-Fernsehübertragungen zur Anzeige der Entfernung des Freistoßes zu realisieren.
\\ 
\linebreak
Diese Arbeit jedoch widmet sich ausschließlich dem industriellen Aspekt und stellt andere Bereiche in den Hintergrund. Die in Kapitel 
\ref{chap:Motivation} aufgeführte Markstudie bestätigt das enorme Potential hinter \acl{AR} und deren Einsetzbarkeit in der Industrie.
Hauptsächlich in der Produktion, der Wartung oder der Reparatur von Maschinen kann Augmentierte Realität eingesetzt werden und zeigt einen 
positiv erzeugten Mehrwert. Dabei können bei Maschinen das Anzeigen von protokollierten Fehlern oder eine visuelle Hilfestellung bei Defekts, sowohl bei der 
Reparatur, als auch bei der Ersetzung einzelner Komponenten eine deutliche Reduzierung des zeitlichen Aufwands oder eine effektivere Arbeitsweise 
vorweisen. 
\\
\textit{Harvard Business Review} legte einen Vergleich offen, indem ein Techniker ein Steuergerät einer Windkraftanlage mithilfe 
eines \acs{AR}-Headsets verkabelt und in Betrieb nimmt. Alle benötigten Informationen wurden Schritt für Schritt über das Headset zur Verfügung 
gestellt. Dadurch gab es keinen Mehraufwand, z.B. das Nachschlagen in einer Dokumentation. Danach führte der Techniker den gleichen Prozess 
ohne die Hilfe der AR-Anwendung durch, lediglich mit Verwendung des vorliegenden Handbuchs, welches physisch beiseite lag.
Dieser Test bestätigte eine Leistungsverbesserung des Arbeiters beim ersten Gebrauch um \textit{34\%}.\cite{harvardbr.2017m} Diese Erkenntnis 
des Tests stützte den Gedanken der Leistungsverbesserung und der Reduzierung des zeitlichen Aufwands, somit lies dieses Resultat den 
Anlass zu dieses Ergebnis auf die Gesamtheit zu projizieren und für alle Anwendungsfälle allgemeingültig zu machen. 

% https://ntrs.nasa.gov/search.jsp?R=19830003536 
% https://ntrs.nasa.gov/archive/nasa/casi.ntrs.nasa.gov/19830003536.pdf 
% https://www.arsoft-company.com/en/dar-project/

\subsection{Virtual Reality, Augmented Reality und Mixed Reality}
Während immer mehr Leute mit dem Begriff Virtual Reality etwas anfangen können, gibt es doch noch viele Unsicherheiten bei den dazukommenden 
Begriffen der Augmented und der Mixed Reality. Diese drei Begriffe lassen sich meist nicht immer voneinander unterscheiden, da es viele 
Überschneidungen aber auch gravierende Unterschiede gibt. 
\\ 
Folgender Abschnitt beleuchtet die Unterschiede und lässt die drei Formen der erweiterten Realität voneinander unterscheidbar machen.
\subsection*{Virtual Reality}
Virtual Reality, dt. Virtuelle Realität (\acs{VR}), ist eine in Echtzeit computergenerierte, interaktive und virtuelle Umgebung. Eine Darstellung 
und gleichzeitige Wahrnehmung der Wirklichkeit in all ihren Facetten und Eigenschaften. Das Ziel dieser Technologie ist, den Nutzer von der 
Außenwelt abzuschirmen und diese durch eine computergenerierte und detaillierte Welt zu ersetzen. \cite{vr.2018n} Auch bekannt als Immersion.
\\ 
\linebreak
Die konventionelle Computergraphik ist für den Menschen spürbar nicht von belangen und weckt keine physischen Emotionen, wogegen VR diese 
etwas beeinflussen kann. Wie diese Unterschiede spürbar sind, wird in folgendem erläutert.
\\  
Die Tabelle \ref{tbl:vrtabelle} fasst die Unterscheidungsmerkmale von Virtueller Realität zur konventionellen Computergraphik zusammen. \cite{springer.2019s}
\begin{table}[!htb]
    \centering
    \begin{tabular}{ll}
      \textbf{3D- Computergraphik}  & \textbf{Virtuelle Realität} \\
      \hline
      Rein visuelle Präsentation & Multimodale Präsentation \\ %(d. h. mehrere Sinnesmodalitäten ansprechende also z. B. gleichzeitig visuelle, akustische und haptische)
      \hline
      Präsentation nicht notwendigerweise zeitkritisch & Echtzeitdarstellung \\
      \hline
      Exozentrische Perspektive & Egozentrische Perspektive \\
      \hline
      Statische Szene oder vorberechnete Animation & Echtzeitinteraktion und -simulation \\ 
      \hline
      2D-Interaktion (Maus, Tastatur) & 3D-Interaktion (Körperbewegung, -gestik) \\ 
      \hline
      Nicht-immersive Präsentation & Immersive Präsentation \\ 
    \end{tabular}
    \caption{Merkmale der Computergraphik gegenüber der VR \cite{springer.2019s}}
    \label{tbl:vrtabelle}
    % Verweis im Text mittels \ref{tbl:vrtabelle}
\end{table}
\\ 
\linebreak 
Virtuelle Realität ist immer in Verbindung mit \acl{HMD}s zu betrachten, da ein Gerät benötigt wird, welches den Nutzer von der realen Welt 
abschottet und in die virtuelle Welt begleitet. Diese werden auf Hochtouren von großen Firmen, wie Microsoft, Sony, Facebook etc. entwickelt. 
Die erste entwickelte und auf dem Markt veröffentlichte Brille war die HoloLens von Microsoft, gefolgt von der Brille Namens Oculus Rift von 
Facebook usw. 
\\ 
Mit der stetigen Weiterentwicklung dieser Brillen und der Technologie wird versucht nach und nach mehr Sinne des Menschen manipulieren zu 
können, bzw. das Spiel- und Gefühlserlebnis bei Konsolen-Spielen immer realistischer zu gestalten. Allerdings sind die Möglichkeiten im 
Massenmarkt stark beschränkt auf die folgenden aufgelisteten Sinne: 
\begin{itemize}
    \item Sehen: Durch \acl{HMD}s (Oculus Rift), die die reale Welt abschirmen und vom Nutzer nicht mehr wahrgenommen werden kann
    \item Hören: Durch Kopfhörer, somit werden Geräusche der Realität übertönt 
    \item Fühlen: Durch Controller mit haptischem Feedback, um Ereignisse der virtuellen Welt physisch spürbar zu gestalten. 
\end{itemize}
Die Entwicklung dieser Technologie wird uns in Zukunft weiter verfolgen. Vielleicht gibt es irgendwann die Möglichkeit weitere Sinne virtuell 
zu steuern. An den Universitäten von Singapur und Tokio zum Beispiel, gibt es Forscher-Teams die ein großes Budget zur Verfügung haben, um 
dieser Sinnestrübung auf den Grund zu gehen. Sie fanden heraus, das thermische und elektrische Stimulationen bestimmte Geschmackseindrücke 
vermitteln können. Auch Wissenschaftler in China und an der Oxford University haben herausgefunden, dass bestimme Audiosignale mit einem süßen 
Geschmack assoziiert werden. \cite{sinnesforschung.2017m} Forscher aus aller Welt gehen mit einer bestimmten Ernsthaftigkeit die Möglichkeiten 
der Sinneserweiterung in der virtuellen Realität an.   
\subsection*{Augmented Reality}
\ac{AR} setzt im Gegenzug zu \ac{VR} auf das tatsächliche Erweitern der Realität durch das Einblenden von Informationen, Vorgängen, Hilfestellungen
oder die Wegbeschreibung bei Head-Up Displays in Personen-, Kraft- und Nutzfahrzeugen, während \acl{VR} den Nutzer in eine völlig eigene 
Welt entlockt und von der Realität abkapselt. Bei \acs{AR} soll dem Nutzer die zu bewältigenden Aufgaben und dessen Bestreben vereinfacht und 
gestützt werden ohne die Realität außer Acht zu lassen. Darüberhinaus bietet \acl{AR} deutlich mehr Ansatzmöglichkeiten diese Technologie 
umzusetzen und zu nutzen, z.B. wie erwähnt durch Head-Up Displays bei Autos und Flugzeugen, durch Smartphones und Tablets die durch die Kamera 
die Realität erweitern, Brillen mit eingeblendeten Projektionen, wie die Oculus Rift und anderweitige große Projektionen. 
\\ 
In Kapitel \ref{subchap:Varianten der AR} wird auf die Varianten genauer eingegangen. 

\subsection*{Mixed Reality}
Dem Namen entsprechend ist \ac{MR} eine Mischung aus \acl{AR} und \acl{VR}, jedoch die Art und Weise ist dabei entscheidend. Es gibt \acs{MR}
-Brillen, die die reale Welt als Basis der Interaktion nehmen und Brillen die alleinig auf digitalen Bildern aufbauen. \cite{mr.2018o}
\subsubsection*{Mixed Reality bezüglich der realen Welt}
Diese Art der erweiterten Realität basiert auf den gegebenen Grundlagen der \acs{AR}. Ähnlich zu \acl{AR} werden Objekte der realen Welt 
hinzugefügt, indem sie an bestimmte Stellen projiziert werden. \acl{MR} baut darauf auf und verankert die digitalen Objekte mit dem realen
dreidimensionalen Raum, sodass eine Realitätsnahe Interaktion stattfinden kann. Virtuelle und echte Welt verschmelzen dadurch endgültig zu 
einer einzigen Welt. 

\subsubsection*{Mixed Reality bezüglich der virtuellen Welt}
Hinsichtlich der \acs{VR} gibt es zu Anfang keine grundlegenden Änderungen gegenüber der \acs{MR}. Hierbei kann der Nutzer durch 
\acs{MR} ebenso die Außenwelt ausblenden und sich lediglich auf die virtuelle Wahrnehmung fixieren. Erst in der Benutzung werden die Unterschiede 
zu \acs{VR} sichtlich und dieser ist gravierend. Der Nutzer kann sich frei bewegen, d.h. die Bewegungsfreiheit wird nur durch die reale Umgebung 
eingeschränkt, während \acl{VR} sich auf einen sensorgestützten Raum begrenzt. \cite{vr.2018n}
\\ 
Bei \acs{MR} wird jeder Schritt und jede Bewegung in die computergenerierte Welt übertragen und schafft so deutlich mehr Bewegungsfreiheiten.
\\ 
\linebreak 
Ein großer Investor dieser Technologie ist Microsoft mit \ac{WMR}, wobei bereits schon Erfolge erzielt werden konnten. Eine Standhafte Plattform 
die nur darauf wartet ausgebaut und intensiver genutzt zu werden. Für die Plattform vorgesehene Brillen gibt es schon viele Produzenten, wie z.B. 
die \textit{HMD Odyssey} von Samsung oder der \textit{Explorer} von Lenovo.

\subsection{Varianten der Augmented Reality}
\label{subchap:Varianten der AR}
Augmented Reality Anwendungen funktionieren alle nach dem gleichen Prinzip und verfolgen das gleiche Ziel, die Realität durch digitale Informationen 
oder Objekte zu erweitern. Die einzige deutliche Abweichung ist das jeweilige Endgerät und die technische Umsetzung in Zusammenhang mit der 
Hardware. Auf zwei dieser Varianten, mobiles und Smart-Brillen- oder auch Headset-AR genannt, wird in Folgendem eingegangen.

\subsection*{Mobiles AR}
Smartphones sind in unserer Gesellschaft nicht mehr wegzudenken und haben sich fest etabliert. Durch die vielseitige und alltägliche Nutzung 
von Tablets und Smartphones eröffnete diese Sparte eine gute Möglichkeit \acl{AR} in den Alltag zu integrieren. Somit beleibt der ständige 
Gebrauch dieser Technologie nicht aus und bereichert die Art und Weise Spiele und Anwendungen zu entwickeln und zeigt den Fortschritt der technologischen 
Aspekte sowohl software-, als auch hardware-technisch.
Durch die im Smartphone integrierte Kamera werden Live-Aufnahmen analysiert und dienen als Ausgangssituation für die \acs{AR}-Anwendung. Durch 
die gegebene Möglichkeit der vorhandenen Kamera, kann ein Echtzeithintergrund erzeugt und zusätzlich mit Informationen per Overlay dargestellt 
werden. Je nach Anwendung kann der Nutzer auf verschiedene Weise mit den eingeblendeten und virtuellen Objekten interagieren, z.B. durch das 
Bewegen des Geräts, um das Objekt aus verschiedenen Blickwinkeln anschauen zu können, oder der direkten Interaktion mit dem Objekt durch 
Drehen, Skalieren oder Verschieben am Bildschirm. Ein Vorreiter der mobilen AR ist das 2016 auf dem Markt erschienenen \textit{Pokémon Go} 
von Niantic, das den Ansatz der \acs{AR} prägt. \cite{pokemongo.2016a}
\\ 
\linebreak
Die weit verbreitete social Media Applikation \textit{Snapchat} baut seit geraumer Zeit ebenso auf \acl{AR}, um Bilder lebhafter zu 
gestalten, wie der folgenden Abbildung \ref{pic:snapchatAR} zu entnehmen ist. 
\\ 
\linebreak 
\begin{figure}[hbt!]
    \centering
    \includegraphics[width=10cm,height=7.5cm,keepaspectratio]{2Grundlagen/Bilder/snapchatAR.jpeg}
    \caption{Mobile-AR in Snapchat}
    \label{pic:snapchatAR}
\end{figure}
\subsection*{Smart-Brillen AR}
Unter Smart- oder Datenbrillen und Smart Glasses wird ein Konstrukt verstanden, das eine Brille mit einem tragbaren Minicomputer verwirklicht. 
Dabei werden Informationen über kleinste Monitore oder Prismen ausgegeben und dem Nutzer zusätzlich in das Sichtfeld projiziert. Im Gegensatz
zur mobilen \acs{AR} hat der Nutzer kein Gerät in der Hand und ist somit in seiner Bewegung weniger eingeschränkt und flexibler. 
\\ 
\linebreak
Die Technologie der Smart-Brillen ist ein ähnliches Konzept zu der \acs{VR}-Brille, allerdings befindet sich die Smart-Brille für den 
öffentlichen Gebrauch noch in der Entwicklungsphase, da nur bedingte Funktionen möglich sind und die Kosten für den Normalverbraucher nur 
bedingt tragbar erscheinen. Mit der überarbeiteten Version der \textit{HoloLens 2} von Microsoft ist eine deutlich komfortablere 
Möglichkeiten des alltäglichen Gebrauchs geboten, da diese im Vergleich zum Vorgängermodell deutlich komfortabler, leichter und 
handlicher ist. Das Design der Datenbrille ist von einer normalen Brille noch weit entfernt, ermöglicht mittlerweile aber das 
uneingeschränkte Sehen und Wahrnehmen der Umgebung. Die Bedienung des Geräts basiert auf schon vorhandenen Möglichkeiten die bereits in anderen 
Geräten Anwendung finden. Darunter gibt es am Gerät angebrachte Touchsensoren, Handbewegungen und -gesten und Eye-Tracking. 
\begin{figure}[hbt!]
    \centering
    \includegraphics[width=10cm,height=7.5cm,keepaspectratio]{2Grundlagen/Bilder/smartglassAW.png}
    \caption{Test der HoloLens 2}
    \label{pic:testholo}
\end{figure}
\\ 
\linebreak 
Speziell im industriellen Bezug bieten die neuentwickelten Brillen ein komfortables und akzeptables Gewicht, sodass diese ohne große 
Probleme dauerhaft tragbar sind und den Mitarbeiter bei seiner Arbeit nicht übermäßig einschränken.
\\ 
\linebreak
Die Programmierung solcher \acs{AR}-Brillen laufen häufig über eigene \acs{SDK}s und plattformunabhängige Laufzeit- und 
Entwicklungsumgebungen, z.B. Unity oder Unreal Engine. Diese gelten als führende Produkte im Bereich der 3D-Echtzeitdarstellung.
Ein marktführendes Produkt ist unter anderem die Microsoft HoloLens 2, die der folgenden Abbildung zu entnehmen ist. 
\begin{figure}[hbt!]
    \centering
    \subfigure[Google Glass 2]{\includegraphics[width=7.5cm,height=5cm,keepaspectratio]{2Grundlagen/Bilder/googleglass.png}}
    \subfigure[HoloLens 2]{\includegraphics[width=7.5cm,height=5cm,keepaspectratio]{2Grundlagen/Bilder/hololensside.png}}
    \caption{Datenbrillen (\acs{HMD})}
    \label{pic:datenbrillen}
\end{figure}
\\ 
\linebreak
Eine schwer zu umgehende Herausforderung der \acl{AR} ist die Bestimmung der Position in der Daten und Objekte projiziert werden. Auf die 
Ansätze, diese Herausforderung zu bewältigen, wird in folgendem Abschnitt \ref{sec:posi} eingegangen.
\subsection{Positionsbestimmung}
\label{sec:posi}
Um ein digitales Objekt als Overlay dem Kamera-Live-Bild hinzuzufügen, werden genauestens definierte Positionen benötigt. Diese Positionen 
können durch unterschiedliche Ansätze ermittelt werden. Je nach Anwendungsfall, z.B. als Navigation, Routenplaner oder Google Maps 
\textit{„Live View“} \cite{googleliveview.2019a}, reicht eine etwas ungenauere Positionsbestimmung per \acs{GPS}, da in Relation zur 
realen Welt eine Abweichung um Zentimeter oder wenige Meter nicht von belangen ist. Bei Positionsbestimmungen auf kleinstem Raum ist eine 
genaue Lokation wichtig und basiert auf einem deutlich präziseren Ansatz. 
\\ 
Welche oben genannten Ansätze es gibt und welche Unterschiede zu beachten sind, wird in Folgendem näher darauf eingegangen. 
\subsubsection*{Marker-basierte Positionsbestimmung}
Speziell bei der Marker-basierten Positionsbestimmung gibt es verschiedene Möglichkeiten den Marker zu gestalten. Es können 
Binär- oder QR-Codes als Markierung verwendet werden, ein Beispiel eines solchen Codes ist der Abbildung \ref{pic:markerARpos} zu entnehmen. 
Diese Codes sind meistens quadratisch und haben ein eindeutiges Zeichen in der Mitte. Um die Rechenzeit gering zu halten gibt es einfache Muster, 
wie die Abbildung \ref{pic:markerARpos} zeigt.
\begin{figure}[hbt!]
    \centering
    \includegraphics[width=5cm,height=5cm,keepaspectratio]{2Grundlagen/Bilder/bildmarkerAR.png}
    \caption{Marker-basierte Augmented Reality Positionsbestimmung}
    \label{pic:markerARpos}
\end{figure}
Neben der einfachen Markierungserkennung gibt es eine weiterentwickelte Möglichkeit der Bild- sowie der Objekterkennung. Diese sind Ansätze 
die grundlegend auf dem Ausgangspunkt des Binär-Code-Verfahrens aufbauen. Eine detailliertere Erläuterung der soeben genannten 
Erkennungsmöglichkeiten findet im Rahmen dieser Arbeit nicht statt. Allerdings wird allgemein kurz auf die Funktionsweise einer 
Marker-basierten Positionsbestimmung eingegangen.
\\ 
Durch ein Kamerabild wird nach einem vordefinierten Marker, bzw. nach einer festgelegten Markierung gesucht. Ist diese mit der Kamera erfasst, 
wird die Markierung durch Bildverarbeitungsalgorithmen und bestimmter Filterung eindeutig identifiziert. Mit den gewonnen Informationen und 
der Übereinstimmung des angegebenen Codes wird die Position, sowie die Orientierung des Markers berechnet. Mit den Angaben der Lage und 
Orientierung wird auf der Markierung das anzuzeigende digitale Objekt generiert und als Overlay über dem Kamerabild angezeigt. Die Markierung 
dient sozusagen als Grundlage um den digitalen Gegenstand überhaupt anzeigen zu können.
\\ 
Da der Marker immer im Blickfeld der Kamera sein muss, um das virtuelle Objekt anzuzeigen, bringt diese Art der Positionsbestimmung eine 
enorme Einschränkung mit sich, die in diesem Bezug unumgänglich ist. 
\\ 
\linebreak
Ein weiterer Ansatz der Positionsbestimmung ist als Überbegriff das Gegenstück zur oben aufgeführten Lokalisierung von Markern, die 
sogenannte Marker-unabhängige Positionsbestimmung, welche ebenso verschiedene Ausführungen vorweist. 
\subsubsection*{\acs{GPS}-basierte Positionsbestimmung}
Die Methode des \acs{GPS}-basierten Positionsbestimmungsverfahren verwendet hauptsächlich die Koordinaten der realen Welt. In Zusammenarbeit 
mit zusätzlichen im Anwendungsgerät verbauten internen Sensoren, bspw. Positions-, Geschwindigkeits- und Beschleunigungssensoren und Teile des 
\ac{INS}, z.B. dem Gyroskop, kann diese Art der Positionsbestimmung optimal Anwendung finden. Allerdings werden dabei deutlich mehr Komponenten
benötigt und ist deutlich komplexer umzusetzen, als ein markerbasiertes System. 
\\ 
Bei einem Anwendungsfall von \acs{GPS}-basierter Positionsbestimmung geht es meist um Routenplaner, Navigation oder Szenarien die sich 
auf offenen Flächen abspielen, da im Gegensatz zu Marker-basierten Anwendungen das Größenverhältnis deutlich geräumiger ist. Der Benutzer ist 
nicht auf einen bestimmten Bezirk beschränkt und ist nicht auf die millimetergenauer Darstellung angewiesen.
\\ 
\linebreak
Eine Alternative zu \acs{GPS} ist die Anwendung des \acs{SLAM}-Verfahrens. Dabei wird eine virtuelle Karte, bzw. ein geometrischen Modell der 
Umgebung erstellt. Wichtige Grundlagen zum Erzeugen eines solchen Modells sind eigenständig gefundene Landmarken die gleichzeitig 
lokalisiert werden. Darauf folgt ein Vergleich der Pose, der Position des Geräts und der geschätzten Karteninformationen aus dem Scan der 
Umgebung. 
Damit sind im erweiterten Sinne Marker geschaffen, die Anhaltspunkte für \acl{AR}-Interaktionen schaffen.
\\ 
In Kapitel \ref{chap:SLAM} wird die Thematik des \acs{SLAM}-Verfahrens genauestens erläutert. 

\subsection{Augmented Reality in der Industrie}
Das weit gefächerte Portfolio der \acl{AR} umfasst viele Anwendungsbereiche und Fachgebiete. Selbst dem übergeordneten Bereich der Industrie 
gibt es viele verschiedene Einsatzgebiete. Aus den vielen Möglichkeiten der Anwendung haben sich über die Jahre der Entwicklung der Technologie 
einige Gebiete in der Industrie herauskristallisiert, die besonders hohen Nutzen stiften. \cite{einsatzgebietear.2017a} In den Bereichen 
Instandhaltung und Wartung, Betrieb und Training ist \acl{AR} auf bestem Wege fester Bestandteil des Alltags zu werden. Bei dieser Betrachtung 
ist es sinnvoll sich auf die damit einhergehenden Lösungen zu fokussieren, die Reduktion der Kosten und dem Zeitaufwand, sowie die Verbesserung 
der Sicherheit. \cite{studieptc.2020j} 
\\ 
\linebreak
Bei einer Wartung oder Reparatur einer Maschine sind notwendige Informationen direkt greifbar und werden Schritt für Schritt angezeigt, sodass 
in bestimmten Situationen selbst ein Leihe die Anweisungen befolgen könnte. So werden zusätzliche Recherchearbeiten oder Unklarheiten über 
Vorgänge aus dem Weg geräumt. Ein Mitarbeiter kann so mit einem Tablet oder einer Smart-Glas Anweisungen visuell auf die realen Maschinen 
projizieren und Arbeitsschritte im Sichtfeld anzeigen lassen. Ebenso geht dieser Vorgang auch bei der Produktion von Bauteilen o.ä., indem 
eine \acs{AR}-Anwendung Anweisungen und Prozessschritte, z.B. auf das Werkstück oder Produkt, projiziert.
\\ 
Die Inbetriebnahme oder Bedienung einer komplexen Anlage oder Maschine ist nach herkömmlichen Standard enorm zeitintensiv und dadurch können
zusätzlich viele Fragen aufkommen, die meist den Prozess noch länger gestalten als vorgesehen. \acs{AR}-Anwendungen können 
Bedienungsanleitung oder -hilfen digitalisiert, indem Informationen, Inhalte oder Bedienelemente durch \acl{AR} auf der Anlage platziert werden. 
\\ 
In Fortbildungen, Schulungen oder Einarbeitungen in neue Geräte kann \acs{AR} durchaus von großem Vorteil sein. Mit wenig Aufwand, einem 
effektiveren Training können Schulungen interaktiver und vor allem sicherer abgehalten werden. Durch die Visualisierung der Trainings- und 
Schulungsinhalten verbessert sich der Lernprozess und damit wird auch die Nutzung der Maschinen verständlicher. \cite{einsatzgebietear.2017a}

\section{SLAM - Simultanious Localization And Mapping}
\label{chap:SLAM}
Als Simultaneous Localization and Mapping auch \acs{SLAM}-Problem gennant, bezeichnet man die Aufgabe, die Trajektorie \footnote{Lösungskurve oder Bewegungspfad eines Objekts} 
samt Orientierungsinformation einer sich bewegenden Plattform, z.B. ein Smartphone, Tablet oder jegliche Art von Roboter, aus 
Beobachtungen zu schätzen und gleichzeitig aus den gewonnenen Informationen eine Karte der Umgebung zu erstellen.
Diese Aufgabe ist für den weiteren Prozess von hoher Bedeutung. Zum einen sollen die generierten Karten sehr präzise sein, um einen hohen 
Wert für den Nutzer oder für spezielle Anwendungen, die auf der Karte aufbauen, darzustellen. Zum anderen benötigen autonome Roboter, 
beispielsweise Saug- oder Mähroboter, ein solch erzeugtes geometrisches Modell der Umgebung, um zielgerichtet selbstständig navigieren zu 
können. %\cite{slamdefi.2016a} 
\begin{quote}
    Das Simultaneous Localization and Mapping oder kurz SLAM Problem behandelt das gleichzeitige Schätzen der Position und Ausrichtung einer 
    mobilen Plattform im Raum anhand der sich an Bord befindlichen Sensoren sowie den Aufbau eines Modells der Umgebung. Dieses Problem ist 
    von großer praktischer Relevanz und ist Kernbestandteil der meisten mobilen Sensor- systeme. \cite{slamdefi.2016a}
\end{quote}
1986 wurden auf der \textit{IEEE Robotics and Automation Conference} erste mathematische Definitionen vorgenommen, die mittels statischer 
Theorien ermittelt und mit ersten Studien belegt wurden. Einige Jahre später,im Jahr 1995, wurde das \acs{SLAM} Problem erstmals auf dem 
internationalen Symposium für Robotikforschung (\textit{ISRR'95}) vorgestellt. Die Forschungen hielten an, bis auf der (\textit{ISRR'99}) 
die erste \acs{SLAM} Sitzung stattfand. 
\subsection{Definition des Problems}
Angenommen ein Roboter startet in einer Position auch Pose gennant und Konfiguration \textit{p0} und bewegt sich durch eine ihm 
unbekannte Umgebung, wobei diese nicht vorhandenen Kenntnise das Hauptproblem darstellen. Die Einstellung beinhaltet Position und Ausrichtung 
des Roboters. Je nach Bewegung in Raum oder Ebene ist die Pose meist als 3- oder 6-dimensionaler Vektor abgebildet. Die Bewegung des 
Roboters wird durch bekannte Kontrollkommandos \textit{u} angewiesen, allerdings mit einer gewissen Unsicherheit versehen. Dabei wird 
zwischen den Zeitpunkten \textit{t-1} und \textit{t} die Bewegung des Roboters mit \textit{ut} beschrieben und somit auch die 
unterschiedlichen Posen \textit{pt-1} nach \textit{pt}. Die Umgebung wird parallel dazu über diverse Sensoren, z.B. interne Sensoren, bspw. 
Gechwindigkeits- oder Positionssensoren, und externer Senoren, unter anderem Abstands- oder taktile Sensoren. Neben der Protokollierung der 
Position, bei der es zu Störungen oder Berechnungsfehler kommen kann, gibt es Beobachtungen durch Sensoren die verrauscht \textit{zt} sind.
\\ 
\linebreak
Mit diesen vorhandenen Werten ist das Ziel die Schätzung der Trajektorie \textit{p0:T=[p0,p1,...,pT]T} des Roboters von Beginn der 
Fortbewegung bis zum Zeitpunkt \textit{T}. Gleichzeitig zur Berechnung der Trajektorie wird eine Karte \textit{m} des Umfelds geschätzt, 
deren Darstellung den Anforderungen entsprechend angepasst werden kann. Mit den Anforderungen sind verschiedene Repräsentationen der Karten 
gemeint, darunter beispielsweise eine Veranschaulichung von Punktansammlungen an Gegenständen, gerenderte Oberflächenmodelle oder 
2D-Rasterkarten und verstärkt visualisierte 3D-Voxelkarten.\footnote{(Zusammensetzung aus dem englischen volume \textit{vox} und elements 
\textit{el}), bez. einen Gitterpunkt in einem dreidimensionalen Gitter.} Basierend auf den Sensormessintervallen \textit{z1:T} und den dabei
stattfindenden Kontrollkommandos \textit{u1:T} wird die Karte des Umfeldes und alle Positionen bestimmt. Die Wahrscheinlichkeitsverteilung 
\textit{p(p0:T,m|z1:T,u1:T)} wird durch die geschätzten Werte \textit{p0:T} und \textit{m} berechnet. 
\\ 
\linebreak
Die Berechnung der Wahrscheinlichkeitsverteilung ist auch unter dem Namen \textit{Offline-\acs{SLAM}} bekannt. In der Praxis ist allerdings 
die Schätzung der Position \textit{xt} und der Karte der Umgebung durch \textit{p(pt,m|z1:t,u1:t)} interessanter, da Roboter Entscheidungen 
basierend auf aktuellen Informationen, z.B. der Posenschätzung und dem Umgebungsmodell, treffen sollen. Die Variante der Echtzeitschätzung 
ist auch als \textit{Online-SLAM} bekannt. \cite{slamdefi.2016a}

\subsection{Localization}
Damit das Endgerät, Smartphone oder der Roboter seine Position in Erfahrung bringen und schätzen kann, werden Informationen und Möglichkeiten 
benötigt die Bewegung in irgendeiner Form zu messen. Da das Nutzergerät eine eigene virtuelle Karte, unabhängig von der \acs{GPS}-basierten 
Position, generiert, ist das \textit{Tracking} über die Weltkoordinaten nicht Bestandteil des eigentlichen Verfahrens. Für die Erfassung der 
internen Systemzustände gibt es sogenannte interne Sensoren die in dem Roboter zur Verfügung stehen. Bestandteile dieser internen 
Sensorik sind unter anderem Positions-, Geschwindigkeits-, Beschleunigungssensoren und das \acl{INS}. Für die Positions- und 
Geschwindigkeitserkennung gibt es z.B. einen optischen Codierer, welcher durch Lichtimpulse die Geschwindigkeit als auch die zurückgelegte 
Strecke schätzen kann. Das \acs{INS} besteht aus Lagesensoren und einem Kreiselkompass (Gyroskop), diese sind essentiell für die Bestimmung 
der Orientierung und Neigung des Geräts. Ausgehend von der Erdkugel beziehen sich diese Sensoren auf das gegebene inertiale Koordinatensystem.
\subsection{Mapping}
Um neben der Berechnung der Position durch die interne Sensorik die Umgebung registrieren und schätzen zu können, gibt es externe Sensoren 
die sich mit der Erfassung der Umwelt beschäftigen. Auch hier gibt es viele Arten von Sensoren mit denen es ermöglicht wird die Umgebung 
wahrzunehmen, darunter Taktile Sensoren, Näherungs-, Abstands-, Positionssensoren und Visuelle Sensoren. Hinsichtlich Abstandssensoren die 
Mesungen zwischen Gegenstand und Sensor durchführen, gibt es allgemein erhebliche Vorteile gegenüber den Näherungssensoren. Vorteile sind 
beispielsweise eine größere Reichweite und Blickfeld als Näherungssensoren, die Ermittlung der genauen Entfernung zu Gegenständen und sie 
sind besser zur Erfassung geometrischer Umweltinformationen geeignet. Für die Messung des Abstandes eignet sich die \ac{TOF}-Kamera, die 
mit dem Laufzeitverfahren Distanzen messen und berechnen kann. 
\\ 
Das Laufzeitverfahren funktioniert wie folgt:
\begin{align*}
    \textit{d} = \frac{1}{2}\textit{ct}
\end{align*}
Abstand zwischen der Zielfläche und dem Sensor \textit{d} ist das Produkt aus der Signalgeschwindigkeit \textit{c} und der messbaren Laufzeit 
\textit{t}. Beispiele für solche \acs{TOF}-Kameras und deren Repräsentation sind folgender Abbildung (\ref{pic:tofCam}) zu entnehmen.
\begin{figure}[hbt!]
    \centering
    \includegraphics[width=13cm,height=13cm,keepaspectratio]{2Grundlagen/Bilder/tof_kamera.png}
    \caption{Time-of-Flight Kamera und deren Repräsentation \cite{robotik.2020m}}
    \label{pic:tofCam}
\end{figure}
\\ 
Auf der linken Seite ist eine PMD-Kamera zu sehen, gefolgt von einer SwissRanger-Kamera und auf der rechten Seite eine IFM-Kamera.
\subsection{Verfahren zur Lösung des SLAM Problems}
Das Lösen des \acs{SLAM}-Problems wird in der Praxis mit den angestellten Verfahren, die sich bei der Anwendung duchsetzten, durchgeführt. In 
folgenden Abschnitten werden zwei dieser Verfahren erläutert, sodass ein Grundverständnis vorhanden ist. Zuerst wird auf das Graph-basierte 
\acs{SLAM}, dem Lösen des Problems unter Anwendung der kleinsten-Quadrate Methode, eingegangen, darauffolgend die Lösung durch einen rekursiven 
Ansatz, dem erweiterten Kalman Filter (\acs{EKF}).

\section{Quaternionen}
\label{chap:Quaternionen}
%%%%%%%%%%%%%%%%%%%%%%%%%%%%%%%%%%%%%%%%%%%%
% -----------> tbd... !! <-----------------




%%%%%%%%%%%%%%%%%%%%%%%%%%%%%%%%

\subsection{Transformation}
\subsection{Rotation}


\section{SLAM - Simultanious Localization And Mapping}
\label{chap:SLAM}
Als Simultaneous Localization and Mapping auch \acs{SLAM}-Problem gennant, bezeichnet man die Aufgabe, die Trajektorie\footnote{Lösungskurve oder Bewegungspfad eines Objekts} 
samt Orientierungsinformation einer sich bewegenden Plattform, z.B. ein Smartphone, Tablet oder jegliche Art von Roboter, aus 
Beobachtungen zu schätzen und gleichzeitig aus den gewonnenen Informationen eine Karte der Umgebung zu erstellen.
Diese Aufgabe ist für den weiteren Prozess von hoher Bedeutung. Zum einen sollen die generierten Karten sehr präzise sein, um einen hohen 
Wert für den Nutzer oder für spezielle Anwendungen, die auf der Karte aufbauen, darzustellen. Zum anderen benötigen autonome Roboter, 
beispielsweise Saug- oder Mähroboter, ein solch erzeugtes geometrisches Modell der Umgebung, um zielgerichtet selbstständig navigieren zu 
können. %\cite{slamdefi.2016a} 
\begin{quote}
    Das Simultaneous Localization and Mapping oder kurz SLAM Problem behandelt das gleichzeitige Schätzen der Position und Ausrichtung einer 
    mobilen Plattform im Raum anhand der sich an Bord befindlichen Sensoren sowie den Aufbau eines Modells der Umgebung. Dieses Problem ist 
    von großer praktischer Relevanz und ist Kernbestandteil der meisten mobilen Sensor- systeme. \cite{slamdefi.2016a}
\end{quote}
1986 wurden auf der \textit{IEEE Robotics and Automation Conference} erste mathematische Definitionen vorgenommen, die mittels statischer 
Theorien ermittelt und mit ersten Studien belegt wurden. Einige Jahre später,im Jahr 1995, wurde das \acs{SLAM} Problem erstmals auf dem 
internationalen Symposium für Robotikforschung (\textit{ISRR'95}) vorgestellt. Die Forschungen hielten an, bis auf der (\textit{ISRR'99}) 
die erste \acs{SLAM} Sitzung stattfand. 
\subsection{Definition des Problems}
Angenommen ein Roboter startet in einer Position auch Pose gennant und Konfiguration \textit{p0} und bewegt sich durch eine ihm 
unbekannte Umgebung, wobei diese nicht vorhandenen Kenntnise das Hauptproblem darstellen. Die Einstellung beinhaltet Position und Ausrichtung 
des Roboters. Je nach Bewegung in Raum oder Ebene ist die Pose meist als 3- oder 6-dimensionaler Vektor abgebildet. Die Bewegung des 
Roboters wird durch bekannte Kontrollkommandos \textit{u} angewiesen, allerdings mit einer gewissen Unsicherheit versehen. Dabei wird 
zwischen den Zeitpunkten \textit{t-1} und \textit{t} die Bewegung des Roboters mit \textit{ut} beschrieben und somit auch die 
unterschiedlichen Posen \textit{pt-1} nach \textit{pt}. Die Umgebung wird parallel dazu über diverse Sensoren, z.B. interne Sensoren, bspw. 
Gechwindigkeits- oder Positionssensoren, und externer Senoren, unter anderem Abstands- oder taktile Sensoren. Neben der Protokollierung der 
Position, bei der es zu Störungen oder Berechnungsfehler kommen kann, gibt es Beobachtungen durch Sensoren die verrauscht \textit{zt} sind.
\\ 
\linebreak
Mit diesen vorhandenen Werten ist das Ziel die Schätzung der Trajektorie \textit{p0:T=[p0,p1,...,pT]T} des Roboters von Beginn der 
Fortbewegung bis zum Zeitpunkt \textit{T}. Gleichzeitig zur Berechnung der Trajektorie wird eine Karte \textit{m} des Umfelds geschätzt, 
deren Darstellung den Anforderungen entsprechend angepasst werden kann. Mit den Anforderungen sind verschiedene Repräsentationen der Karten 
gemeint, darunter beispielsweise eine Veranschaulichung von Punktansammlungen an Gegenständen, gerenderte Oberflächenmodelle oder 
2D-Rasterkarten und verstärkt visualisierte 3D-Voxelkarten\footnote{(Zusammensetzung aus dem englischen volume \textit{vox} und elements 
\textit{el}), bez. einen Gitterpunkt in einem dreidimensionalen Gitter.}. Basierend auf den Sensormessintervallen \textit{z1:T} und den dabei
stattfindenden Kontrollkommandos \textit{u1:T} wird die Karte des Umfeldes und alle Positionen bestimmt. Die Wahrscheinlichkeitsverteilung 
\textit{p(p0:T,m|z1:T,u1:T)} wird durch die geschätzten Werte \textit{p0:T} und \textit{m} berechnet. 
\\ 
\linebreak
Die Berechnung der Wahrscheinlichkeitsverteilung ist auch unter dem Namen \textit{Offline-\acs{SLAM}} bekannt. In der Praxis ist allerdings 
die Schätzung der Position \textit{xt} und der Karte der Umgebung durch \textit{p(pt,m|z1:t,u1:t)} interessanter, da Roboter Entscheidungen 
basierend auf aktuellen Informationen, z.B. der Posenschätzung und dem Umgebungsmodell, treffen sollen. Die Variante der Echtzeitschätzung 
ist auch als \textit{Online-SLAM} bekannt. \cite{slamdefi.2016a}

\subsection{Localization}
Damit das Endgerät, Smartphone oder der Roboter seine Position in Erfahrung bringen und schätzen kann, werden Informationen und Möglichkeiten 
benötigt die Bewegung in irgendeiner Form zu messen. Da das Nutzergerät eine eigene virtuelle Karte, unabhängig von der \acs{GPS}-basierten 
Position, generiert, ist das \textit{Tracking} über die Weltkoordinaten nicht Bestandteil des eigentlichen Verfahrens. Für die Erfassung der 
internen Systemzustände gibt es sogenannte interne Sensoren die in dem Roboter zur Verfügung stehen. Bestandteile dieser internen 
Sensorik sind unter anderem Positions-, Geschwindigkeits-, Beschleunigungssensoren und das \acl{INS}. Für die Positions- und 
Geschwindigkeitserkennung gibt es z.B. einen optischen Codierer, welcher durch Lichtimpulse die Geschwindigkeit als auch die zurückgelegte 
Strecke schätzen kann. Das \acs{INS} besteht aus Lagesensoren und einem Kreiselkompass (Gyroskop), diese sind essentiell für die Bestimmung 
der Orientierung und Neigung des Geräts. Ausgehend von der Erdkugel beziehen sich diese Sensoren auf das gegebene inertiale Koordinatensystem.
Mittels \textit{Odometrie} können die von den Sensoren bereitgestellten Informationen über Position und Orientierung berechnet werden. Radgetriebene 
Systeme führen die Berechnung durch, da diese basierend auf dem Durchmesser des Rades berechnet wird. In Kombination mit \textit{Koppelnavigation}\footnote{Engl. dead reckoning, ist die laufende Positionsbestimmung eines bewegten Objekts infolge der Bewegungsrichtung und Geschwindigkeit \cite{koppelnavigation.2019j}}
gilt dieses Verfahren für Roboter, bzw. Fahrzeuge auf Land, als Grundlage der Navigation. Durch die theoretische Berechnung werden 
Fehlerbetrachtungen, z.B. Verschleiß, Schlupf oder Unrundheit von Rädern, vernachlässigt und ist somit nicht als alleiniges Verfahren einzusetzen. 
\begin{align*}
    \Delta U = \frac{\pi D}{\textit{n}C}N
\end{align*}
Mit \textit{D} als Raddurchmesser, \textit{n} als Getriebeübersetzung. \textit{C} als Enkoderauflösung und \textit{N} als Anzahl der 
Enkoderimpulse wird die Positions- und Orientierung berechnet. 
\\ 
Smartphones berechnen die Fortbewegung über gegebene Sensoren, darunter Beschleunigungs-und Neigungssensoren und Gyroskop. Dieses Zusammenspiel an 
Sensoren ermöglicht die Wahrnehmung der Positionsveränderung und kann somit diese bestimmen.
\subsection{Mapping}
Um neben der Berechnung der Position durch die interne Sensorik die Umgebung registrieren und schätzen zu können, gibt es externe Sensoren 
die sich mit der Erfassung der Umwelt beschäftigen. Auch hier gibt es viele Arten von Sensoren mit denen es ermöglicht wird die Umgebung 
wahrzunehmen, darunter Taktile Sensoren, Näherungs-, Abstands-, Positionssensoren und Visuelle Sensoren. Hinsichtlich Abstandssensoren die 
Messungen zwischen Gegenstand und Sensor durchführen, gibt es allgemein erhebliche Vorteile gegenüber den Näherungssensoren. Vorteile sind 
beispielsweise eine größere Reichweite und Blickfeld als Näherungssensoren, die Ermittlung der genauen Entfernung zu Gegenständen und sie 
sind besser zur Erfassung geometrischer Umweltinformationen geeignet. Für die Messung des Abstandes eignet sich die \ac{TOF}-Kamera, die 
mit dem Laufzeitverfahren Distanzen messen und berechnen kann. 
\\ 
Das Laufzeitverfahren funktioniert wie folgt:
\begin{align*}
    \textit{d} = \frac{1}{2}\textit{ct}
\end{align*}
Abstand zwischen der Zielfläche und dem Sensor \textit{d} ist das Produkt aus der Signalgeschwindigkeit \textit{c} und der messbaren Laufzeit 
\textit{t}. Beispiele für solche \acs{TOF}-Kameras und deren Repräsentation sind folgender Abbildung (\ref{pic:tofCam}) zu entnehmen.
\begin{figure}[hbt!]
    \centering
    \includegraphics[width=13cm,height=13cm,keepaspectratio]{2Grundlagen/Bilder/tof_kamera.png}
    \caption{Time-of-Flight Kamera und deren Repräsentation \cite{robotik.2020m}}
    \label{pic:tofCam}
\end{figure}
\\ 
Auf der linken Seite ist eine PMD-Kamera zu sehen, gefolgt von einer SwissRanger-Kamera und auf der rechten Seite eine IFM-Kamera.

\subsection{Verfahren zur Lösung des SLAM Problems}
Das Lösen des \acs{SLAM}-Problems wird in der Praxis mit den angestellten Verfahren, die sich bei der Anwendung duchsetzten, durchgeführt. In 
folgenden Abschnitten werden zwei dieser Verfahren erläutert, sodass ein Grundverständnis vorhanden ist. Zuerst wird auf das Graph-basierte 
\acs{SLAM}-Verfahren eingegangen und darauffolgend die Lösung des Problems mit einem rekursiven Ansatz, dem erweiterten Kalman Filter (\acs{EKF}).
\subsection*{Graph-basiertes SLAM Verfahren}
Grundlegend wird bei dem Algorithmus des Graph-basierten Verfahrens, während der Bewegungsaufzeichnung des Roboters, ein Graph modelliert, 
dessen Positionen an verschiedenen Zeitpunkten durch Knoten dargestellt werden. Alle aufeinander korrespondierenden Positionen, bzw. 
Knoten im Graphen werden über eine Kante verknüpft. Gebunden an die Bewegung des Roboters werden die Kanten modelliert. Der folgende Graph 
(\ref{pic:GraphSLAM}) veranschaulicht ein solches Modell.
\begin{figure}[hbt!]
    \centering
    \includegraphics[width=10cm,height=10cm,keepaspectratio]{2Grundlagen/Bilder/graph_SLAM.png}
    \caption{Graphische Darstellung des SLAM-Verfahrens \cite{graphSLAM.2010}}
    \label{pic:GraphSLAM}
\end{figure}
\\ 
Der Abbildung ist zu entnehmen, dass der Graph auch Kanten darstellt die nicht den folgernden Koten ansprechen, sondern eine 
Position ansprechen die in ihrer Beobachtung auf korrespondierende Knoten zurückgeht. Durch wiederholte Scans können genauere Merkmale und 
Bedingungen festgelegt werden, um die daraus resultierenden Positionen besser differenzieren zu können. Bei Verwendung von Kameras können 
genauer identifizierten Merkmale die geschätzten relativen Orientierungen sein, die durch eine zusätzliche Funktion berechnet werden 
können. Bei der Nutzung eines Lasersensors wird meist der \ac{ICP} Algorithmus angewendet, um die Veränderungen zwischen den Aufnahmepositionen 
wahrzunehmen. \acs{ICP} ist ein iteratives Verfahren, das die korrespondierenden Punkte schätzt und solange transformiert bis diese unter 
einem gewissen Schwellwert liegen. Korrespondierende Punkte sind die, die den kleinsten Abstand zueinander haben (Closest Point). \cite{robotik2.2020m}
\\ 
Der inkrementelle Ansatz des Graph-basierten \acs{SLAM}-Verfahrens ist ursprünglich ein \textit{offline-Verfahren}, welches über die Jahre 
schrittweise zu einem \textit{online-Verfahren} führte. Durch diese ständige Beobachtung und Neuberechnung der aktuellen Position zu jedem 
Zeitpunkt wurde die Bedeutung der inkrementellen Verfahren gesteigert. %Dieser Ansatz ist besonders für monokulare Kameras ratsam und sinnvoll.

\subsection*{Rekursives Schätzproblem}
Der erweiterte Kalman-Filter (\acs{EKF}) ist eine wahrscheinlichkeitsbasierte rekursive Schätzung der Position des Objekts und der Position der 
Landmarken unter Nutzung linearer Bewegungsmuster basierend auf dem Bayes-Filter. Der Bayes-Filter-Algorithmus ist das generelle Verfahren der 
rekursiven Zustandsschätzung und wird über den Satz von Bayes hergeleitet. 
\\
Unter der Voraussetzung einer Normalverteilung und eines linearen Bewegungsmodells, worauf der Kalman-Filter setzt, kann diese durch das 
\acs{SLAM} Problem nicht erfüllt werden. An dieser Stelle wird der erweiterte Kalman-Filter eingesetzt, da dieser nicht-lineare 
Bewegungsmodelle, bzw. Funktionen mittels Taylorreihe approximiert. Somit kann eine Messung in etwas vorhergesagt werden. Diese Schätzung 
dient dann als Grundlage zur eigentlichen Messung. Nach der Berechnung wird der Zustand aktualisiert und ausgegeben. Mit diesem Ergebnis 
werden die darauffolgenden Punkte rekursiv berechnet, bis die Scan-Phase zu Ende ist.

\section{Quaternionen}
\label{chap:Quaternionen} %\cite{quaternionHamilton.2006} 
Quaternionen, lat. \textit{„Menge der Vier“} wurden erstmals 1843 von Sir William Rowan Hamilton beschrieben. Sie beschreiben einen 
Zahlenbereich der die reellen Zahlen erweitert. Die mathematische Formel wird häufig zur Darstellung einer Drehung im dreidimensionalen Raum 
verwendet. Die Theorie der Quaterionen basiert auf der mathematischen Grundlage der komplexen Zahlen, die 1833 von Hamilton als Bildung 
einer Algebra galten und werden daher als hyperkomplexe Zahlen aufgefasst.
\\ 
\linebreak
Mit \textit{a,b,c,d $\in q$} werden Quaternionen wie folgt beschrieben:
\begin{align*}
    \textit{q} = \textit{a} + \textit{b} * \textit{i} + \textit{c} * \textit{j} + \textit{d} * \textit{k}
\end{align*}
In oben genannter Formel ist die Variable \textit{a} der Realteil und (\textit{b,c,d})$^T$ der Imaginärteil \textit{u} der Gleichung. 
Auch geschrieben als \textit{q = (a, u)$^T$}, um die Teile differenzieren zu können. 
\\ 
Mit der sogenannten \textit{Brougham Bridge}-Formel ist es möglich Vektoren zu dividieren. 
\begin{align*}
    \textit{i$^2$} = \textit{j$^2$} = \textit{k$^2$} = \textit{ijk} = \textit{-1} 
    \\
    \textit{ij} = \textit{-ji} = \textit{k} 
    \\
    \textit{jk} = \textit{-kj} = \textit{i} 
    \\ 
    \textit{i$^2$} = \textit{j$^2$} = \textit{j}
\end{align*}
\subsection{Rotation}
%\subsection{Transformation}
Quaternionen zählen als eine gute Möglichkeit Drehungen, bzw. Rotationen darzustellen, unter anderem wegen der reduzierten Anzahl an 
benötigten Parametern im Vergleich zur Berechnung von Rotationsmatrizen und deren Translation und bieten eine deutlich kompaktere 
Darstellung der Rechenwege. Somit können Rotationen intuitiver dargestellt werden, d.h. es sind direkte Angaben von Drehwinkel und 
-achse möglich. Die kompaktere Darstellung beinhaltet lediglich vier Werte im Vergleich zu den neun Werten der Berechnung der Rotationsmatrix
und die Rotation kann um eine bestimmte Achse berechnet werden. Zudem ist eine sehr hohe numerische Stabilität gewährleistet. 
\\ 
Der Drehwinkel wird wie folgt berechnet:
\begin{align*}
    %\textit{a} = $\cos z$ \to \textit{z} = \frac{t}{2} \to t = $\theta$  %\textit{z} = \textit{\frac{$\theta$}{2}} 
    %\\
    \textit{$\theta$} = \textit{2} * \textit{$\arccos a$}
\end{align*}
Der Realteil \textit{a} des Quaternions wird mittels Arkuskosinus berechnet, so erhält man die Drehung des Objekts. Die Drehachse lässt sich 
bestimmen, indem alle Werte des Imaginärteils (\textit{b,c,d})$^T$ mit \textit{u} addiert werden. Die Variable \textit{u} setzt sich wie 
folgt zusammen: 
\begin{align*}
    \textit{u} = \textit{$\sin w$}, \to \textit{w} = \frac{t}{2}, \to \textit{t = $\theta$}
\end{align*}
Mit diesen zwei Berechnungen werden Drehwinkel und Drehachse bestimmt und können für weitere Berechnungen verwendet werden.
\include{2Grundlagen/kapitel2Aa}
%%%%%%%%%%%%%%%%%%%%%%%%%%%%%%%%%%%%%%%%%%%%%%%%%%%%%%%%%%%%%%%%%%%%%%%%%%%%%
%% Descr:       Vorlage für Berichte der DHBW-Karlsruhe, Ein Kapitel
%% Author:      Prof. Dr. Jürgen Vollmer, vollmer@dhbw-karlsruhe.de
%% $Id: kapitel2.tex,v 1.5 2017/10/06 14:02:51 vollmer Exp $
%%  -*- coding: utf-8 -*-
%%%%%%%%%%%%%%%%%%%%%%%%%%%%%%%%%%%%%%%%%%%%%%%%%%%%%%%%%%%%%%%%%%%%%%%%%%%%%%%

\section{OpenGL}
\label{chap:OpenGL}
\subsection{Projektionen}
\subsection{Shader}

\section{Softwarearchitektur}
\label{chap:Softwarearchitektur}
Komponenten in Form von Klassen, Objekten oder Bibliotheken und deren Verbindungen zwischen einzelnen Komponenten beschreibt die Architektur 
eines Softwaresystems. Vielmehr geht es bei der Software-Architektur darum, Anforderungen und deren Zusammenhänge untereinander von dem zu 
konstruierenden System zu beschreiben und nicht einen detaillierten Entwurf vorzulegen. Jedoch hat die Architektur einen enormen Einfluss 
auf die qualitativen und nicht-funktionalen Eigenschaften des daraus resultierenden Systems. 
\\ 
\linebreak
Die Terminologie nach dem \textit{IEEE-Standard 1471-2000} zur Software Architekturbeschreibung, deren Aufgaben und Zweck \cite{swarchitekturieee.2005} 
sind wie folgt definiert: 
\begin{quote}
    Die grundlegende Organisation eines Systems, dargestellt durch dessen Komponenten, deren Beziehungen zueinander und zur Umgebung sowie den 
    Prinzipien, die den Entwurf und die Evolution des Systems bestimmen. \cite{architektursw.2006f}
\end{quote}
Softwarearchitektur bietet viele Möglichkeiten ein System zu entwerfen und Anforderungen und Eigenschaften umzusetzen, daher gibt es auch 
hier viele Ansätze, Lösungen und Abwandlungen, um den Standards und Richtlinien gerecht zu werden. Beispiele dafür sind unter anderem die 
Modulare Software Architektur (\ref{chap:Modulare Software Architektur}) und das Architektur-Entwurfsmuster MVVM (\ref{chap:MVVM}).
\\
Das Buch \textit{Design Patterns - Elements of Reusable Object-Oriented Software} von E. Gamma, R. Helm, R. Johnson und J. Vlissides befasst 
sich mit den verschiedensten Möglichkeiten und Ausprägungen der Softwarearchitektur. 
\\ 
Im Rahmen dieser Ausarbeitung findet keine Aufzählung und Beschreibung verschiedener Arten der Architekturmuster und -stile statt, lediglich die 
für das Projekt verwendeten Muster werden in folgenden Kapiteln aufgegriffen. 
\\ 
\linebreak
Die Abbildung \ref{pic:mvcdiagram} veranschaulicht eine vereinfachte Struktur eines Architekturmusters und deren Komponenten und Zusammenhänge 
zueinander. Die Zeichnung dient zur Veranschaulichung, um darzulegen wie ein Architekturdiagramm aussehen kann, bzw. wie die allgemeine 
Struktur repräsentiert wird.
\begin{figure}[hbt!]
    \centering
    \includegraphics[width=15cm,height=7.5cm,keepaspectratio]{2GrundlagenX/Bilder/MVCArchitecture.png}
    \caption{MVC Architektur Diagramm \cite{mvcbild.2020}}
    \label{pic:mvcdiagram}
\end{figure} 
Im Falle dieser Abbildung wurde das Strukturmuster \ac{MVC} ausgewählt, welches die zu sehenden Komponenten in drei in sich 
geschlossene und unabhängige Fragmente unterteilt die miteinander interagieren. 
\\ 
\linebreak
Neben der Strukturierung von Systemen und Applikationen kann die Modellierung einer Softwarearchitektur auch dabei helfen eine Architektur genauer 
zu beschreiben und zu dokumentieren. Ebenso können Diagramme auch das Management sowie die Planung beeinflussen und verstärken. Die Modellierung 
der Softwarearchitektur stellt somit keinen Selbstzweck dar, sondern bietet einen Mehrwert indem es zur Verständigung, Dokumentation und 
Kommunikation zwischen Entwicklern und Kunden zusätzlich beiträgt. 
\\ 
Durch die Modellierung der Architektur kann frühzeitig eine sinnvolle Evaluierung und Bewertung des Entwurfs durchgeführt werden. Mit diesen 
entstehenden Bewertungen können folgende Schritte besser geplant und umgesetzt werden. 
\\ 
\linebreak
Eine weitere Variante, bzw. Ausprägung der Software-Architektur ist die Modulare Software Architektur (siehe Abschnitt \ref{chap:Modulare Software Architektur}).
\subsection{Modulare Software Architektur}
\label{chap:Modulare Software Architektur}
\subsection{MVVM}
\label{chap:MVVM}
\subsection{Android Architecture Components}
\label{chap:AAC}



\section{Datenmodellierung}
\label{chap:Datenmodellierung}
%%%%%%%%%%%%%%%%%%%%%%%%%%%%%%%%%%%%%%%%%%%%%%%%%%%%%%%%%%%%%%%%%%%%%%%%%%%%%
%% Descr:       Vorlage für Berichte der DHBW-Karlsruhe, Ein Kapitel
%% Author:      Prof. Dr. Jürgen Vollmer, vollmer@dhbw-karlsruhe.de
%% $Id: kapitel2.tex,v 1.5 2017/10/06 14:02:51 vollmer Exp $
%%  -*- coding: utf-8 -*-
%%%%%%%%%%%%%%%%%%%%%%%%%%%%%%%%%%%%%%%%%%%%%%%%%%%%%%%%%%%%%%%%%%%%%%%%%%%%%%%

\chapter{Konzeption}
\label{chap:Konzeption}
In diesem Kapitel wird das erarbeitete Konzept dieser Ausarbeitung dargelegt. Unter dessen die Überlegungen zu einzelnen 
Arbeitsschritten und dem grundsätzlich angedachten Aufbau des Projekts, sowie Entscheidungs- und Beweggründe wieso bestimmte 
Technologien gewählt wurden. Zu Beginn wird auf die Einsatzmöglichkeiten (\ref{chap:Arbeitsumgebung}), in der die Applikation Anwendung 
finden könnte, eingegangen. Anschließend werden die Grundgedanken zu den einzelnen Phasen, Scan-Phase (\ref{chap:Scan-Phase}) und 
Visualisierungs-Phase (\ref{chap:Visualisierungs-Phase}) des Systems erläutert, was sie beinhalten und wie sie funktionstechnisch 
angedacht sind. Mit den zugrundeliegenden Informationen wird auf das Architekturkonzept (\ref{chap:Architekturkonzept}), sowie auf das 
Softwarekonzept (\ref{chap:Architekturkonzept}) eingegangen. Des Weiteren werden die Hintergründe der Wahl des AR-Frameworks 
(\ref{chap:Auswahl des AR Frameworks}) aufgezeigt und abschließend wird noch das konzipierte und prototypische Datenmodell 
(\ref{chap:Datenmodell}) dargelegt.

\section{Arbeitsumgebung / Umfeld}
\label{chap:Arbeitsumgebung}

\section{Objekterkennung / Scan-Phase}
\label{chap:Scan-Phase}
\section{Visualisierungs-Phase}
\label{chap:Visualisierungs-Phase}

\section{Architekturkonzept}
\label{chap:Architekturkonzept}

\section{Softwarekonzept}
\label{chap:Softwarekonzept}

\section{Auswahl des AR Frameworks}
\label{chap:Auswahl des AR Frameworks}
\subsection{Google ARCore}
\subsection{ARToolKit}

\section{Datenmodell}
\label{chap:Datenmodell}

%%%%%%%%%%%%%%%%%%%%%%%%%%%%%%%%%%%%%%%%%%%%%%%%%%%%%%%%%%%%%%%%%%%%%%%%%%%%%
%% Descr:       Vorlage für Berichte der DHBW-Karlsruhe, Ein Kapitel
%% Author:      Prof. Dr. Jürgen Vollmer, vollmer@dhbw-karlsruhe.de
%% $Id: kapitel2.tex,v 1.5 2017/10/06 14:02:51 vollmer Exp $
%%  -*- coding: utf-8 -*-
%%%%%%%%%%%%%%%%%%%%%%%%%%%%%%%%%%%%%%%%%%%%%%%%%%%%%%%%%%%%%%%%%%%%%%%%%%%%%%%

\chapter{Umsetzung}
\label{chap:Umsetzung}
Dieses Kapitel greift die in der Konzeption (\ref{chap:Konzeption}) aufgeführten Planungen und Ideen auf und geht auf Details bei der 
Implementierung ein. Darunter Lösungsansätze, aufgetretene Probleme und deren Behebung. Allgemein die Umsetzung der beiden Kernfunktionen 
sowohl die FrontEnd als auch die BackEnd Aspekte werden in der Implementierung (\ref{chap:implementierung}) aufgezeigt. Abschließend zu 
dem Kapitel wird ein Szenario dargestellt, indem die Anwendung beschrieben wird.

\section{Implementierung}
\label{chap:implementierung}
\subsection*{FrontEnd}
\subsection*{BackEnd}

\subsection{Scan-Phase} %Umgebungserkennung /
\subsection*{FrontEnd}
\subsection*{BackEnd}

\subsection{Visualisierungs-Phase} 
\subsection*{FrontEnd}
\subsection*{BackEnd}

\section{Testdurchlauf / Test-Szenario}
\label{chap:testdurchlauf}
\subsubsection{BackEnd}
Mit Beginn der Backend-Entwicklung wurden in erster Linie die Grundstrukturen implementiert, um der \textit{MVVM}-Architektur gerecht zu werden. Dabei lag 
der Fokus bei der Instanziierung der notwendigen Komponenten der \textit{Android Architecture Components}. Nach chronologischer Reihenfolge wurden die Klassen 
erstellt und mit den dazugehörigen Funktionen und Methoden versehen. Angefangen mit der Entity-Klasse zur Beschreibung des Objekts und deren allgemeiner Aufbau, 
dieser bereits in der Konzeption (\ref{chap:Konzeption}) unter dem Datenmodell (\ref{chap:Datenmodell}) festgehalten wurde. Darauffolgend wurde das „Data Object“ 
erstellt, welches die Zugriffe der Datenbankobjekte verwaltet. Abschließend im Bereich der Datenbank ein Room-Layer, um die eigentliche Datenbank zu 
instanziieren und eine Zugriffsschicht auf diese zu implementieren. 
\\ 
Zur Modularisierung und zur Generierung einer Schnittstelle für den Bezug zu mehreren Datenbeziehungspunkten wurde ein Repository erstellt, das eine saubere 
\acs{API} für den Datenzugriff auf den Rest der Anwendung bietet. Um die vorhandenen Klassen zum Datentransfer und zur Persistierung der Daten mit der 
eigentlichen Benutzeroberfläche zu verbinden, wurde ein ViewModel implementiert, welches die Daten zwischen den einzelnen Komponenten teilt und bereitstellt. 
\\ 
\linebreak 
Nach Aufzählung der einzelnen Bestandteile wird auf diese nun genauer eingegangen. 
\\ 
\linebreak
In einer Java-Klasse wird mithilfe der gegebenen Bibliothek „Room“ ein Entity-Objekt erstellt. Hierbei gibt es eine eindeutige Annotation, die die Klasse und 
deren beinhalteten Variablen deklariert. So wird, dem folgend aufgeführten Code-Beispiel (\ref{code:entity}) zu entnehmen, eine Klasse in eine Entity und somit 
in eine Datenbank-Tabelle konvertiert. In dieser Tabelle, definiert als \textit{„object\_table“}, gibt es weitere Variablen, die mit Annotationen versehen sind. 
Damit werden diese gemäß der Anforderungen des Konzepts definiert. Diese Variablen repräsentieren die Attribute des Datenbankschematas und stellen die einzelnen 
Informations- , bzw. -Datenbankspalten dar. Zur Veranschaulichung, die Initialisierung der \acs{ID}-Vergabe eines Objekts. Diese wird als \textit{„id“}-Spalte 
und ebenso als Primärschlüssel\footnote{Einmaliger und eindeutiger Wert einer Tabelle, bzw. eines Attributs, um dieses eindeutig zu kennzeichnen.}-Variable 
deklariert.
\\ 
\linebreak
\begin{lstlisting}[language=C,
    frame=lines,           % Ein Rahmen um den Code (single for box, lines for top and bottom)
    xleftmargin=\parindent,  % Rahmen link von den Zahlen
    style=algoBericht,
    label={code:entity},
    captionpos=b,           % Caption unter den Code setzen
caption={Entity Code zur Initialisierung der Objekte}]
@Entity(tableName = "object_table")
public class Object {
    @ColumnInfo(name = "id")
    @NonNull
    @PrimaryKey(autoGenerate = true)
    private int id;

    public int getId() { return this.id; }
    public void setId(int id) { this.id = id; }
    ... 
}
\end{lstlisting}
\pagebreak
Das mit der Entity-klasse kommunizierende Modul ist das \textit{„Data Object“}, welches als Interface angelegt wurde. Das „Object Dao“ verwendet ebenso die Bibliothek 
„Room“ und wandelt die Java-Klasse per Annotation in ein „Dao“ um, dieses beinhaltet hauptsächlich die SQL-Queries zur Datenabfrage der vorhandenen Informationen. 
\\
\linebreak
\begin{lstlisting}[language=C,
    frame=lines,           % Ein Rahmen um den Code (single for box, lines for top and bottom)
    xleftmargin=\parindent,  % Rahmen link von den Zahlen
    style=algoBericht,
    label={code:query},
    captionpos=b,           % Caption unter den Code setzen
caption={SQL-Query zur Abfrage der Objekt-Namen}]
@Query("SELECT * FROM object_table ORDER BY name")
LiveData<List<Object>> getObjectName();
\end{lstlisting}
Anschließend nachdem die beiden Klassen erstellt waren, wurde das Datenbank-Layer auf der eigentlichen Datenbank implementiert. Über einen „Builder“ wird die 
Datenbank-Instanz erzeugt und mit einem Namen versehen. Im Falle des Assistenzsystems als \textit{„object\_database“} deklariert. 
\\ 
In zukünftiger Entwicklungen, 
falls notwendig, gäbe es die Möglichkeit in diesem Schema weitere Datenbanken zu erstellen und diese über weitere „Dao“s zu referenzieren. Darüber hinaus ist die 
Möglichkeit gegeben weitere Datenbank-Tabellen zu erstellen, um weitere Objekte und Informationen speichern zu können. Ebenso ist durch ein Repository die Option 
geboten, die derzeit auf dem Smartphone gespeicherte Datenbank auf einen externen Server auszulagern, um so die Daten weitläufiger zur Verfügung zu haben. 
Eine Datenbeziehung von generierten Daten der Maschinen und Geräte selbst wäre unter anderem auch vorstellbar, allerdings müssten diese vorab noch aufbereitet 
und zur Nutzung bereitgestellt werden. Diese Methode wird allerdings nicht näher betrachtet, da diese nicht teil dieser Arbeit ist. 
\\ 
Im Ausblick (\ref{chap:Ausblick}) wird darauf nochmals eingegangen.
\\ 
\linebreak
\begin{lstlisting}[language=C,
    frame=lines,           % Ein Rahmen um den Code (single for box, lines for top and bottom)
    xleftmargin=\parindent,  % Rahmen link von den Zahlen
    style=algoBericht,
    label={code:dblayer},
    captionpos=b,           % Caption unter den Code setzen
caption={Erzeugung des Datenbank-Layers „Room“}]
@Database(entities = {Object.class}, version = 1, exportSchema = false)
public abstract class ObjectRoomDatabase extends RoomDatabase {
    ...
    INSTANCE = Room.databaseBuilder(context.getApplicationContext(),
    ObjectRoomDatabase.class, "object_database")
    .addCallback(sRoomDatabaseCallback)
    .build();
    ...
}
\end{lstlisting}
Letzter wichtiger zu implementierender Aspekt war das Kommunikations-Modul zwischen den Datenbank-Transfers und der Benutzeroberfläche, das ViewModel. 
In dieser Klasse wird eine Liste erstellt, welche den Wert des zu speichernden Objekts besitzt. Diese werden über eine „get“-Funktion aus dem zuvor 
instanziierten Repository geladen und in der erzeugten Liste lokal abgelegt. Diese Liste wird dann für die Präsentation der Informationen auf der 
Nutzeroberfläche verwendet. 
\\ 
\linebreak
Ursprünglich war die Implementierung der eigentlichen Scan-Phase unter Beachtung der ARCore-\acs{API} einfach angedacht, die sich im Laufe der Entwicklung 
allerdings als nicht so praktikabel herauskristallisierte. Diese aufgetretene Problemstellung wird in weiterem Verlauf präzisiert, nachdem der eigentliche 
Verlauf geschildert wurde. 
\\ 
Das Fragment auf dem \acl{UI} der Scan-Phase (\ref{pic:scan}) wird als \acs{AR}-Fragment, unter der Benutzung des Sceneform \acs{SDK}s, deklariert. Basierend auf 
dieser Initialisierung können die Interaktionen mit den ARCore-, bzw. Sceneform- Elementen gewährleistet werden. Dadurch ist ebenso die Nutzung der Kamera 
gegeben. Diese Funktionen sind Bestandteile der ARCore- und Sceneform-\acs{API}. Über einen \textit{„FragmentManager“} wird die „.xml“-Datei mit dem darin 
enthaltenen Fragment der ID \textit{„sceneform\_fragment“} initialisiert.
\\
\begin{lstlisting}[language=C,
    frame=lines,           % Ein Rahmen um den Code (single for box, lines for top and bottom)
    xleftmargin=\parindent,  % Rahmen link von den Zahlen
    style=algoBericht,
    label={code:arfragment},
    captionpos=b,           % Caption unter den Code setzen
caption={Initialisierung des Fragments}]
private ArFragment fragment;
...
fragment = (ArFragment)
getSupportFragmentManager().findFragmentById(R.id.sceneform_fragment);
...
\end{lstlisting}
Um anschließend Objekte auf diesem Fragment einblenden zu können, ist es vorab notwendig eine \textit{„Session“} zu generieren, die unter anderem für die 
Konfigurationen der Kamera zuständig ist. Dadurch wird bei Start der Scan-Phase ein dreidimensionales Koordinatensystem erstellt, welches den Ursprungspunkt der 
Anwendung darstellt. Mit Verwendung der internen Sensoren des Smartphones werden darauffolgende Bewegungen registriert und mittels \acs{SLAM} Verfahren berechnet. 
So ist die Kalkulation der Lokalisierung möglich und anhand der Kamera kann auf Basis des \acs{SLAM} Verfahrens die Umgebung abgebildet werden. Durch diese 
Gegebenheiten wird die virtuelle Karte des Umfelds erzeugt und dient so zur Veranschaulichung der zu platzierenden Objekte. In der Abbildung (\ref{pic:koordin}) 
ist ein Beispiel zu sehen, indem ein Objekte erstellt wird, welches die Koordinaten von dem Ursprungspunkt berechnet. So wird deutlich, wie die Position eines 
Objekts errechnet wird, unter der Verwendung des von ARCore gegebenen \acs{SLAM} Verfahrens. Vorausgesetzt der Anwender startet an der Position (x,y,z = 0), 
bewegt sich frei im Raum und setzt an Postion (x = 5, y = 1, z = 3.5) ein Objekte.
\begin{figure}[hbt!]
    \centering
    \includegraphics[width=10cm,height=10cm,keepaspectratio]{4Umsetzung/Bilder/koordin.jpeg}
    \caption{Aufbau der Positionsberechnung von Objekten}
    \label{pic:koordin}
\end{figure}
\\ 
Beim setzen eines Objekts wird als Anhaltspunkt der Mittelpunkt des Bildschirms berechnet, um das Objekt auf dem Bidlschrim zentriert platzieren zu können. Ist 
diese Position berechnet, folgt durch Hilfe des \acs{SLAM} Verfahrens die genaue Ermittlung der virtuelle Position. Anhand der Information dieser Position 
ist das Objekt zu platzieren. 
\\ 
Das Objekte wird gesetzt, indem der Nutzer in der \acs{UI} (\ref{pic:scan}) im Bereich der „Gallery“ ein Objekt anklickt.
Über die der Sceneform-\acs{API} zur Verfügung gestellte Methode („ModelRenderable“) wird das Objekt erstellt und als „Anchor“ auf die berechnete Position 
gesetzt. Ein Anchor ist die fixe Position des Objekts, an dem dies platziert wird und so lange dort vorhanden bleibt, bis die Applikation beendet wird. Die Datei 
die als dreidimensionales asset generiert werden soll, wird über die „uri“-Variable lokalisiert und als Parameter übergeben. 
\begin{lstlisting}[language=C,
    frame=lines,           % Ein Rahmen um den Code (single for box, lines for top and bottom)
    xleftmargin=\parindent,  % Rahmen link von den Zahlen
    style=algoBericht,
    label={code:modelrenderable},
    captionpos=b,           % Caption unter den Code setzen
caption={ModelRenderable Builder}]
ModelRenderable.builder()
.setSource(owner.get(), uri)
.build()
.handle((renderable, throwable) -> {
    ...
}
\end{lstlisting}
Während der Objekt-Renderung öffnet sich die \acs{GUI} (\ref{pic:createObject}) zur Eingabe der Informationen, die sich auf das Objekt beziehen. Diese 
werden dann zusammen mit der virtuellen Position des Objekts abgegriffen und in die Datenbank geschrieben. Drückt der Anwender den „save“-Button der Oberfläche 
(\ref{pic:createObject}) beendet er diesen Vorgang und kehrt auf die \acs{UI} der Scan-Phase zurück, um weiter Objekte platzieren zu können. 
\\ 
Zu der Position des Objekts wird dessen Rotation durch die einfache Berechnung durch Quaternionen ebenso in die Datenbank gespeichert, um anhand der Berechnung 
die Darstellung so real wie möglich wiederzugeben. Damit erfolgt die Feststellung, dass das Objekt immer in Blickrichtung der Kamera positioniert ist, da die 
Rotation ausgehen von der Kamerahaltung berechnet wird. 
%--> Problemschilderung der Speicherung der Session
\\ 
\linebreak
Ursprünglich war geplant, die generierte Session sowie die darin erstellten Objekte in der Datenbank abzuspeichern, um diese bei erneutem Aufruf der 
Scan-Phase, bzw. bei Anwendung der Visualisierungs-Phase zu laden und innerhalb dieser Session wieder anzeigen zu lassen. Da die Datenbank nur gewisse Datentypen 
zulässt, war das erste auftretende Problem die Speicherung der Session als Objekt, welches zunächst in eine „BLOB“-Datei konvertiert wurde, um diese speichern 
zu können. Eine „BLOB“-Datei ist ein Binary Large Object, welches anhand der Binärcodierung abgespeichert wird. Dabei kann es sich unter anderm um große 
Bild- oder Audio-Dateien handeln, auch können so anderweitig große Dateien gespeichert werden. Mit dieser Umsetzung tat sich auch schon das nächste Problem auf, 
welches deutlich schwerwiegender war und auch in diesem Sinne die Visualisierungs-Phase betraf. 
\\ 
Da eine erzeugte Session nur eine gewissen Zeit verwendet wird, ist diese für den weitern Gebrauch nicht vorgesehen, dies bedeutet, dass nach erneutem Starten 
des Assistenzsystems alle zuvor erzeugten Objekte zwar auf die dazugehörige Session bestimmt sind, allerdings auf der zugewiesenen Position 
nicht mehr erreichbar wären. Durch die Unterschiedlichen Startpunkte, die bei wiederholtem Starten der Anwendung erzeugt worden würden, 
würde sich die ursprüngliche Ausgangsposition der Applikation verschieben und so auch die Objekte an eine fälschliche Position projizieren. Um genau zu sein, 
wären die Objekte dann ausgehend von dem neu erzeugten Startpunkt zu referenzieren. Somit wäre das 
Ergebnis nicht exakt und könnte keine Anwendung finden, da die Applikation Informationen anzeigen würde, die der Realität nicht entsprechen. 
\\ 
Basierend auf dieser Erkenntnis musste umdisponiert und ein neuer Lösungsansatz konzipiert werden. Mit dem Wissen über den aktuellen Stand der ARCore \acs{API} 
war es die neue Aufgabe eine Lösung zu entwickeln, welche exakt und zuverlässig arbeitet, um Informationen nicht inkorrekt zu interpretieren. 
\\ 
Dieser Ansatz wird nun erläutert.
\\ 
\linebreak
Die Idee war es, einen Fixpunkt zu erstellen, welcher dazu dient, einen immer gleichbleibenden Startpunkt vorzugeben. Dadurch gäbe es keine Unterschiede des 
Ausgangspunktes mehr und der Nutzer könnte trotz seiner aktuellen Position starten. Dazu muss er beim Start der Applikation an den Ursprungspunkt kehren, 
um daraufhin die Funktion zu starten. Für diesen Ansatz wurde ein Marker gewählt (siehe Abbildung \ref{pic:initialMarker}), der vor dem ersten Gebrauch 
der Software an einer Position angebracht wird und an dieser dauerhaft bestehen bleibt. Somit ist ein immer gleichbleibender Ausgangspunkt geschaffen, von dem aus 
die Objekte referenziert werden können.
\begin{figure}[hbt!]
    \centering
    \includegraphics[width=5cm,height=5cm,keepaspectratio]{4Umsetzung/Bilder/cjt_logo_tracking.png}
    \caption{Marker zur Erkennung der Ausgangsposition}
    \label{pic:initialMarker}
\end{figure}
\\
Um diesen Ansatz umsetzen zu können, war es notwendig eine weitere Funktion zur Applikation hinzuzufügen (siehe Abbildung \ref{pic:image_tracking} 
in Scan-Phase \ref{chap:scan_implementation} Frontend), damit ein Marker verfolgt werden kann. Wird dieser Marker erkannt, folgt die eigentliche Scan-Phase, 
die nach der Umkonzeptionierung sowohl bei der Objektplatzierung als auch bei der Datenspeicherung ebenso überarbeitet wurde. 
\\ 
\linebreak
Angenommen der Nutzer starten die Applikation an einer beliebigen Position im Raum und platziert Objekte an von ihm vorgesehenen Stellen, dann würden die 
Objekte erstellt, aber keinerlei Anhaltspunkte geschaffen werden, um bei nachzutragenden Objekten die gleichen Voraussetzungen des zu referenzierenden Standpunktes 
zu erfüllen. Daher wird bei der Scan-Phase die Methode zur Markererkennung eingebaut, um einen initialen Fixpunkt zu erstellen. Dadurch ist es gewährleistet, 
ausgehend von diesem Punkt erneut Objekte zu platzieren, egal wo im Raum die Applikation gestartet wurde. Demnach werden die Objekte immer zu so erstellt, dass sie 
abhängig zur Position des Fixpunktes gespeichert und angezeigt werden. Dies bedeutet, die Objekte sind immer eine Referenz zu dem initialen Marker. Auch wenn 
ein Objekt nachträglich hinzugefügt werden soll, ist es lediglich vom Nutzer erforderlich den Marker einzuscannen und danach an die gewünschte Position, an der 
das Objekt platziert werden soll, zu laufen und das Objekt erstellen. Demnach wird auch beim Schreibbefehl, in die Datenbank, lediglich die Differenz, bzw. der 
Abstand zu dem initialen Marker gespeichert, um immerzu diese Position als Ursprungspunkt zu nehmen. Die Berechnung der Distanz zwischen Objekt und 
Ursprungsmarker erfolgt durch eine präzise Subtraktion, bei der lediglich die Ursprungskoordinaten von den aktuellen Positionskoordinaten subtrahiert werden. 
Dieser Vorgang ist dem folgenden Code-Beispiel (\ref{chap:AAC}) zu entnehmen. Dabei wird das aktuelle Objekt „object“ und das initiale Objekt „initialObject“ 
als Parameter übergeben, die sogenannte „call by Reference“, um die Distanz berechnen zu können. Das Ergebnis wird in mehreren lokalen Variablen 
zwischengespeichert und zum Schluss in eine neue Position „Pose“ mit Translation und Rotation, von der Berechnung durch Quaternionen, ermittelt. Der Rückgabewerte 
dieser Funktion ist demnach eine „Pose“, die nach Aufruf der Methode übergeben und so in der Datenbank, mittels „insert“-Befehl, hinterlegt wird. Mit dem 
Resultat ist die Position und Rotation des einzelnen Objekts gegeben und kann in der Visualisierungs-Phase verwendet werden.
\\
\linebreak
\begin{lstlisting}[language=C,
    frame=lines,           % Ein Rahmen um den Code (single for box, lines for top and bottom)
    xleftmargin=\parindent,  % Rahmen link von den Zahlen
    style=algoBericht,
    label={code:differencetoinitial},
    captionpos=b,           % Caption unter den Code setzen
caption={Berechnung der Distanz zwischen Marker und Ursprungspunkt}]
public Pose returnValueFromPosition(Object object, Object initialObject){
    float tX = object.getTx() - initialObject.getTx();
    float tY = object.getTy() - initialObject.getTy();
    float tZ = object.getTz() - initialObject.getTz();

    float qX = object.getQx() - initialObject.getQx();
    float qY = object.getQy() - initialObject.getQy();
    float qZ = object.getQz() - initialObject.getQz();
    float qW = object.getQw() - initialObject.getQw();

    float[] rotation = {qX,qY,qZ,qW};
    float[] translation = {tX,tY,tZ};

    return new Pose(translation, rotation);
}
\end{lstlisting}
Zur Veranschaulichung der zuvor beschriebenen Methodik dient die Abbildung (\ref{pic:differenztoinitial}). Durch diese Skizze wird nochmals verdeutlicht, dass 
sich alles in einem erstellten Koordinatensystem abspielt und lediglich die Distanz der einzelnen Objekte zum Marker berechnet und in der Datenbank persistiert 
werden.
\begin{figure}[hbt!]
    \centering
    \includegraphics[width=15cm,height=15cm,keepaspectratio]{4Umsetzung/Bilder/difcalc.jpeg}
    \caption{Positionsberechnung zum Ursprungsmarker}
    \label{pic:differenztoinitial}
\end{figure}
\pagebreak
\\ 
\linebreak
Nachdem die Scan-Phase nun vollends beschrieben wurde, geht es in folgendem Kapitel um die Erläuterung des dritten und letzten Use Cases. Dabei werden sowohl 
die Frontend als auch die Backend Aspekte aufgezeigt. 
\subsection{Visualisierungs-Phase} 
Nach vollständiger Beendigung der Scan-Phase ging es in der Entwicklung weiter mit der Visualisierungs-Phase. Die zuvor gespeicherten Objekte und 
deren Informationen werden dabei abgerufen und können jederzeit erneut visualisiert und im Raum platziert werden. Die folgende Darlegung der 
Frontend-Entwicklung der Visualisierungs-Phase gibt einen Einblick auf die Benutzeroberfläche dieser Phase und wie diese aufgebaut ist. Dadurch sind die 
zugrundeliegenden Informationen veranschaulicht, welche für die nachstehende Backend-Implementierung von Relevanz sind.   
\subsubsection{FrontEnd}
Bei der Visualisierungs-Phase gibt es prinzipiell eine Benutzeroberfläche, die diesen Use Case ausmacht. Unter eigentlicher Anwendung dieser Phase und 
deren \acs{GUI} befinden sich darüber hinaus weitere Benutzeroberflächen, die bereits in der Scan-Phase schon erläutert wurden. Darunter zum Beispiel 
die in Abbildung (\ref{pic:image_tracking}) zu entnehmende Markererkennungs-\acs{UI}, die ebenso Bestandteil der Visualisierungs-Phase ist. 
\\ 
Die primäre \acs{GUI} ist ähnlich zu der Benutzeroberfläche der Scan-Phase aufgebaut. Ein Fragment, welches das Livebild der Kamera repräsentiert, das 
sich über den ganzen Bildschirm erstreckt. Dadurch können die bereits in der Scan-Phase erstellten Objekte erneut angezeigt werden. Des Weiteren befinden 
sich auf dieser Oberfläche zwei Buttons, zum einen zur Navigation, um auf das Startmenü zu gelangen und zum anderen, um die Objekte von der Datenbank 
abzugreifen, rendern und auf dem Bildschirm anzeigen zu lassen. Diese befinden sich jeweils in der linken und rechten unteren Ecke des Bildschirms, wie 
der Abbildung \ref{pic:visual} zu entnehmen ist. 
\\ 
Da diese \acs{UI} nur als Ansicht der zu visualisierenden Objekte gedacht ist, stehen dem Nutzer nur eingeschränkt Interaktionen mit der Oberfläche zur 
Verfügung. Diese belaufen sich lediglich auf die zuvor genannten Buttons und die dynamisch erzeugten Schaltflächen der Objekte, um zu den einzelnen 
Darbietungen der Objekte mehrere Informationen zu erlangen. 
\begin{figure}[hbt!]
    \centering
    \includegraphics[width=10cm,height=7.5cm,keepaspectratio]{4Umsetzung/Bilder/visual-phase.jpg}
    \caption{Visualisierungs-Phase der Applikation}
    \label{pic:visual}
\end{figure} 
\\
Nun erfolgt die Beschreibung der Backend-Implementierung und Einblicke in die Programmierung der Visualisierungs-Phase werden gewährleistet. 
\subsubsection{BackEnd}
Nachdem das Problem der variablen Startposition in der Implementierung der Scan-Phase umgangen war, konnte die Visualisierungs-Phase entwickelt werden. 
\\ 
\linebreak
Wählt der Anwender auf der Startmenü-\acs{UI} (\ref{pic:startmenu}) den Button „Virtual-Phase“, wird eine Anmerkung getätigt, indem der Nutzer darüber in 
Kenntnis gesetzt wird, dass er zu Beginn die Scan-Phase durchgeführt haben muss, bevor die Visualisierungs-Phase gestartet werden kann. Diesen Vermerk kann der 
Benutzer ignorieren, da dieser weiß, dass ein Scan durchgeführt wurde, oder er akzeptiert den Hinweis und wird auf die dementsprechende Funktion weitergeleitet. 
Dies hat zur Folge, dass der Nutzer davor bewahrt wird eine Funktion starten zu wollen, die ohne Inhalt keine Änderungen aufzeigt. Damit werden folgende 
Verwunderung und Verwirrungen dem Anwender gegenüber erspart.
\\ 
Nachdem wird der Nutzer aufgefordert den initialen Marker (\ref{pic:initialMarker}) einzuscannen, um die ursprüngliche Position, bzw. die Markierung, in Erwägung 
zu bringen. Damit kann der weitere Programmablauf folgen, indem die \acs{GUI} der Visualisierungs-Phase (\ref{pic:visual}) angezeigt und eine neue Session zur 
Objektplatzierung gestartet wird. 
\\
Über eine ViewModel-Komponente werden die Objekte aus der Datenbank abgefragt und in einer Liste gespeichert, um diese zu späterem Zeitpunkt zur Platzierung der 
Objekte verwenden zu können. 
\\ 
Ist der Marker erkannt worden, wird mithilfe des Buttons, der sich im rechten unteren Eck der Benutzeroberfläche (\ref{pic:visual}) befindet, die Methode zur 
Datenabfrage, Erzeugung und Platzierung der Objekte gestartet. Dabei wird über den Button-Klick eine Funktion aufgerufen, die die Liste der Objekte anspricht und 
die jeweils darin enthaltenen Objekte nacheinander erstellt. Nach betätigen des Buttons ist dieser auf der \acs{UI} nicht mehr aufzufinden, um einer weiteren 
Platzierung aller Objekte vorzubeugen. Die von dem ViewModel gegeben Liste wird, abhängig von der Länge der Liste iterativ abgearbeitet und die Objekte 
nacheinander erstellt. Dies erfolgt über eine „for“-Schleife, welche abhängig von dem Ursprungspunkt, dem Marker „objectMarkerCenter“, die Position des 
aktuellen Objekts „obj“, an der Stelle „i“ in der Liste, neu errechnet und darstellt.
\\ 
\linebreak
\begin{lstlisting}[language=C,
    frame=lines,           % Ein Rahmen um den Code (single for box, lines for top and bottom)
    xleftmargin=\parindent,  % Rahmen link von den Zahlen
    style=algoBericht,
    label={code:listOfObjects},
    captionpos=b,           % Caption unter den Code setzen
caption={Abfolge der Objekte in der Liste}]
... 
for (int i = 1; i < listOfObj.size(); i++){
    obj = listOfObj.get(i);
    if(!obj.getName().equals("origin")) {
        calculateObjectPosition(obj, i, objectMarkerCenter);
    }
}
\end{lstlisting}
Mit der gegebenen „calculateObjectPosition“ wird dann die Distanz vom Ursprungsmarker zu dem Objekt addiert, um 
die Positionen und Distanzen der einzelnen Objekte der Scan-Phase exakt wiederzugeben. Ist die Addition zu einem Objekt durchgeführt, repräsentiert diese 
Berechnung die neue Position, an der das Objekt platziert werden muss, unabhängig von der Startposition des Anwenders. Somit ist der Nutzer bei Verwendung der 
Applikation in der Lage aus einem vorab unbestimmten Punkt zu starten und alle Objekte exakt auf die referenzierte, ursprüngliche Position zu setzen, die 
in der Scan-Phase festgelegt wurde. Nachdem die Position des aktuellen Objekts berechnet wurde, wird diese als neue „Pose“ initialisiert und übergeben, um 
das zu erstellende Objekt zu rendern und auf der Oberfläche anzuzeigen. Dabei findet eine Abfrage statt, um welchen Status es sich handelt, um daraufhin 
entscheiden zu können in welcher Farbe das Objekt zu Präsentieren ist.  
\\ 
\linebreak
\begin{lstlisting}[language=C,
    frame=lines,           % Ein Rahmen um den Code (single for box, lines for top and bottom)
    xleftmargin=\parindent,  % Rahmen link von den Zahlen
    style=algoBericht,
    label={code:additionOfObject},
    captionpos=b,           % Caption unter den Code setzen
caption={Berechnung der Markerplatzierung}]
public void calculateObjectPosition(Object object, int i, Object objOrigin){
    ... 
    qx = object.getQx() + objOrigin.getQx();
    qy = object.getQy() + objOrigin.getQy();
    qw = object.getQw() + objOrigin.getQw();
    qz = object.getQz() + objOrigin.getQz();

    tx = object.getTx() + objOrigin.getTx();
    ty = object.getTy() + objOrigin.getTy();
    tz = object.getTz() + objOrigin.getTz();

    float[] rotation = {qx,qy,qz,qw};
    float[] translation = {tx,ty,tz};

    Pose pose = new Pose(translation, rotation);

    if(objectViewModel.getAllObjectsName().getValue().get(i).getState() == 1) {
        addObject(Uri.parse("object.sfb"), pose, i);
    }

    if(objectViewModel.getAllObjectsName().getValue().get(i).getState() == -1){
        addObject(Uri.parse("object_red.sfb"), pose, i);
    }
}
\end{lstlisting}
Sind alle Listenelemente abgearbeitet, ist die Erstellung und Renderung der Objekte fertig. Danach werden dem Nutzer alle vorhandenen Objekte angezeigt und 
visualisiert. Mit diesen Erkenntnissen kann der Nutzer durch den Raum laufen und alle Objekte überwachen. Wird ein Objekt anvisiert, kann über eine Berührung 
des Objekts alle vorhandenen Informationen eingesehen werden. Somit sind über kurzen Zeitraum die Informationen des einzelnen Objekts sichtbar.  
Die Applikation kann beliebig oft geschlossen und neu gestartet werden. Der Nutzer kann sich zu jeder Zeit die Objekte visualisieren lassen, vorausgesetzt, der 
Ursprungsmarker wird zu Anfang eingescannt. Daraufhin können alle Objekte präsentiert und die dazugehörigen Informationen angezeigt werden. %  zu jeder Zeit
\\ 
\linebreak
Um die soeben beschriebene Backend-Implementierung der Visualisierungs-Phase zu demonstrieren, wird diese anhand eines kleinen Szenarios veranschaulicht. 
\\ 
Startet der Nutzer die Visualisierungs-Phase, muss er den initialen Marker scannen, um fortfahren zu können. Darauffolgend öffnet sich die \acs{GUI} dieser 
Phase und mit einem Klick auf den Button rechts unten auf der Oberfläche, synchronisiert er die Funktion und alle Objekte, die in der Scan-Phase gesetzt wurden, 
werden erneut generiert. Daraufhin kann der Nutzer die Objekte einsehen, an den Stellen an denen sich die Objekte ursprünglich befinden. So werden diese samt Status 
angezeigt und der Nutzer hat vollen Überblick auf alle Objekte und kann diese beobachten. Dieser Schritt ist der Abbildung (\ref{pic:visual_objects}) 
zu entnehmen. 
\begin{figure}[hbt!]
    \centering
    \includegraphics[width=10cm,height=7.5cm,keepaspectratio]{4Umsetzung/Bilder/show_objects_after_loading.jpg}
    \caption{Funktion Visualisierungs-Phase Teil 1}
    \label{pic:visual_objects}
\end{figure}
Eine weiter Funktion dieser Phase ist die Verwaltung der Objekte und deren Informationen. Wird ein Objekt anvisiert, kann über einen Klick auf das Objekt 
deren Informationen über ein PopUp angezeigt werden. Dieses beinhaltet den Namen des Objekts, den Status und die ID, die zu Anfang automatisch generiert 
und vergeben wurde. So kann zu jedem einzelnen Objekt, ohne großen Aufwand, dessen Informationen zur Verfügung gestellt werden. 
\begin{figure}[hbt!]
    \centering
    \subfigure[Anzeige der Objektinformationen 1]{\includegraphics[width=10cm,height=7.5cm,keepaspectratio]{4Umsetzung//Bilder/show_data_to_obj_1.jpg}}
    \subfigure[Anzeige der Objektinformationen 2]{\includegraphics[width=10cm,height=7.5cm,keepaspectratio]{4Umsetzung/Bilder/show_data_to_obj_2.jpg}}
    \subfigure[Anzeige der Objektinformationen 3]{\includegraphics[width=10cm,height=7.5cm,keepaspectratio]{4Umsetzung/Bilder/show_data_to_obj_3.jpg}}
    \caption{Funktion Visualisierungs-Phase Teil 2}
    \label{pic:showdatatoobj}
\end{figure}
\pagebreak 
\\ 
\linebreak
Nachdem die einzelnen Aspekte der Front- und Backend-Entwicklung der drei Use Cases geschildert und so erfolgreich Einblicke in die Entwicklung, in 
aufgetretenen Hindernissen und Problemstellungen gewährleistet wurden, geht es nun mit der Evaluierung des Systems, dem Fazit und dem Ausblick weiter.

%werden die implementierten Funktionen anhand eines Testdurchlaufs nochmals genauer 
%dargelegt.
%\section{Testdurchlauf} %/ Test-Szenario
%\label{chap:testdurchlauf}
%Ergänzend zur Implementierung wird in diesem Abschnitt das Resultat und die Anwendung der Applikation anhand eines unabhängigen Tests praktiziert, um die 
%Erzeugnisse der Entwicklung darzustellen und gegebenenfalls erweiternd zur eigentlichen Entwicklungsbeschreibung Ergebnisse zu präsentieren.
\chapter{Fazit}
\label{chap:Fazit}

\chapter{Ausblick}
\label{chap:Ausblick}



% Ab hier beginnt der Anhang
\appendix
\addcontentsline{toc}{chapter}{Anhang}

\addcontentsline{toc}{chapter}{Index}
\printindex

\addcontentsline{toc}{chapter}{Literaturverzeichnis}

% Haben Sie das "biblatex"-Paket nicht installiert, benutzen Sie folgendes:
% Ohne das "biblatex"-Paket (s. bericht.sty) produziert folgendes
% "deutsche" Zitate in Literaturverzeichnissen gemaß der Norm DIN 1505,
% Teil 2 vom Jan. 1984.
% Die Zitatmarken werden alphabetisch nach Verfassern
% sortiert und sind durch abgekürzte Verfasserbuchstaben plus
% Erscheinungsjahr in eckigen Klammern gekennzeichnet.

% \bibliographystyle{alphadin}
% \bibliography{bericht}

%%%%%%%%%%%%%%%%%%%%%%%%%%%%%%%%%%%%%%%5
% BIBLATEX
% Benutzt man das "biblatex"-Paket, muß man folgendes schreiben:
\def\refname{Literaturverzeichnis}
\printbibliography
%%%%%%%%%%%%%%%%%%%%%%%%%%%%%%%%%%%%%%%5
%%%%%%%%%%%%%%%%%%%%%%%%%%%%%%%%%%%%%%%


%\newpage
%\addcontentsline{toc}{chapter}{Liste der ToDo's}
%\listoftodos[Liste der ToDo's]


\end{document}